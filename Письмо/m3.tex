\subsection{Подсистема остаток номенклатуры} \hypertarget{3_1}{Подсистема остаток номенклатуры}
\subsubsection{Описание подсистемы остатков номенклатуры}
Согласно озвученному необходимо разделить номенклатуру на три номенклатурные группы А, В, С.
\begin{displayquote}
	\textit{"группу С убрать в отд группу назвать «группа С» - до продать, не заказывать"}
\end{displayquote} 
\begin{enumerate}	
	\item Анализа остатков номенклатуры исходя из "не снижаемого остатка товара"  (см.\hyperlink{3_1}{\textit {"Подсистема остаток номенклатуры"}})
\end{enumerate}
\subsubsection{Реализация подсистемы Цены поставщиков}
\paragraph{Добавления в системе:}
%\begin{enumerate}	
% \blindtext Документ «Установка договорных цен поставщика» (Документ позволяет зафиксировать цену поставщика на позицию номенклатуры на определенную дату)
%	\item Периодический регистр сведений «Цены поставщика», хранящий цены в разрезе поставщика и магазина.
%\end{enumerate}
\paragraph{Изменения в системе:}
\begin{enumerate}	
	\item 
	\item 
	\item 
\end{enumerate}