\section{Продажи}
%\marginnote{\Date{Ср.}{08}{Апр.}{2020}}[-20pt]
\subsection{Отчет о розничных продажах}


\begin{itemize}
	\item При открытии документа выполняется проверка по регистру "крюПраваПользователей", есть ли у текущего пользователя права на изменение документа, если нет то табличная часть <<Товары>> блокируется на изменения (реквизит <<ТолькоПросмотр>> устанавливается в Истину) (расш.)
	\Nameref{503}  
	\item Заблокирована возможность проведения документа в Центральном узле.
	\item При проведении документа создаются документы "Сборка товаров".(описание)\Nameref{1004} 
	\item При включенной константе <<КрюОпросКассираПриОкончанииСмены>> выполняется проверка на совпадение суммы при опросе кассира и суммы документа
	\item В подвале документа добавлен реквизит Наличные в группе "ГруппаСуммы"
	\item На форме документа предусмотрен механизм позволяющий получить количество чеков в ОРП
	\item В форме списка переопределено условное оформление поле "Дата" ("ПриСозданииНаСервере")
	
	
\end{itemize}

\subsubsection{Отчет о розничных продажах дополнительные механизмы}
\begin{itemize}
	\item В модуле менеджера изменен запрос по акцизным марка для ускорения выполнения (ТекстЗапросаАкцизныеМарки2)
\end{itemize}



\subsection{Кассовая смена}

\begin{itemize}
	\item [--]
\end{itemize}



\subsection{Возврат товаров от покупателя}
\begin{itemize}
	\item При открытии документа корректно заполняется значение глобальной переменной НомерДокументаКассыККМ
	 с учетом нашего разделения на две кассы. 
\item Исправлена процедура печати чека с учетом двух касс

\end{itemize}


