\section{Продажи}
%\marginnote{\Date{Ср.}{08}{Апр.}{2020}}[-20pt]
\subsection{Отчет о розничных продажах}


\begin{itemize}
	\item При открытии документа выполняется проверка по регистру "крюПраваПользователей", есть ли у текущего пользователя права на изменение документа, если нет то табличная часть <<Товары>> блокируется на изменения (реквизит <<ТолькоПросмотр>> устанавливается в Истину) (расш.)
	\Nameref{503}
	\item Заблокирована возможность проведения документа в Центральном узле.
	\item При проведении документа создаются документы "Сборка товаров".(описание)\Nameref{1004}
	\item При включенной константе <<КрюОпросКассираПриОкончанииСмены>> выполняется проверка на совпадение суммы при опросе кассира и суммы документа
	\item В подвале документа добавлен реквизит Наличные в группе "ГруппаСуммы" (механизм не работает, рассмотреть вопрос об отключении)
	\item При установленной константе <<КрюОпросКассираПриОкончанииСмены>>,  на форме документа предусмотрен реквизит показывающий количество чеков в документе в момент закрытия смены. Также механизм позволяющий при нажатии на кнопку \keys{Определить текущее количество чеков} получить текущее количество чеков в <<ОРП>> (это работает независимо от установленной константы <<КрюОпросКассираПриОкончанииСмены>>)
	\item В форме списка переопределено условное оформление поле "Дата" ("ПриСозданииНаСервере"), добавлено время документа.


\end{itemize}

\subsubsection{Отчет о розничных продажах дополнительные механизмы}
\begin{itemize}
	\item В модуле менеджера изменен запрос по акцизным марка для ускорения выполнения (ТекстЗапросаАкцизныеМарки2)
\end{itemize}



\subsection{Кассовая смена}

\begin{itemize}
	\item [--]
\end{itemize}



\subsection{Возврат товаров от покупателя}
\begin{itemize}
	\item При открытии документа корректно заполняется значение глобальной переменной НомерДокументаКассыККМ
	 с учетом нашего разделения на две кассы.
\item Исправлена процедура печати чека с учетом двух касс

\end{itemize}


