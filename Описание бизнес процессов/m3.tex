\section{Склад}
%\marginnote{\Date{Ср.}{08}{Апр.}{2020}}[-20pt]
\subsection{Оприходование товаров}


\begin{itemize}
	\item Перед записью выполняется проверка, что документ создан на основании пересчета, тогда выставляется признак выгрузки в бухгалтерию.(расш.)\Nameref{502}
	\item Если табличная часть <<Товары>> содержит алкогольную продукцию, а документ не имеет в основании Инвентаризацию, то блокируется проведение документа.
	(расш.)
	\Nameref{500}
	\item При установленной константе <<крюПроверятьПомеченнуюНаУдалениеВДокументах>>
	документ не должен проводиться, если в табличной части <<Товары>> присутствует
	номенклатура помеченная на удаление.(описание)\Nameref{1005}
	\item При проведении документ должен делать движения по регистру "СебестоимостьНоменклатуры"(описание)\Nameref{1002}
	\item При установленной константе "крюКонтролироватьВозвратнуюТару" и наличии в табличной части <<Товары>> номенклатуры
	с видом "Возвратная тара" или "Оборудование" проведение документа блокируется (описание)\Nameref{1003}
\end{itemize}

\subsection{Списание товаров}

\begin{itemize}
	\item При установленной константе <<крюПроверятьПомеченнуюНаУдалениеВДокументах>>
	документ не должен проводиться, если в табличной части <<Товары>> присутствует
	номенклатура помеченная на удаление.(описание)\Nameref{1005}
	\item При установленной константе "крюКонтролироватьВозвратнуюТару" и наличии в табличной части <<Товары>> номенклатуры
	с видом "Возвратная тара" или "Оборудование" проведение документа блокируется (описание)\Nameref{1003}

\end{itemize}



\subsection{Пересчет товаров}
\begin{itemize}
	\item [--]
\end{itemize}


\subsection{Сборка товаров}
\begin{itemize}
	\item Документ создается при проведении ОРП (описание механизма сборки товаров)\Nameref{1004}
	\item При проведении документ должен делать движения по регистру "СебестоимостьНоменклатуры" (описание)\Nameref{1002}
\end{itemize}



\subsection{Приходный ордер на товары}
\begin{itemize}
	\item При установленной константе <<крюПроверятьПомеченнуюНаУдалениеВДокументах>>
	документ не должен проводиться, если в табличной части <<Товары>> присутствует
	номенклатура помеченная на удаление.(описание)\Nameref{1005}
	\item При установленной константе "крюКонтролироватьВозвратнуюТару" и наличии в табличной части <<Товары>> номенклатуры
	с видом "Возвратная тара" или "Оборудование" проведение документа блокируется (описание)\Nameref{1003}
\end{itemize}


\subsection{Расходный ордер на товары}
\begin{itemize}
	\item При установленной константе <<крюПроверятьПомеченнуюНаУдалениеВДокументах>>
	документ не должен проводиться, если в табличной части <<Товары>> присутствует
	номенклатура помеченная на удаление.(описание)\Nameref{1005}
	\item При установленной константе "крюКонтролироватьВозвратнуюТару" и наличии в табличной части <<Товары>> номенклатуры
	с видом "Возвратная тара" или "Оборудование" проведение документа блокируется (описание)\Nameref{1003}
\end{itemize}


\subsection{Акт постановки на баланс}
\begin{itemize}
	\item На форму списка добавлена кнопка \keys{Печать}
	\item В модуле менеджера определены процедуры печати
	\item При проведении документ делает движения по регистру "крюОстаткиАлкогольнойПродукцииТоргЗалЕГАИС"
	(описание)\Nameref{1009}
\end{itemize}




\subsection{Акт списания}
\begin{itemize}
	\item Изменен вызов формы подбора алкогольной продукции, форма должна открываться с отбором по текущей номенклатуре и характеристике(?)
	(расш.)
	\Nameref{501}
	\item В документ добавлена кнопка \keys{Перезаполнить на основании ОРП} , позволяющая перезаполнить документ в случае корректировок ОРП(описание)\Nameref{1010}
	\item Добавлена процедура и форма печати "Акта списания"
	\item При проведении документ делает движения по регистру "крюОстаткиАлкогольнойПродукцииТоргЗалЕГАИС"
	(описание)\Nameref{1009}
	\item При закрытии кассовой смены документы Акт списания создаются с датой на начало смены
	\item При установленной константе "крюОтправлятьАктыЕГАИСПриЗакрытииСмены" Акты списания отправляются при закрытии кассовой смены
\end{itemize}

\subsection{Передача в регистр 2}
\begin{itemize}
	\item При проведении документ делает движения по регистру "крюОстаткиАлкогольнойПродукцииТоргЗалЕГАИС"
	(описание)\Nameref{1002}
\end{itemize}

\subsection{Возврат из регистра 2}
\begin{itemize}
	\item При проведении документ делает движения по регистру "крюОстаткиАлкогольнойПродукцииТоргЗалЕГАИС"
	(описание)\Nameref{1002}
\end{itemize}

