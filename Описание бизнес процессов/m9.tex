\section{Активные расширения}

\subsection{РедактированиеПТиУИтогСНастройкой}\label{500}
\subsubsection{Описание}
Расширение создано для работы с документом "Поступление товаров и услуг". При создании формы на сервере проверяется наличие в основании документа "Товарно-транспортная накладная ЕГАИС (входящая)" и при наличии блокируются на изменение реквизиты документа:
\begin{itemize}[label={--}]
	\item Контрагент
	\item ДатаВходящегоДокумента
	\item НомерВходящегоДокумента
\end{itemize}
(код - крю\_0001)\par
Если документ  "Поступление товаров и услуг" создан на основании "Товарно-транспортная накладная ЕГАИС (входящая)", то перед проведением документа проверяется, что сумма документа совпадает с суммой входящей ТТН и если суммы не совпадают, выдается предупреждение. Заблокировать проведение нет возможности, так как у нас действительно могут быть бумажные документы, сумма в которых не будет совпадать с документом в 1С. Эти ошибки происходят за счет округления.
(код - крю\_0002)\par

\subsection{Заплатки ЕГАИС 2 2 12 30}\label{501}
\subsubsection{Описание}
\begin{itemize}[label={--}]
	\item Корректный пересчет даллов в литры, корректное вычисление цены. Реализовано в модуле объекта в процедуре <<РасшЗаплаткиЕГАС\_ЗаполнитьТоварыПоступленияПоТТН()>> документа <<Поступление товаров>>
	\item Отключена видимость и доступность реквизита <<ЕстьРасхождения>>  документа <<Поступление товаров>>, так как признак расхождения теперь указывается в документе <<Товарно-транспортная накладная ЕГАИС (входящая)>>. Реализовано в процедуре <<РасшЗаплаткиЕГАС\_ПриСозданииНаСервереПеред(Отказ, СтандартнаяОбработка)>> формы документа <<Поступление товаров>>.
	\item В модуле формы списка документов <<Товарно-транспортная накладная ЕГАИС (входящая)>> замещена процедура <<ОбработатьОтветНаВопросОСвязыванииСПрикладнымДокументом>>, 	перед созданием документа на основании ТТН , проверяется - была ли выполнена проверка поступившей алкогольной продукции в ТТН. И если нет, то выдается сообщение и создание документа не происходит.
	\item В модуле объекта документа <<Товарно-транспортная накладная ЕГАИС (исходящая)>> в процедуре после обработки заполнения происходит корректировка номера ТТН, если документ был создан на основании документа <<Возврат товаров поставщику>>.
	\item В документе <<Акт списания ЕГАИС>> в модуле формы документа замещена процедура <<ТоварыАлкогольнаяПродукцияНачалоВыбора>>. Теперь открытие формы выбора происходит с отбором по текущей номенклатуре, т.е. показывается только сопоставленная алкогольная продукция. В противном случае, когда открывается полный список, происходит большое количество ошибок.
	\item В справочнике <<КлассификаторАлкогольнойПродукцииЕГАИС>> в форме выбора в процедуре <<ПриСозданииНаСервере>> перед, формируется отбор для открытия динамического списка, только для сопоставленной алкогольной продукции.
\end{itemize}
\subsection{ВыгрузкаСБухгалтерию}\label{502}
\subsection{РаботаСБлокировкойНоменклатурыПоРеквизитам}\label{503}
\subsection{ршДДС}\label{504}
\subsection{СбросФильтраСотрудниковВВыплатеЗП}\label{505}
\subsection{БлокировкаСпрОрганизации}\label{506}
\subsection{БлокировкаСпрФизЛица}\label{507}
\subsection{ршДДС}\label{508}
\subsection{ДатыЗапрета}\label{509}
\subsection{ЗаплаткиПосле 2 2 12 30}\label{510}
\subsection{ЗаплаткиМОТП}\label{511}
\subsection{ВесыДобавлениеPLU}\label{512} (описание)\Nameref{1001}
\subsection{Халва\_В\_РМК}\label{513}
\subsection{ДоработкаПоДисконту}\label{514}






%\index{РедактированиеПТиУИтогСНастройкой}

\begin{itemize}[label={--}]
	\item
		\item
		\item




%	\begin{warning}
%	\textbf{Внимание!!! Нельзя использовать для чеков коррекции алкоголь!!!}
%	\end{warning}


\end{itemize}

