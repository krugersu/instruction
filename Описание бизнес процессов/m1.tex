\section{Закупки}
%\marginnote{\Date{Ср.}{08}{Апр.}{2020}}[-20pt]
\subsection{Поступление товаров}
\subsubsection{Поступление товаров на основании ТТН Входящей ЕГАИС}



\begin{itemize}
	\item Реквизит "Поставщик" должен быть недоступен для изменения (расш.)
	\Nameref{500}
	\item Реквизиты <<НомерВходящегоДокумента>> <<ДатаВходящегоДокумента>> должны быть недоступны для изменения (расш.)\Nameref{500}
	\item В табличной части <<Товары>> для редактирования доступны только поля:
	<<Цена>>, <<Всего>>, <<\%НДС->>, <<НДС>>.  (расш.)\Nameref{500}
	\item <<галочка>> "Есть расхождения" невидима (расш.)\Nameref{501}
	\item При создании документа на основании Входящей ТТН ЕГАИС в табличной части <<Товары>>
	корректно пересчитывается количество из  декалитров в литры (расш.)\Nameref{501}
	\item При заполнении документа на основании Входящей ТТН ЕГАИС реквизит <<Поставщик>> заполняется из реквизита <<Грузоотправитель>> Входящей ТТН ЕГАИС

\end{itemize}

\subsubsection{Поступление товаров общие свойства}

\begin{itemize}
	\item При установленной константе <<крюПроверятьПомеченнуюНаУдалениеВДокументах>>
	документ не должен проводиться, если в табличной части <<Товары>> присутствует
	номенклатура помеченная на удаление.(описание)\Nameref{1005}
	\item При проведении документ должен делать движения по регистру <<СебестоимостьНоменклатуры>> (описание)\Nameref{1002}
	\item При установленной константе <<крюПроводитьТару>> и наличии в табличной части <<Товары>> номенклатуры
	с видом "Возвратная тара" или <<Оборудование>>, документ должен делать движения по регистру <<ТараНаСкладах>> (описание)\Nameref{1003}
	\item Если табличная часть <<Товары>> содержит алкогольную продукцию, а документ не имеет в основании Входящую ТТН ЕГАИС, то документ не проводится (расш.)\Nameref{500}
	\item Если сумма являющейся основанием Входящей ТТН ЕГАИС не совпадает с суммой документа, то выдается предупреждение с требованием откорректировать суммы (расш.)
	\Nameref{500}
	\item Если в табличной части <<Товары>> есть строки содержащие цены на товар, которые отсутствуют или выше чем те, которые хранятся в регистре <<крюЦеныНоменклатурыКонтрагентов>> по данному поставщику, то проведение документа блокируется, и на назначенные адреса электронной почты уходит письмо с сообщение о данном событии. (расш.)\Nameref{500}
	(описание)\Nameref{1006}
\end{itemize}
\vspace{\baselineskip}\par
\subsubsection{Поступление товаров дополнительные механизмы}
\begin{itemize}
	\item Существует константа <<крюДатаЗапретаПроведенияДокументовБезТранспортнойТары>> при
	включении которой (установке даты) включается механизм контроля заполнения транспортной тары
	\begin{myquote}
     Необходимо проверить актуальность этого механизма.
	\end{myquote}


\end{itemize}



\subsection{Возврат товаров поставщику}
\begin{itemize}
	\item При установленной константе <<крюПроверятьПомеченнуюНаУдалениеВДокументах>>
	документ не должен проводиться, если в табличной части <<Товары>> присутствует
	номенклатура помеченная на удаление.(описание)\Nameref{1005}
	\item При установленной константе "крюПроводитьТару" и наличии в табличной части
	 Товары номенклатуры	с видом "Возвратная тара" или "Оборудование", документ должен делать движения по регистру "ТараНаСкладах" (описание)\Nameref{1003}
	\item  Регулировка механизма выгрузки в бухгалтерию. На форме добавлены два дополнительных реквизита "крюВыгружатьВБухгалтериюФ" и "крюНЕВыгружатьВБухгалтериюФ" при "включении" одного реквизита второй должен отключаться
	 (расш.)
	 \Nameref{502}
	\item Установленная константа "крюОчищатьНДСВВозвратеПоставщику" очищает признак учета НДС в Возврате поставщику.
	\item При печати ТОРГ-12 в представление организации в шапку добавлен КПП и параметр вид операции-Возврат (модуль менеджера)

\end{itemize}


\subsection{Перемещение товаров}

\begin{itemize}
	\item При установленной константе <<крюПроверятьПомеченнуюНаУдалениеВДокументах>>
	документ не должен проводиться, если в табличной части <<Товары>> присутствует
	номенклатура помеченная на удаление.(описание)\Nameref{1005}
	\item При установленной константе "крюКонтролироватьВозвратнуюТару" при наличии в табличной части <<Товары>> номенклатуры
	с видом "Возвратная тара" или "Оборудование", документ не проводится. (описание)\Nameref{1003}
	\item Если в в табличной части <<Товары>> в какой-либо строке отсутствует цена, то документ не проводится.\Nameref{1014}
	\item При проведении документ должен делать движения по регистру "СебестоимостьНоменклатуры" (описание)\Nameref{1002}
	\item В документе убрана возможность заполнение цен "по виду цен" и "по розничной цене". Добавлено заполнение "по себестоимости".(описание)\Nameref{1007}
\end{itemize}


\subsection{Входящие ТТН}

\begin{itemize}
	\item Перед созданием документа на основании ТТН, в ТТН должна быть выполнена проверка поступившей алкогольной продукции, в противном случае в создании документа на основании ТТН отказано.
    (расш.)
	\Nameref{501}
\end{itemize}


\subsection{Исходящие ТТН}

\begin{itemize}
	\item Если документ создается на основании <<Возврата товаров поставщику>>, то в качестве номера ТТН подставляется номер документа основания без префиксов и лидирующих нулей.
	(описание)\Nameref{1008}
	(расш.)
	\Nameref{501}
\end{itemize}


