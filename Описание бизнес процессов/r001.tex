% !TeX encoding = UTF-8
% !TeX spellcheck = russian_english
% !TeX root = Описание бизнес процессов.tex
\section{Дополнительные описания}

\subsection{Работа с весами}\label{1001}

Для решения задачи автоматической <<оттарки>> на весах, было сделано следующее:
Написана внешняя утилита-загрузчик по работе с весами <<Scales.exe>>. Утилита-загрузчик должна быть расположена в каталоге с базой данных.
Используется расширение \Nameref{512}.
В расширении после штатной загрузки весов происходит вызов  утилиты-загрузчика с помощью вызова штатной функции <<ЗапуститьПриложение>>, с передачей ей в командной строке:
\vspace{\baselineskip}

%\begin{itemize}
%	\item ПутьКФайлуЗагрузчика - Каталог с программой + имя файла утилиты-загрузчика
%	\item P\_RemoteHost - Адрес весов для обмена
%	\item P\_RemotePort - Порт весов для обмена
%	\item P\_TimeoutUDP - Таймаут
%\end{itemize}
%\caption{Параметры коммандной строки}
\begin{tabular}{p{0.3\linewidth}p{0.3\linewidth}}
	\toprule
	%	\hline
	Параметр & Значение \\
	\midrule
	ПутьКФайлуЗагрузчика & Каталог с программой + имя файла утилиты-загрузчика \\
%	\hline
	 P\_RemoteHost & Адрес весов для обмена \\
%	\hline
	 P\_RemotePort  & Порт весов для обмена \\
%	\hline
	 P\_TimeoutUDP  & Таймаут \\
%	\hline
	%	\hline
	\bottomrule %%% верхняя линейка
\end{tabular}


\vspace{\baselineskip}

Утилита-загрузчик читает данные из весов и на каждую запись (на каждое PLU) в весах создает пять дополнительных записей, в которых указан вес тары, а номер PLU увеличивается на 1000 с каждой тарой после первой, первая увеличивается на 2000, что бы оставить свободной первую тысячу номеров PLU.

%\begin{itemize}
%	\item контейнер маленький  0,006г PLU +2000
%	\item контейнер средний    0,008г PLU +3000
%	\item контейнер большой    0,012г PLU +4000
%	\item коробка фри          0,020г PLU +5000
%	\item пакет бумажный       0,015г PLU +6000
%\end{itemize}


\vspace{\baselineskip}
\begin{tabular}{p{0.3\linewidth}p{0.3\linewidth}}
	\toprule
	%	\hline
	контейнер маленький & 0,006г PLU +2000 \\
%	\midrule
	контейнер средний & 0,008г PLU +3000 \\
	%	\hline
	контейнер большой & 0,012г PLU +4000 \\
	%	\hline
	коробка фри  & 0,020г PLU +5000 \\
	%	\hline
	пакет бумажный  & 0,015г PLU +6000 \\
	%	\hline
	%	\hline
	\bottomrule %%% верхняя линейка
\end{tabular}
\vspace{\baselineskip}


%\footnote{}
 Затем весы полностью очищаются и затем загружаются измененные данные.Если при загрузке весы не пустые, то загрузка останавливается.
 В случае если что то пошло не так (неполные данные в весах, отсутствие позиций и пр.) Нужно повторить выгрузку.


\subsection{Себестоимость номенклатуры}\label{1002}

В программе отключен штатный механизм расчета себестоимости. Расчет себестоимости производит ся в собственном общем модуле <<крюРасчетСебестоимости>>

В данные момент расчет себестоимости происходит при проведении следующих документов:
\begin{itemize}
	\item Оприходование товаров
	\item Перемещение товаров
	\item Поступление товаров
	\item Сборка товаров

\end{itemize}

\subsection{Движение возвратной тары и оборудования}\label{1003}

Для решения задачи учета возвратной тары и оборудования создан новый регистр накопления <<ТараНаСкладах>>.

Регистрация движений в регистре происходит при проведении следующих документов:
\begin{itemize}
	\item Возврат товаров поставщику
	\item Поступление товаров
\end{itemize}

Для корректного отображения движения тары и оборудования, т.к. тара и оборудование могут придти только от конкретного поставщика и списание со склада может быть тоже только с указанием конкретного контрагента, то происходит блокировка проведения при наличии возвратной тары или оборудования в табличной части товары в  документах:

\begin{itemize}
	\item Перемещение товаров
	\item Оприходование товаров
	\item Приходный ордер на товары
	\item Расходный ордер на товары
\end{itemize}

Этот механизм реализован в общем модуле <<крюРасчетСебестоимости>>


\subsection{Механизм сборки товаров}\label{1004}

Документ <<СборкаТоваров>> создается в момент проведения документа <<ОтчетОРозничныхПродажах>>.
Если документ <<ОтчетОРозничныхПродажах>> проводится повторно, то предварительно, ранее созданные документы <<СборкаТоваров>> на основании текущего, удаляются и создаются новые. Этот механизм позволяет актуализировать возможные изменения в документе <<ОтчетОРозничныхПродажах>>.

Документ <<СборкаТоваров>> создается на каждую строку номенклатуры табличной части <<Товары>>, документа  <<ОтчетОРозничныхПродажах>>, которая имеет запись в регистре сведений "КомплектующиеНоменклатуры" (кулинария).
Документ <<СборкаТоваров>> приходует на склад нужное количество комплектующих для корректной продажи конечной позиции номенклатуры. Так же в момент проведения документа <<СборкаТоваров>> расчитывается себестоимость итоговой позиции на основании себестоимости комплектующих.
При этом документ <<СборкаТоваров>> записывается по времени раньше, чем документ <<ОтчетОРозничныхПродажах>>. Это сделано для корректного списания остатков

\subsection{Помеченная на удаление номенклатура в документах}\label{1005}

Осуществляется проверка на наличие в табличной части <<Товары>> номенклатуры с пометкой удаления.
Выполняется в модуле <<крюРасчетСебестоимости>> для документов:


\begin{itemize}
	\item Оприходование товаров
	\item Перемещение товаров
	\item Поступление товаров
	\item Списание товаров
	\item Приходный ордер на товары
	\item Расходный ордер на товары
	\item Возврат товаров поставщику

\end{itemize}

\subsection{Цены в документе Поступление товаров}\label{1006}

Перед проведением документа <<Поступление товаров>>, при установленной константе <<крюБлокировкаПродаж>>, выполняется проверка цен номенклатуры в табличной части <<Товары>>. Проверка на данном этапе не выполняется для следующих групп номенклатуры:
%\begin{itemize}
%	\item ЦБ-00005065 Сигареты без маркировки
%	\item ЦБ-00004758 Сигареты маркированные
%	\item КН000792 Кулинария
%\end{itemize}

\vspace{\baselineskip}
\begin{tabular}{p{0.3\linewidth}p{0.3\linewidth}}

	\toprule
	ЦБ-00005065 & Сигареты без маркировки \\
	%	\midrule
	ЦБ-00004758 & Сигареты маркированные \\

	КН000792 & Кулинария \\
	\bottomrule %%% верхняя линейка
\end{tabular}
\vspace{\baselineskip}\par
Для проверки берется цена номенклатуры из регистра сведений <<крюЦеныНоменклатурыКонтрагентов>>,
которые устанавливаются специальным работником в разрезе магазина и контрагента и в случае использования характеристик номенклатуры - характеристики.\par
Если цена в документе оказывается больше чем установленная на данную номенклатуру в регистре или если цена на эту номенклатуру отсутствует, то в этом случае документ не проводится и формируется признак информационного сообщения. Если цена в документе ниже установленной в регистре, то формируется признак информационного сообщения.\par
После окончания проверки цен в табличной части документа, признак информационного  сообщения установлен, то информация по расхождению цен в документе отправляется по настроенным адресам электронной почты.\par
Адреса для отправки берутся из справочника <<УчетныеЗаписиЭлектроннойПочты>>, отбираются записи, у которых реквизит  <<ИмяПользователя>> содержит - <<КонтрольПТИУ>>.
(расш.)\Nameref{500}


\subsection{Документ перемещение, заполнение по виду цен}\label{1007}

В процедуре <<ПриСозданииНаСервере>>, при установленной константе <<крюСебестоимостьВПеремещении>>, отключается видимость элементов <<ТоварыЗаполнитьЦеныПоВидуЦен>> и <<ТоварыЗаполнитьЦеныПоРозничнымЦенам>>.
Далее создается команда и кнопка <<ЗаполнитьПоСебестоимости>>, при нажатии на которую цены в табличной части <<Товары>> заполняются по себестоимости.


\subsection{Документ ТТН Исходящая, на основании возврата}\label{1008}

Если документ <<Товарно-транспортная накладная ЕГАИС (исходящая)>> создается на основании<<Возврата товаров поставщику>>, то в расширении (расш.)
\Nameref{501} после процедуры <<ОбработкаЗаполнения>> вызывается штатная процедура <<ПрефиксацияОбъектовКлиентСервер.НомерНаПечать>>,которая из номера документа удаляет префикс и лидирующие нули.

\subsection{Контроль остатков алкогольной продукции}\label{1009}
%\renewcommand{\thefootnote}{\fnsymbol{footnote}}
В конфигурации создан регистр <<крюОстаткиАлкогольнойПродукцииТоргЗалЕГАИС>> для целей партионного учета алкогольной продукции. Регистр позволяет хранить остатки алкогольной продукции в торговом зале в разрезе справок В. \footnote{Тогда как при перемещении товара в регистр 2(Торговый зал) ЕГАИС справки В теряются} Это позволяет осуществлять корректные возвраты из регистра 2 с указанием конкретной Справки В по которой алкогольная продукция была передана в торговый зал. Упрощает создание возвратов поставщику, так можно выбрать по какой справке будем возвращать данную алкогольную продукцию. Позволяет легче решать проблемы с пересортицей товаров или возникшей ошибке продажи товара в минус. \par
\vspace{\baselineskip}\par
\vspace{\baselineskip}\par
Регистраторами у регистра являются следующие документы:
\begin{itemize}
	\item Акт постановки на баланс ЕГАИС
	\item Акт списания ЕГАИС
	\item Возврат из регистра №2 ЕГАИС
	\item Корректировка регистров
	\item Передача в регистр №2 ЕГАИС
\end{itemize}

\subsection{Акт списания, перезаполнение на основании ОРП}\label{1010}
%\renewcommand{\thefootnote}{\fnsymbol{footnote}}
На форму документа <<АктСписанияЕГАИС>>  добавлена кнопка \keys{Перезаполнить на основании ОРП} \par
Добавлена команда <<ЗаполнитьНаОснованииОРП>>, которая вызывает типовую процедуру <<ИнтеграцияЕГАИСРТ.ЗаполнитьАктСписанияЕГАИСНаОснованииОтчетаОРозничныхПродажах>>


\subsection{Документ <<Зарплата к выплате организаций>>}\label{1011}
В табличную часть <<Зарплата>> документа <<Зарплата к выплате организаций>> добавлен реквизит <<КлючСтроки>>. Данный реквизит заполняется обработкой по выдаче зарплаты, которая генерирует уникальный ключ строки и заполняет им реквизит. Это необходимо для механизма обмена с УТ10. В УТ на каждую строку документа создается РКО. <<КлючСтроки>> фиксируется в этом РКО, и по немк в случае перезагрузки документов из Розницы, происходит поиск и удаление документов РКО в УТ10.

\subsection{Цены в документе <<Установка цен номенклатуры>>}\label{1012}

\subsection{Цены в документе <<Перемещение товаров>>}\label{1014}
При проведении документа <<Перемещение товаров>> в собственном общем модуле <<крюРасчетСебестоимости>>
реализован контроль наличия цены в строках табличной части <<Товары>>. При отсутствии цены, в какой-либо из строк, проведение документа блокируется.
Сделано для того, что бы не потерять историю цен и корректно расчитать себестоимость.