\section{ЕГАИС}
%\marginnote{\Date{Ср.}{08}{Апр.}{2020}}[-20pt]
\subsection{Алгоритмы ЕГАИС}


\begin{itemize}
	\item Реализован алгоритм получения Номера и даты ГТД и Производителя из справки А
	\item Реализованна корректировка позволяющая попадать в список складов для которых создаются Акты списания, склады с разливным пивом
	\item Изменен алгоритм обработки заполнения Актов постановки на баланс
	\item Доработан типовой алгоритм заполнения табличной части документа Акт постановки на баланс на 	основании документа оприходования алгоритм изменен только для оприходования в торговый зал
	\item Изменен алгоритм заполнения акта списания на основании ОРП. В связи с тем, что не используем вскрытие тары, списание алкогольной продукции  нужно производить и для разливного пива

	\item Изменен алгоритм заполнения акта списания на основании Списания товаров
	\item Создан свой модуль для автоматического запроса остатков ЕГАИС
	\item Реализован механизм работы с транспортной тарой
	\item В регистре "НастройкиОбменаСЕГАИС" в форме записи всегда видны настройки обмена на сервере или клиенте
	\item В справочнике "Классификатор алкогольной продукции" скрыт реквизит "Импортер", вместо него показан новый "крюИмпортер", подставлено значение из регистра сведений "Дополнительные сведения по Алкогольной продукции ЕГАИС", запрещено редактировать поле, оставлена только возможность открывать выбранный элемент.

\end{itemize}
\newpage
\subsection{Особенности партионного учета}

Основой для партионного учета в конфигурации является добавленный регистр накопления <<крюОстаткиАлкогольнойПродукцииТоргЗалЕГАИС>>. И документы, которые при проведении делают движения по этому регистру. \par
\noindent \underline{Регистр имеет следующую структуру:}

\vspace{\baselineskip}
\begin{minipage}{0.5\textwidth}
   % \begin{flushleft}
        \noindent\textbf{Измерения}
        \begin{itemize}
            \item ОрганизацияЕГАИС
            \item Склад
            \item Номенклатура
            \item АлкогольнаяПродукция
            \item СправкаБ
            \item ДокументПриход
            \item ТранспортнаяТара
        \end{itemize}
        \textbf{Ресурсы}
        \begin{itemize}
            \item КоличествоУпаковок
        \end{itemize}
        \textbf{Реквизиты}
        \begin{itemize}
            \item Упаковка
        \end{itemize}
        \vspace{\baselineskip}
  %  \end{flushleft}
\end{minipage}
\hfill
\begin{minipage}{0.5\textwidth}
 %   \begin{flushright}
      \noindent\textit{Регистраторами являются документы:}

       \begin{itemize}
           \item АктПостановкиНаБалансЕГАИС
           \item АктСписанияЕГАИС
           \item ВозвратИзРегистра2ЕГАИС
           \item КорректировкаРегистров
           \item ПередачаВРегистр2ЕГАИС
       \end{itemize}
 %   \end{flushright}
\end{minipage}

Регистр добавлен потому, что в регистре 2 ЕГАИС отсутствуют справки Б. Таким образом нет возможности корректно списывать остатки алкогольной продукции.

Учет ведется в разрезе <<Справки Б>> поступающей с документом <<ТТНВходящаяЕГАИС>>. Документ  <<ПередачаВРегистр2ЕГАИС>> оправляет алкогольную продукцию с текущей справкой в наш регистр. При списании алкогольной продукции из регистра 2,  при наличии нескольких позиций с разными справками Б идет списание с самой ранней по времени прихода справки Б.

%Для партионного учета:
%В конфигураци Розница ООО для нашего регистра партионного учета ‘крюОстаткиАлкогольнойПродукцииТоргЗалЕГАИС’ регистраторами являются пять документов - АктПостановкиНаБалансЕГАИС, АктСписанияЕГАИС, ВозвратИзРегистра2ЕГАИС, КорректировкаРегистров, ПередачаВРегистр2ЕГАИС.

