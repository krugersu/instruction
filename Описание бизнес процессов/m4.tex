\section{Финансы}
%\marginnote{\Date{Ср.}{08}{Апр.}{2020}}[-20pt]
\subsection{Приходный кассовый оредер}


\begin{itemize}
	\item Исправлено заполнение на основании РКО для статьи ДДС подотчет.
	\item Для возврата из <<Подотчета>> - Если статья ДДС <<Подотчет>>, то без документа основания РКО документ не проводится.
	\item При открытии документа корректно заполняется значение глобальной переменной НомерДокументаКассыККМ
	с учетом нашего разделения на две кассы.
	\item Корректное получение кассы при печати чека (используем?)
	с учетом нашего разделения на две кассы.
\end{itemize}

\subsection{Расходный кассовый ордер}

\begin{itemize}
	\item  Регулировка механизма выгрузки в бухгалтерию. На форме добавлены два дополнительных реквизита <<крюВыгружатьВБухгалтериюФ>> и <<крюНЕВыгружатьВБухгалтериюФ>> при <<включении>> одного реквизита второй должен <<отключаться>>
	(расш.)
	\Nameref{502}
	\item При записи документа контролируется что: в табличной части <<РасшифровкаПлатежа>> в каждой строке указано субконто, в противном случае проведение документа блокируется. Для каждой строки табличной части <<РасшифровкаПлатежа>> проверяется, что статья ДДС соответствует установленному списку, в противном случае проведение документа блокируется.(расш.)
	\Nameref{504}
	\item При открытии документа становится видимым добавленный реквизит <<крюБанкВноситель>>, если хозяйственная операция - <<СдачаДенежныхСредствВБанк>>
	\item При открытии документа корректно заполняется значение глобальной переменной НомерДокументаКассыККМ
	с учетом нашего разделения на две кассы.
	\item Корректно устанавливается номер чека ККМ при печати чека <<КлиентИнкассация>>  с учетом наших двух касс
	\item Корректно устанавливается номер чека ККМ при печати чека "Клиент"  с учетом наших двух касс
	\item При изменении кассы корректно заполняется значение глобальной переменной НомерДокументаКассыККМ
	\item Добавлено заполнение структуры для заполнение печатных форм РКО
	\item Добавлен вариант печати для <<ПрепроводительнаяВедомостьНакладная>>
	\item изменен принцип и наименование команды печати при установленной константе <<крюПечатьРКОСразуНаПринтер>>
	\item Добавляется печатная форма <<ПечатьПрепроводительнаяВедомостьНакладнаяКСумке>>
	\item Изменено заполнение параметра печати <<НаименованиеБанкаВносителя\_БИК>> для инкассации.


\end{itemize}



\subsection{Зарплата к выплате организаций}
\begin{itemize}
     \item При создании-открытии документа сбрасывается фильтр на сотрудников и появляется возможность выбрать любого сотрудника, а не только устроенного в данные магазин (расш.)
     \Nameref{505}
	 \item В табличную часть "Зарплата" добавлен реквизит <<КлючСтроки>> для использования при обмене с УТ
	 	(описание)\Nameref{1011}

\end{itemize}

