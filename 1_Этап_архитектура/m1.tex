%\section*{Варианты архитектуры первого этапа относительно цен поставщика}
\marginnote{\Date{Пн.}{11}{Апр}{2017}}[20pt]
\section{Варианты архитектуры первого этапа относительно цен поставщика}
\subsection{Вариант № 1}
\reversemarginpar
\marginpar{\textit{Вариант не подходит}.}
\subsubsection  {Описание} 
\begin {itemize}	

\item Максимально используем штатные механизмы с минимум изменений. Справочник ,,Виды цен`` делаем иерархическим и таким путем получаем <<деление>> на цены контрагентов.
\item В документ ,,УстановкаЦенНоменклатуры`` изменения не вносим.
\todo[inline, color=green!40]{testing testing}
\item Новый регистр не добавляем.
\item Изменяется документ ,,ПоступлениеТоваров``.
\end{itemize}

%%
\subsubsection{Плюсы}
\begin{itemize}	
\item Минимальные изменения в конфигурации.
\item Скорость разработки и внедрения.
\end{itemize}
\subsubsection{Минусы}
\begin{itemize}	
\item Большой и относительно плохо структурированный справочник ,,Виды цен``.
\item  При заполнении документа ,,УстановкаЦенНоменклатуры`` при выборе цен, список для выбора цен очень большой,
что создает неудобство и повышает вероятность \CommentText{ConsecA}{ConsecB}{ошибки}.
\item Неудобство при заполнении <<эталонными>> ценами документа ,,ПоступлениеТоваров``, т.к. в справочнике ,,Виды цен`` отсутствует явная привязка цен к конкретному контрагенту.	
\end{itemize}
\CommentPar[draw=brown][draw=red, shift={(2.5cm,2.5cm)}, rotate=40, text width=2cm, align=center][out=-120, in=20, distance=0.5cm,->]{ConsecA}{ConsecB}{~~~Ошибки  \\ это~~ \\ нехорошо!!!}

\subsection{Вариант № 2}

\reversemarginpar

%\marginpar[$\Rightarrow$]{$\Leftarrow$}{

\marginpar{%
\centering \includegraphics[width=0.3\linewidth]{images/ops}




}
%{\textit{Вариант не подходит}.}
%\begin{marginfigure}
%	\includegraphics[width=0.02\linewidth]{images/ops}
%	\caption{\textit{Вариант не подходит}.}
%\end{marginfigure}
%\marginpar  {\textit{Вариант не подходит}.}


\subsubsection {Описание}
\begin{itemize}	
\item Создаем новый документ ,,Установка цен номенклатуры контрагентов``.
\item Добавляем регистр ,,Цены номенклатуры контрагентов``.
\item В регистре цена привязывается к контрагенту.
\item Справочник ,,Виды цен`` делаем иерархическим и таким путем получаем <<деление>> на цены контрагентов.
\end{itemize}
\subsubsection{Плюсы}
\begin{itemize}	
\item Получаем возможность выбрать цены по конкретному контрагенту из регистра в момент заполнения документа ,,ПоступлениеТоваров``.
\end{itemize}
\subsubsection{Минусы}
\begin{itemize}	
\item \CommentText{MexicanA}{MaxicanB}{Большой} и относительно плохо структурированный справочник ,,Виды цен``.
\item При заполнении документа ,,УстановкаЦенНоменклатуры`` при выборе цен, список для выбора цен очень большой, т.к. всё еще отсутствует возможность однозначно отобрать в справочнике цены по конкретному контрагенту что создает неудобство и повышает вероятность ошибки.
\end{itemize}
\CommentPar[brown][text width=3cm, color=violet, shift={(9.7cm,-3.5cm)}, rotate=-65][brown, out=225, in=-80, distance=0.7cm, ->]{MexicanA}{MaxicanB}{Большой справочник породит проблемы}


\subsection{Вариант № 3 Утвержден!!!!!!!}
\subsubsection{Описание}
\begin{itemize}	
\item Решение которое было описано в проекте тех. задания.
\item Добавляем регистр ,,Цены номенклатуры контрагентов``.
\item Добавляем справочник ,,Цены номенклатуры контрагентов`` в котором храним наименование и тип цены с привязкой к контрагенту.
\item В регистре цена привязывается к справочнику ,,Цены номенклатуры контрагентов``.
\end{itemize}
\subsubsection{Плюсы}
%\todo[inline,caption={}]{
%	\begin{itemize}
%		\item blah blah blah
%		\item blah blah blah
%		\item blah blah blah
%	\end{itemize}
%}
\begin{itemize}	
\item Получаем возможность выбрать цены по конкретному контрагенту и виду цены  из регистра в момент заполнения документа ,,ПоступлениеТоваров``.

\item Документ ,,Установка цен номенклатуры контрагентов`` при заполнении цен в выборку попадают только цены конкретного выбранного контрагента, что существенно снижает вероятность ошибки оператора.
\item Получаем хорошо структурированную связь между контрагентом, ценой и типом цены.	
\item Справочник ,,Виды цен`` остается компактным и легко читаемым.
\sidenote{Тестовая сноска}
\end{itemize}
\subsubsection{Минусы}
\begin{itemize}	
\item Более  \CommentText{estA}{estB}{длительный срок разработки}  по сравнению с первыми двумя вариантами.
\item Дополнительные изменения внесенные в конфигурацию .
\end{itemize}
\CommentPar[brown][color=blue, shift={(0.70cm,1.0cm)}, rotate=-50][draw=none]{estA}{estB}{Да!!}