\section{Развертывание новой базы ИП}
\subsection{Подготовка новой базы ЦУ}

Если заготовка ЦУ еще отсутствует, необходимо выполнить следующее:
\begin{itemize}
    \item Выгрузить cf из тестовой базы (база для разработки)
    \item Загрузить cf в чистую базу нового ЦУ
\end{itemize}


\subsubsection{Настройки}
\begin{itemize}
     \item Настройки пользователей и прав
\begin{todolist}
    \item[\done] Группы пользователей
\end{todolist}
\item Организации и финансы
\begin{todolist}
    \item Несколько организаций
    \item[\done] Выплаты заработной платы в магазинах

\end{todolist}

\item Настройки номенклатуры
\begin{todolist}
    \item[\done] Упаковки номенклатуры
    \item[\done] Код и артикул в колонке отображения номенклатуры
    \item[\done] Полнотекстовый поиск при подборе
    \item[\done] Алкогольная продукция
    \item[\done] Выгружать продажи немаркируемой продукции в ЕГАИС
\end{todolist}


\item Маркетинг
\begin{todolist}
    \item[\done] Скидки наценки и ограничения продаж
    \item[\done] Использовать запрет розничной продажи
    \item[\done] Бонусные программы лояльности
    \item[\done] Подключаться к дисконтному серверу
\end{todolist}

\item Запасы и закупки
\begin{todolist}
    \item Учет себестоимости
    \item[\done] Ордерная схема
\end{todolist}

\item Продажи
\begin{todolist}
    \item[\done] Оплата платежными картами
    \item[\done] Логирование действий кассира в РМК
\end{todolist}

	\begin{warning}
    \textbf{При переносе справочников нужно снимать <<галки>> <<Выгружать при необходимости>>
        с тех справочников, который не нужны для выгрузки, в частности: Магазины, Склады, Организации.}
\end{warning}
%\newpage
\item Справочники переносим стандартной обработкой "Выгрузка и загрузка данных XML"
Переносятся следующие справочники:
\begin{itemize}
    \item Номенклатура
    \item Контрагенты
    \item Аналитика хозяйственных операций
    \item Дисконтные карты
    \item Физические лица
    \item Группы пользователей
    \item Пользователи
\end{itemize}

\end{itemize}
