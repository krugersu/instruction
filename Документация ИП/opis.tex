\section{Описание}

Для работы с ИП создается отдельная база Розницы. При разработке необходимо максимально избежать изменения конфигурации Для облегчения поддержки и обновления. ЦУ ИП создан из cf файла ЦУ розницы. Штатными методами нарезаны два узла.

Создан регистр сведений,который отсутствует в конфигрурации ООО - <<крюСостояниеТТН>>. Регистр предназначен для сохранения истории и текущего состояния документов ЕГАИС. На данных из этого регистра будет основано рабочее место сотрудника ЕГАИС.  \par

Отличия конфигурации <<ИП>> от <<ООО>> заключаются в различном подходе к работе с ЕГАИС.
%\raisebox{0pt}[0pt][0pt]{\Large%
%    \textbf{Aaaa\raisebox{-0.3ex}{a}%
%        \raisebox{-0.7ex}{aa}%
%        \raisebox{-1.2ex}{r}%
%        \raisebox{-2.2ex}{g}%
%        \raisebox{-4.5ex}{h}
%    }
%}
 \framebox[0.65\textwidth][c]{\textcolor{blue}{И мы до сих пор не знаем есть ли отличия в торговле табаком!}} Основное это наличие только одного ключа ЕГАИС не зависимо от количества магазинов. Соответственно возникла необходимость корректировать алгоритмы работы с ЕГАИС применительно к ИП. При этом было принято решение сохранить алгоритмы партионного учета присутствующие в нашей конфигурации.\par
 Ключ устанавливается в центральном узле ИП. Все действия с алкогольной продукцией производятся только в центральном узле ИП. Магазины осуществляют только управленческую деятельность. В связи с этим  был принят следующий алгоритм работы ИП с ЕГАИС:
 \begin{itemize}
     \item Получение документа <<Товарно-транспортная накладная ЕГАИС (входящая)>> осуществляется в ЦУ ИП.
     Расширение <<ДляРегистрацииВходящейТТН>> добавляет в документ реквизит <<крюМагазин>> в котором, после получения ТТН указывается магазин по котрому необходимо выполнить регистрацию данного документа для выгрузки. Проводит документ. Данные из этого реквизита сохраняются в регистре сведений <<крюСостояниеТТН>> в месте с текущим состоянием (док зарегистрирован) и перезаполняются в документе при открытии из этого регистра.
     \item  Ответственное лицо запускает обмен РИБ с указанным магазином и сообщает товароведу данного магазина, о том что им отправлена ТТН.
     \item В магазине ТТН получают. Осуществляют приемку товара \footnote{Так же есть возможность проконтролировать приемку товара по справкам Б}. Заполняют данные о количестве принятого в ТТН (Выполняют проверку алкогольной продукции). Создают <<Поступление товаров>> на основании ТТН, проводят его. Выполняют обмен с ЦУ. Сообщают ответственному лицу о завершении работы с ТТН.
     \item Отвественное лицо выполняя обмен в ЦУ осуществляет прием обработанной ТТН и далее отправляет ТТН в ЕГАИС с соответствующей причиной \footnote{ТТН может быть подтверждена полностью, в случае если весть товар получен и притензий нет. Частично, если часть товара не пришла или товар не качественный или не соответствует накладно. Полный отказ, если товар не пришел, не соответствующего качества или иные причины}.
     \item После подтверждения входящей ТТН, ответственное лицо создает на основании подтвержденной документ <<Передача в регистр №2 ЕГАИС>>  и проводит его. Алкогольная продукция присутствующая в документе перемещается на второй регистр. Ответственный передает данные в ЕГАИС.
     \item В узле магазина  на основании документа <<Отчет о розничных продажах>> созданного при закрытии кассовой смены, формируется документ <<Акт списания ЕГАИС>> заполненный алкогольной продукцией присутствующей в <<Отчете о розничных продажах>> \footnote{Необходимо различать номенклатурную позицию это тот алкоголь которым мы торгуем и алкогольную продукцию, которая является элементом системы ЕГАИС. Эти две сущности могут иметь различное наименование, но тем неменее позиция номенклатуры и алкогольная продукция имеют четкую связь} <<Акт списания ЕГАИС>> автоматически проводится и с обменом уходит в ЦУ.
     \item В ЦУ на данном этапе в ручном режиме, а в дальнейшем с помощью фонового задания документы <<Акт списания ЕГАИС>> со всех магазинов отправляются в ЕГАИС. Документ <<Акт списания ЕГАИС>> списывает с остатка алкогольную продукцию в ЕГАИС и с нашего регистра накопления <<крюОстаткиАлкогольнойПродукцииТоргЗалЕГАИС>>.
 \end{itemize}
