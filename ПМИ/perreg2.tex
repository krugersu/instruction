\subsection{Формирование документа Передача в регистр №2 ЕГАИС}

\renewcommand{\arraystretch}{1.8} %% расстояние между строками таблицы
%\begin{landscape}
\begin{longtable}{|p{0.02\linewidth}|p{0.3\linewidth}|p{0.3\linewidth}|p{0.3\linewidth}|}


%****************************************************************************************************
 \hline
\multicolumn{4}{|c|}{\textbf{\textit{Движение по регистру <<Остатки алкогольной продукции в торговом зале ЕГАИС>>}}} \\
\hline

\hline
\Rownum &  Перейти в раздел Склад, выбрать <<Передачи в регистр №2>>.  & 1. Открылся список документов  <<Передачи в регистр №2>>;\par
2. Отображаются все документы &  \\
\hline
\Rownum & Открыть существующий документ имеющий статус <<Проведен в ЕГАИС>>   & 1. Открылась форма существующего документа;\par
2. Статус документа имеет значение <<Проведен в ЕГАИС>>
&  \\
\hline
\Rownum	& Нажать кнопку \keys{Провести} &  Документ проводится без ошибок &  \\
\hline
\Rownum	& Выбрать команду <<Движения документа>> & Откроется отчет по движениям документа &  \\
\hline
\Rownum	& Найти в отчете движения по регистру <<Остатки алкогольной продукции в торговом зале ЕГАИС>> & Движения документа по регистру сведений <<Остатки алкогольной продукции в торговом зале ЕГАИС>> присутствуют. Измерения <<Период>>, <<Организация ЕГАИС>>,<<Склад>>, <<Номенклатура>>, <<Алкогольная продукция>>, <<Справка Б>>, <<Документ приход>> заполнены. Значение ресурса <<Количество упаковок>> заполнено  &  \\
\hline
%****************************************************************************************************
\end{longtable}