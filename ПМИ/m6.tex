\subsection{Формирование документа Перемещение товаров}



\renewcommand{\arraystretch}{1.8} %% расстояние между строками таблицы
%\begin{landscape}
\begin{longtable}{|p{0.02\linewidth}|p{0.3\linewidth}|p{0.3\linewidth}|p{0.3\linewidth}|}
    %  {|c|c|l|c|}
    \hline
    № & \textbf{Действие} & \textbf{Ожидаемый результат} & \textbf{Фактический результат} \\
    %****************************************************************************************************
    \hline
    \hline
    \endhead
    \multicolumn{4}{|c|}{\textbf{\textit{Проверка на номенклатуру помеченную на удаление}}} \\
    \hline
    \hline
    \Rownum & Проверить, что включена константа <<крюПроверятьПомеченнуюНаУдалениеВДокументах>>  & Значение константы - Истина &  \\
    \hline
    \Rownum &Перейти в раздел Закупки, выбрать <<Перемещения товаров>>.  & 1. Открылся список документов  <<Перемещения товаров>>;\par
    2. Отображаются все документы &  \\
    \hline
    \Rownum & Создать новый документ по кнопке \keys{Создать}  & 1. Открылась форма создания документа;\par
    2. По умолчанию в открывшейся форме заполнено поле <<Магазин отправитель>> &  \\
    \hline
    \Rownum & Заполнить реквизит <<Склад отправитель>> значением соответствующим выбранному магазину &Заполнен <<Склад отправитель>> и <<Организация отправитель>> ;    &  \\
    \hline
    \Rownum	& Заполнить реквизит <<Магазин получатель>> значением <<74. Лазурная 27/2, Новосибирск>> & Заполны реквизиты: 1. <<Магазин получатель>> значением <<74. Лазурная 27/2, Новосибирск>>;\par
    2. <<Склад получатель>> значением <<74. Лазурная 27/2, Новосибирск>>;\par
    3. <<Организация получатель>> значением <<ООО "КРЮГЕР ХАУС" КОЛЬЦОВО (Новосибирск, Лазурная ул, 27/2)>>  &  \\
    \hline
     \Rownum	& Нажать кнопку <<Добавить>> в табличной части <<Товары>>  & Открылась форма выбора справочника <<Номенклатура>>  &  \\
    \hline
    \Rownum	& Выбрать из справочника <<Номенклатура>> элемент помеченный на удаление & Заполнились поля в табличной части <<Код>>, <<Артикул>>, <<Номенклатура>>, <<Ед.изм>> &  \\
    \hline
    \Rownum	&Заполнить поле <<Количество>> значением <<1>>  & Заполнилось поле <<Количество>> &  \\
    \hline
    \Rownum	& Заполнить поле <<Цена>> значением <<1>>  & Заполнилось поле <<Цена>> &  \\
    \hline
    \Rownum	& Нажать кнопку \keys{Провести и закрыть} & 1. Программа выдает сообщение о неудачи проведения документа;\par 2. При закрытии окна сообщения в строке сообщений появился текст ошибке с информацией, что документ содержит удаленную номенклатуру с указанием номеров строк и наименований &  \\
    %****************************************************************************************************

    %****************************************************************************************************

    %****************************************************************************************************

    %****************************************************************************************************
    \hline
    \hline
    \multicolumn{4}{|c|}{\textbf{\textit{Контроль тары и оборудования}}} \\
    \hline
    \hline
    \Rownum & Проверить, что включена константа <<крюКонтролироватьВозвратнуюТару>>  & Значение константы - Истина &  \\
    \hline
    \Rownum &Перейти в раздел Закупки, выбрать <<Перемещения товаров>>.  & 1. Открылся список документов  <<Перемещения товаров>>;\par
   2. Отображаются все документы &  \\
   \hline
   \Rownum & Создать новый документ по кнопке \keys{Создать}  & 1. Открылась форма создания документа;\par
   2. По умолчанию в открывшейся форме заполнено поле <<Магазин отправитель>> &  \\
   \hline
   \Rownum & Заполнить реквизит <<Склад отправитель>> значением соответствующим выбранному магазину &Заполнен <<Склад отправитель>> и <<Организация отправитель>> ;    &  \\
   \hline
   \Rownum	& Заполнить реквизит <<Магазин получатель>> значением <<74. Лазурная 27/2, Новосибирск>> & Заполны реквизиты: 1. <<Магазин получатель>> значением <<74. Лазурная 27/2, Новосибирск>>;\par
   2. <<Склад получатель>> значением <<74. Лазурная 27/2, Новосибирск>>;\par
   3. <<Организация получатель>> значением <<ООО "КРЮГЕР ХАУС" КОЛЬЦОВО (Новосибирск, Лазурная ул, 27/2)>>  &  \\
   \hline
   \Rownum	& Нажать кнопку <<Добавить>> в табличной части <<Товары>>  & Открылась форма выбора справочника <<Номенклатура>>  &  \\
   \hline
    \Rownum	& Выбрать из справочника <<Номенклатура>> элемент с кодом <<00000013>> - <<КЕГ (50 л)>> & Заполнились поля в табличной части <<Код>>, <<Артикул>>, <<Номенклатура>>, <<Ед.изм>> &  \\
    \hline
    \Rownum	&Заполнить поле <<Количество>> значением <<1>>  & Заполнилось поле <<Количество>> &  \\
    \hline
    \Rownum	& Заполнить поле <<Цена>> значением <<1>>  & Заполнилось поле <<Цена>> &  \\
    \hline
    \Rownum	& Нажать кнопку \keys{Провести и закрыть} & 1. Программа выдает сообщение о неудачи проведения документа;\par 2. При закрытии окна сообщения в строке сообщений появился текст ошибке с информацией, что документ содержит возвратную тару или оборудование с указанием номеров строк и наименований &  \\
    %****************************************************************************************************

    %****************************************************************************************************
    \hline
    \hline
    \multicolumn{4}{|c|}{\textbf{\textit{Контроль наличия цены в табличной части <<Товары>>}}} \\
    \hline
    \hline
    \Rownum &Перейти в раздел Закупки, выбрать <<Перемещения товаров>>.  & 1. Открылся список документов  <<Перемещения товаров>>;\par
    2. Отображаются все документы &  \\
    \hline
    \Rownum & Создать новый документ по кнопке \keys{Создать}  & 1. Открылась форма создания документа;\par
    2. По умолчанию в открывшейся форме заполнено поле <<Магазин отправитель>> &  \\
    \hline
    \Rownum & Заполнить реквизит <<Склад отправитель>> значением соответствующим выбранному магазину &Заполнен <<Склад отправитель>> и <<Организация отправитель>> ;    &  \\
    \hline
    \Rownum	& Заполнить реквизит <<Магазин получатель>> значением <<74. Лазурная 27/2, Новосибирск>> & Заполны реквизиты: 1. <<Магазин получатель>> значением <<74. Лазурная 27/2, Новосибирск>>;\par
    2. <<Склад получатель>> значением <<74. Лазурная 27/2, Новосибирск>>;\par
    3. <<Организация получатель>> значением <<ООО "КРЮГЕР ХАУС" КОЛЬЦОВО (Новосибирск, Лазурная ул, 27/2)>>  &  \\
    \hline
    \Rownum	& Нажать кнопку <<Добавить>> в табличной части <<Товары>>  & Открылась форма выбора справочника <<Номенклатура>>  &  \\
    \hline
    \Rownum	& Выбрать из справочника <<Номенклатура>> элемент с кодом <<00000013>> - <<КЕГ (50 л)>> & Заполнились поля в табличной части <<Код>>, <<Артикул>>, <<Номенклатура>>, <<Ед.изм>> &  \\
    \hline
    \Rownum	& Заполнить поле <<Количество>> значением <<1>>  & Заполнилось поле <<Количество>> &  \\
    \hline
    \Rownum	& Не заполнять поле <<Цена>>, если поле <<Цена>> заполнено, очистить его.  & Поле <<Цена>> не содержит значения &  \\
    \hline
    \Rownum	& Нажать кнопку \keys{Провести и закрыть} & 1. Программа выдает сообщение о неудачи проведения документа;\par 2. При закрытии окна сообщения в строке сообщений появился текст ошибке с информацией, что в  документ отсутствует  цена с указанием номеров строк &  \\
    %****************************************************************************************************





    %****************************************************************************************************
    \hline
    \multicolumn{4}{|c|}{\textbf{\textit{Себестоимость}}} \\
    \hline
    \hline
    \Rownum &Перейти в раздел Закупки, выбрать <<Перемещения товаров>>.  & 1. Открылся список документов  <<Перемещения товаров>>;\par
    2. Отображаются все документы &  \\
    \hline
    \Rownum & Создать новый документ по кнопке \keys{Создать}  & 1. Открылась форма создания документа;\par
    2. По умолчанию в открывшейся форме заполнено поле <<Магазин отправитель>> &  \\
    \hline
    \Rownum & Заполнить реквизит <<Склад отправитель>> значением соответствующим выбранному магазину &Заполнен <<Склад отправитель>> и <<Организация отправитель>> ;    &  \\
    \hline
    \Rownum	& Заполнить реквизит <<Магазин получатель>> значением <<74. Лазурная 27/2, Новосибирск>> & Заполны реквизиты: 1. <<Магазин получатель>> значением <<74. Лазурная 27/2, Новосибирск>>;\par
    2. <<Склад получатель>> значением <<74. Лазурная 27/2, Новосибирск>>;\par
    3. <<Организация получатель>> значением <<ООО "КРЮГЕР ХАУС" КОЛЬЦОВО (Новосибирск, Лазурная ул, 27/2)>>  &  \\
    \hline
    \Rownum	& Нажать кнопку <<Добавить>> в табличной части <<Товары>>  & Открылась форма выбора справочника <<Номенклатура>>  &  \\
    \hline
    \Rownum	& Выбрать из справочника <<Номенклатура>> элемент не помеченный на удаление & Заполнились поля в табличной части <<Код>>, <<Артикул>>, <<Номенклатура>>, <<Ед.изм>>, <<Цена>> если у позиции номенклатуры установлена цена &  \\
    \hline
    \Rownum	&Заполнить поле <<Количество>> значением <<1>>  & Заполнилось поле <<Количество>> &  \\
    \hline
    \Rownum	& Если не заполнено, заполнить поле <<Цена>> значением <<1>>  & Заполнилось поле <<Цена>> &  \\
    \hline
    \Rownum	& Нажать кнопку \keys{Провести} &  Документ проводится без ошибок &  \\
    \hline
    \Rownum	& Выбрать команду <<Движения документа>> & Открылся отчет по движениям документа &  \\
    \hline
    \Rownum	& Найти в отчете движения по регистру <<Себестоимость номенклатуры>> & Движения документа по регистру сведений <<Себестоимость номенклатуры>> присутствуют. Измерения <<Период>>, <<Магазин>>, <<Номенклатура>> заполнены. Значение ресурса <<Цена>> равно значению установленному в документе  &  \\
    \hline
    %****************************************************************************************************







    %****************************************************************************************************
    \hline
    \hline
    \multicolumn{4}{|c|}{\textbf{\textit{Заполнение цен в документе}}} \\
    \hline
    \hline
    \Rownum &Перейти в раздел Закупки, выбрать <<Перемещения товаров>>.  & 1. Открылся список документов  <<Перемещения товаров>>;\par
    2. Отображаются все документы &  \\
    \hline
    \Rownum & Создать новый документ по кнопке \keys{Создать}  & 1. Открылась форма создания документа;\par
    2. По умолчанию в открывшейся форме заполнено поле <<Магазин отправитель>> &  \\
    \hline
    \Rownum & Заполнить реквизит <<Склад отправитель>> значением соответствующим выбранному магазину &Заполнен <<Склад отправитель>> и <<Организация отправитель>> ;    &  \\
    \hline
    \Rownum	& Заполнить реквизит <<Магазин получатель>> значением <<74. Лазурная 27/2, Новосибирск>> & Заполны реквизиты: 1. <<Магазин получатель>> значением <<74. Лазурная 27/2, Новосибирск>>;\par
    2. <<Склад получатель>> значением <<74. Лазурная 27/2, Новосибирск>>;\par
    3. <<Организация получатель>> значением <<ООО "КРЮГЕР ХАУС" КОЛЬЦОВО (Новосибирск, Лазурная ул, 27/2)>>  &  \\
    \hline
    \Rownum	& Нажать кнопку <<Добавить>> в табличной части <<Товары>>  & Открылась форма выбора справочника <<Номенклатура>>  &  \\
    \hline
    \Rownum	&  Выбрать из справочника <<Номенклатура>> элемент с кодом <<МА000091>> - <<Пакет Крюгер Хаус большой>> & Заполнились поля в табличной части <<Код>>, <<Артикул>>, <<Номенклатура>>, <<Ед.изм>>, <<Цена>> если у позиции номенклатуры установлена цена &  \\
    \hline
    \Rownum	& Заполнить поле <<Количество>> значением <<1>>  & Заполнилось поле <<Количество>> &  \\
    \hline
    \Rownum	& Нажать на кнопку \keys{Заполнить цены} & Выпадающий список содержит только значение <<По себестоимости>> & \\
    \hline
    \Rownum	& Выбрать пункт <<По себестоимости>> & 1. В строке документа перезаполнится цена на текущую позицию.;\par
    2. Значение реквизита цена станет равно <<0,01>> &
     \\
    \hline

    %****************************************************************************************************





%    \hline
%    \Rownum	& test &  &  \\ %\nopagebreak Для запрещения разбиения страниц применяется команда \nopagebreak сразу после двух слешей в конце строчки.
%    \hline
\end{longtable}