\subsection{Формирование документа Перемещение товаров}



\renewcommand{\arraystretch}{1.8} %% расстояние между строками таблицы
%\begin{landscape}
\begin{longtable}{|p{0.02\linewidth}|p{0.3\linewidth}|p{0.3\linewidth}|p{0.3\linewidth}|}
    %  {|c|c|l|c|}
    \hline
    № & \textbf{Действие} & \textbf{Ожидаемый результат} & \textbf{Фактический результат} \\
    %****************************************************************************************************
    \hline
    \hline
    \endhead
    \multicolumn{4}{|c|}{\textbf{\textit{Проверка на номенклатуру помеченную на удаление}}} \\
    \hline
    \hline
    \Rownum & Проверить, что включена константа <<крюПроверятьПомеченнуюНаУдалениеВДокументах>>  & &  \\
    \hline
    \Rownum &Перейти в раздел Закупки, выбрать <<Перемещения товаров>>.  & 1. Открылся список документов  <<Перемещения товаров>>;\par
    2. Отображаются все документы &  \\
    \hline
    \Rownum & Создать новый документ по кнопке \keys{Создать}  & 1. Открылась форма создания документа;\par
    2. По умолчанию в открывшейся форме заполнено поле <<Магазин отправитель>> &  \\
    \hline
    \Rownum & Заполнить реквизит <<Склад отправитель>> значением соответствующим выбранному магазину &Заполнен <<Склад отправитель>> и <<Организация отправитель>> ;    &  \\
    \hline
    \Rownum	& Заполнить реквизит <<Магазин получатель>> значением <<74. Лазурная 27/2, Новосибирск>> & Заполны реквизиты: 1. <<Магазин получатель>> значением <<74. Лазурная 27/2, Новосибирск>>;\par
    2. <<Склад получатель>> значением <<74. Лазурная 27/2, Новосибирск>>;\par
    3. <<Организация получатель>> значением <<ООО "КРЮГЕР ХАУС" КОЛЬЦОВО (Новосибирск, Лазурная ул, 27/2)>>  &  \\
    \hline
     \Rownum	& Нажать кнопку <<Добавить>> в табличной части <<Товары>>  & Откроется форма выбора справочника <<Номенклатура>>  &  \\
    \hline
    \Rownum	& Выбрать из справочника <<Номенклатура>> элемент помеченный на удаление & Заполнились поля в табличной части <<Код>>, <<Артикул>>, <<Номенклатура>>, <<Ед.изм>> &  \\
    \hline
    \Rownum	&Заполнить поле <<Количество>> значением <<1>>  & Заполнилось поле <<Количество>> &  \\
    \hline
    \Rownum	& Заполнить поле <<Цена>> значением <<1>>  & Заполнилось поле <<Цена>> &  \\
    \hline
    \Rownum	& Нажать кнопку \keys{Провести и закрыть} & 1. Программа выдает сообщение о неудачи проведения документа;\par 2. При закрытии окна сообщения в строке сообщений появляется текст ошибке с информацией, что документ содержит удаленную номенклатуру с указанием номеров строк и наименований &  \\
    %****************************************************************************************************

    %****************************************************************************************************

    %****************************************************************************************************

    %****************************************************************************************************
    \hline
    \hline
    \multicolumn{4}{|c|}{\textbf{\textit{Контроль тары и оборудования}}} \\
    \hline
    \hline
    \Rownum & Проверить, что включена константа <<крюКонтролироватьВозвратнуюТару>>  & &  \\
    \hline
    \Rownum &Перейти в раздел Закупки, выбрать <<Перемещения товаров>>.  & 1. Открылся список документов  <<Перемещения товаров>>;\par
   2. Отображаются все документы &  \\
   \hline
   \Rownum & Создать новый документ по кнопке \keys{Создать}  & 1. Открылась форма создания документа;\par
   2. По умолчанию в открывшейся форме заполнено поле <<Магазин отправитель>> &  \\
   \hline
   \Rownum & Заполнить реквизит <<Склад отправитель>> значением соответствующим выбранному магазину &Заполнен <<Склад отправитель>> и <<Организация отправитель>> ;    &  \\
   \hline
   \Rownum	& Заполнить реквизит <<Магазин получатель>> значением <<74. Лазурная 27/2, Новосибирск>> & Заполны реквизиты: 1. <<Магазин получатель>> значением <<74. Лазурная 27/2, Новосибирск>>;\par
   2. <<Склад получатель>> значением <<74. Лазурная 27/2, Новосибирск>>;\par
   3. <<Организация получатель>> значением <<ООО "КРЮГЕР ХАУС" КОЛЬЦОВО (Новосибирск, Лазурная ул, 27/2)>>  &  \\
   \hline
   \Rownum	& Нажать кнопку <<Добавить>> в табличной части <<Товары>>  & Откроется форма выбора справочника <<Номенклатура>>  &  \\
   \hline
    \Rownum	& Выбрать из справочника <<Номенклатура>> элемент с кодом <<00000013>> - <<КЕГ (50 л)>> & Заполнились поля в табличной части <<Код>>, <<Артикул>>, <<Номенклатура>>, <<Ед.изм>> &  \\
    \hline
    \Rownum	&Заполнить поле <<Количество>> значением <<1>>  & Заполнилось поле <<Количество>> &  \\
    \hline
    \Rownum	& Заполнить поле <<Цена>> значением <<1>>  & Заполнилось поле <<Цена>> &  \\
    \hline
    \Rownum	& Нажать кнопку \keys{Провести и закрыть} & 1. Программа выдает сообщение о неудачи проведения документа;\par 2. При закрытии окна сообщения в строке сообщений появляется текст ошибке с информацией, что документ содержит возвратную тару или оборудование с указанием номеров строк и наименований &  \\
    %****************************************************************************************************

    %****************************************************************************************************
    \hline
    \hline
    \multicolumn{4}{|c|}{\textbf{\textit{Контроль наличия уены в табличной части <<Товары>>}}} \\
    \hline
    \hline
    \Rownum & Проверить, что включена константа <<крюОчищатьНДСВВозвратеПоставщику>>  & &  \\
    \hline
    \Rownum &Перейти в раздел Закупки, выбрать <<Возвраты товаров поставщикам>>.  & 1. Открылся список документов  <<Возвраты товаров поставщикам>>;\par
    2. Отображаются все документы &  \\
    \hline
    \Rownum & Создать новый документ по кнопке \keys{Создать}  & 1. Открылась форма нового документа;\par
    2. По умолчанию в открывшейся форме заполнено поле <<Магазин>>\par
    3. Значение реквизитов <<ЦенаВключаетНДС>> и <<УчитыватьНДС>> на вкладке  <<Дополнительно>> равно <<Ложь>> &  \\
    \hline
    %****************************************************************************************************

    %****************************************************************************************************
    \hline
    \hline
    \multicolumn{4}{|c|}{\textbf{\textit{Печатная форма ТОРГ 12}}} \\
    \hline
    \hline
    \Rownum &Перейти в раздел Закупки, выбрать <<Возвраты товаров поставщикам>>.  & 1. Открылся список документов  <<Возвраты товаров поставщикам>>;\par
    2. Отображаются все документы &  \\
    \hline
    \Rownum & Открыть любой существующий документ  & 1. Открылась форма существующего документа;\par
    &  \\
    \hline
    \Rownum	& По кнопке выбора печатных форм выбрать <<ТОРГ-12(Товарная накладная на возврат)>>  & Сформировалась печатная форма &  \\
    \hline
    \Rownum	& Проверить представление организации и поле <<Вид операции>> в шапке & 1. В представлении организации присутствует КПП организации\par
    2. Поле <<Вид операции>> заполнено значением <<Возврат>>  &  \\


    %****************************************************************************************************





    \hline
    \Rownum	& test &  &  \\ %\nopagebreak Для запрещения разбиения страниц применяется команда \nopagebreak сразу после двух слешей в конце строчки.
    \hline
\end{longtable}