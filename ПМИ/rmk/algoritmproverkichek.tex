\begin{longtable}{|p{0.02\linewidth}|p{0.3\linewidth}|p{0.3\linewidth}|p{0.3\linewidth}|}
    %  {|c|c|l|c|}
    \hline
    № & \textbf{Действие} & \textbf{Ожидаемый результат} & \textbf{Фактический результат} \\
    %****************************************************************************************************
    \hline
    \hline
    \endhead

   %****************************************************************************************************

    \multicolumn{4}{|c|}{\textbf{\textit{Тестирование алгоритма проверки чеков ККМ}}} \\
\hline
\hline
\Rownum & Запустить конфигурацию магазина выбрав пользователя <<Абрамовская Е. (кассир)>> & 1.Открылся общий интерфейс программы;\par
2. Отображаются разделы <<Главное>> и <<Продажи>>;\par
3. Открылась обработка <<Рабочее место кассира>> ;\par
4. Открылась форма <<Ошибка непроведенных чеков>> с сообщением об ошибке <<Есть не обработанные чеки ККМ. Необходимо зайти «Регистрация продаж» -> Кнопка «Проверить Чеки ККМ» и проанализировать почему чек не пробит>> &  \\
\hline
\Rownum & Нажать кнопку \keys{Закрыть} на форме с описанием ошибки  & Форма с описанием ошибки закрылась
&  \\
\hline
\Rownum	& Нажать кнопку \keys{Регистрация продаж} в меню РМК & 1. Форма меню РМК закрыта;\par
2. Открыта форма с информационным сообщением для кассиров;\par
3. Кнопка \keys{ОК} в нижней части формы недоступна &  \\
\hline
\Rownum	& Отметить чек бокс с надписью <<Мною прочитано и понято>> & 1. Чек бокс с надписью <<Мною прочитано и понято>> отмечен ;\par
2. Кнопка \keys{ОК} в нижней части формы доступна &   \\
\hline
\Rownum	& Нажать кнопку \keys{ОК} в нижней части формы & 1. Форма с информационным сообщением для кассиров закрыта.;\par
2. Открыта форма Рабочего места кассира  &  \\
\hline
\Rownum	& Нажать кнопку \keys{Проверить чеки ККМ} в верхней части формы  &1.  Открыта форма <<Проверка чека ККМ>>;\par
2. В табличной части формы содержатся две строки с номерами чеков в первой колонке  и описанием проблемы: <<Наличная оплата. Фискальный чек возможно был распечатан. Проверка будет выполнена после нажатия на кнопку «Распечатать фискальный чек». При необходимости чек будет распечатан.>> во второй;\par
3. Внизу формы в окне сообщений отображаются две строки с текстом <<В первую очередь должен быть обработан ---> Чек>> и номером чека &  \\
\hline
\Rownum	& Закрыть окно сообщений  & Окно сообщений закрыто &  \\
\hline
\Rownum	& В табличной части выделить строку содержащую чек с номером, который был указан в строке сообщений  & В табличной части Выделена строка с чеком &  \\
\hline
\Rownum	& Нажать кнопку \keys{Распечатать фискальный чек}, кнопка с изображением фискального регистратора, справа от табличной части, третья сверху.  & 1. Фискальный регистратор распечатал текущий чек;\par
2. В табличной части отсталась одна выделенная строка с чеком &  \\
\hline
\Rownum	& Нажать кнопку \keys{Команда выбрать}, кнопка с изображением зеленого треугольника, справа от табличной части, четвертая сверху. или горячую клавишу \keys{Ctrl + Enter}   & Открылась форма текущего чека  &  \\
\hline
\Rownum	& Закрыть форму текущего чека либо просто закрыв форму, либо нажав кнопку \keys{Провести и закрыть}  & Форма чека закрылась&  \\
\hline
\Rownum	& Нажать кнопку \keys{Распечатать фискальный чек}, кнопка с изображением фискального регистратора, справа от табличной части, третья сверху.  & 1. Фискальный регистратор распечатал текущий чек;\par
2. Табличная часть очистилась &  \\
\hline
\Rownum	& Закрыть форму <<Проверка чека ККМ>> &  Форма <<Проверка чека ККМ>> закрыта &  \\
\hline


\Rownum & Закрыть рабочее место кассира нажав последовательно горячие клавиши \keys{F10} - \keys{F12}  &1.  Открылось меню <<Рабочего места кассира>>;\par
2. В меню доступны все кнопки   &  \\
\hline
\Rownum & Нажать кнопку \keys{Завершение работы}   & Конфигурация закрылась
&  \\
\hline
%****************************************************************************************************


\end{longtable}