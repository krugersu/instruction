\begin{longtable}{|p{0.02\linewidth}|p{0.3\linewidth}|p{0.3\linewidth}|p{0.3\linewidth}|}
    %  {|c|c|l|c|}
    \hline
    № & \textbf{Действие} & \textbf{Ожидаемый результат} & \textbf{Фактический результат} \\
    %****************************************************************************************************
    \hline
    \hline
    \endhead

    \multicolumn{4}{|c|}{\textbf{\textit{Проверка запуска РМК при старте}}} \\
    \hline
    \Rownum & Запустить конфигурацию магазина  & 1.Открылся общий интерфейс программы;\par
    2. Отображаются все доступные разделы  &  \\
    \hline
    \Rownum & Перейти в раздел <<Администрирование>>   & 1. Открылся отдел <<Администрирование>>
    &  \\

    \hline
    \Rownum	& Выбрать пункт <<Настройки пользователей и прав>>  & Открылся раздел <<Настройки пользователей и прав>>   &  \\
    \hline
    \Rownum	& Выбрать пункт  <<Пользователи>> & Открылся раздел <<Пользователи>> &  \\
    \hline
    \Rownum & В списке пользователей открыть пользователя с именем  <<Абрамовская Екатерина>> & Открылась форма элемента справочника  <<Пользователи>> со значением <<Абрамовская Екатерина>> &  \\
    \hline
    \Rownum	& Перейти в редактирование закладки <<Группы>> & 1. Открылся список групп, в которые включен пользователь;\par
    2. В списке выбранна группа <<Кассиры>>  &  \\
    \hline
    \Rownum	& Снять выбор с группы <<Кассиры>> и установить выбор на группу <<Заведующие магазинами>>  & Снят выбор с группы <<Кассиры>> и установлен выбор на группу <<Заведующие магазинами>>  &  \\
    \hline
    \Rownum	& Нажать кнопку \keys{Записать}  & Изменения сохранились &  \\
    \hline
    \Rownum	& Перейти в редактирование закладки <<Основное>>  & Открылась форма с основными настройками пользователя  &  \\
    \hline
    \Rownum	& Нажать кнопку \keys{Записать и закрыть} & Закрылась форма элемента справочника  <<Пользователи>> со значением <<Абрамовская Екатерина>>  &  \\
    \hline
    \Rownum	& Закрыть конфигурацию  & Конфигурация закрылась  &  \\
    \hline
    \Rownum & Запустить конфигурацию магазина выбрав пользователя <<Абрамовская Е. (кассир)>> & 1.Открылся общий интерфейс программы;\par
    2. Отображаются все доступные разделы;\par
    3. Обработка <<Рабочее место кассира>> не открылась &  \\
    \hline
    \hline
    \Rownum & Перейти в раздел <<Администрирование>>   & 1. Открылся отдел <<Администрирование>>
    &  \\

    \hline
    \Rownum	& Выбрать пункт <<Настройки пользователей и прав>>  & Открылся раздел <<Настройки пользователей и прав>>   &  \\
    \hline
    \Rownum	& Выбрать пункт  <<Пользователи>> & Открылся раздел <<Пользователи>> &  \\
    \hline
    \Rownum & В списке пользователей открыть пользователя с именем  <<Абрамовская Екатерина>> & Открылась форма элемента справочника  <<Пользователи>> со значением <<Абрамовская Екатерина>> &  \\
    \hline
    \Rownum	& Перейти в редактирование закладки <<Группы>> & 1. Открылся список групп, в которые включен пользователь;\par
    2. В списке выбранна группа <<Заведующие магазинами>>  &  \\
    \hline
    \Rownum	& Снять выбор с группы <<Заведующие магазинами>> и установить выбор на группу <<Кассиры>>  & Снят выбор с группы <<Заведующие магазинами>> и установлен выбор на группу <<Кассиры>>  &  \\
    \hline
    \Rownum	& Нажать кнопку \keys{Записать}  & Изменения сохранились &  \\
    \hline
    \Rownum	& Перейти в редактирование закладки <<Основное>>  & Открылась форма с основными настройками пользователя  &  \\
    \hline

    \Rownum	& Нажать кнопку \keys{Записать и закрыть} & Закрылась форма элемента справочника  <<Пользователи>> со значением <<Абрамовская Екатерина>>  &  \\
    \hline
    \Rownum	& Закрыть конфигурацию  & Конфигурация закрылась  &  \\
    \hline
    \Rownum & Запустить конфигурацию магазина выбрав пользователя <<Абрамовская Е. (кассир)>> & 1.Открылся общий интерфейс программы;\par
    2. Отображаются разделы <<Главное>> и <<Продажи>>;\par
    3. Открылась обработка <<Рабочее место кассира>>  &  \\
    \hline
    %****************************************************************************************************

\end{longtable}