\begin{longtable}{|p{0.02\linewidth}|p{0.3\linewidth}|p{0.3\linewidth}|p{0.3\linewidth}|}
    %  {|c|c|l|c|}
    \hline
    № & \textbf{Действие} & \textbf{Ожидаемый результат} & \textbf{Фактический результат} \\
    %****************************************************************************************************
    \hline
    \hline
    \endhead

   %****************************************************************************************************

     \multicolumn{4}{|c|}{\textbf{\textit{При возникновении ошибок прежде чем пробивать новый чек, необходимо закончить работу со старым}}} \\
\hline
\hline
\Rownum & Запустить конфигурацию магазина выбрав пользователя <<Абрамовская Е. (кассир)>> & 1.Открылся общий интерфейс программы;\par
2. Отображаются разделы <<Главное>> и <<Продажи>>;\par
3. Открылась обработка <<Рабочее место кассира>> ;\par
4. Открылась форма <<Ошибка непроведенных чеков>> с сообщением об ошибке <<Есть не обработанные чеки ККМ. Необходимо зайти «Регистрация продаж» -> Кнопка «Проверить Чеки ККМ» и проанализировать почему чек не пробит>> &  \\
\hline
\Rownum & Нажать кнопку \keys{Закрыть} на форме с описанием ошибки  & Форма с описанием ошибки закрылась
&  \\
\hline
\Rownum	& Нажать кнопку \keys{Регистрация продаж} в меню РМК & 1. Форма меню РМК закрыта;\par
2. Открыта форма с информационным сообщением для кассиров;\par
3. Кнопка \keys{ОК} в нижней части формы недоступна &  \\
\hline
\Rownum	& Отметить чек бокс с надписью <<Мною прочитано и понято>> & 1. Чек бокс с надписью <<Мною прочитано и понято>> отмечен ;\par
2. Кнопка \keys{ОК} в нижней части формы доступна &   \\
\hline
\Rownum	& Нажать кнопку \keys{ОК} в нижней части формы & 1. Форма с информационным сообщением для кассиров закрыта.;\par
2. Открыта форма Рабочего места кассира  &  \\
\hline
\Rownum	& Нажать кнопку \keys{Поиск (F11)} в верхней части формы или горячую клавишу \keys{F11} & Открыта форма поиска и подбора товара в РМК &  \\
\hline
\Rownum	& Выбрать поиск по наименованию в выпадающем списке <<Поиск>> верхней части формы  & Выбран режим поиска по наименованию &  \\
\hline
\Rownum	& В поле поиска ввести <<Трое в лодке светлое>>  & В табличной части <<Товары>> осталась номенклатура, в наименовании которой содержится <<Трое в лодке светлое>> &  \\
\hline
\Rownum	& В табличной части <<Товары>> выбрать позицию с артикулом <<11697>>  & 1. Форма поиска закрылась;\par
2. Под табличной частью <<Товары>> появилось сообщение об ошибке содержащее номер чека, вид оплаты и сообщением <<Для дальнейшей работы на данной кассе требуется устранить указанную проблему!>>, ниже выводится еще одно сообщение <<Нажмите кнопку «Проверить Чеки ККМ»>>
&  \\
\hline
\Rownum	& Закрыть сообщение об ошибке & 1. Сообщение об ошибке закрылось;\par
2. Табличная часть <<Товары>> не содержит строк. &  \\
\hline

\Rownum & Закрыть рабочее место кассира нажав последовательно горячие клавиши \keys{F10} - \keys{F12}  &1.  Открылось меню <<Рабочего места кассира>>;\par
2. В меню доступны только кнопки: \keys{Регистрация продаж}, \keys{Закрыть}, \keys{Завершение работы}   &  \\
\hline
\Rownum & Нажать кнопку \keys{Завершение работы}   & Конфигурация закрылась
&  \\
\hline
%****************************************************************************************************


\end{longtable}