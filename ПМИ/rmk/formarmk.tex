\begin{longtable}{|p{0.02\linewidth}|p{0.3\linewidth}|p{0.3\linewidth}|p{0.3\linewidth}|}
    %  {|c|c|l|c|}
    \hline
    № & \textbf{Действие} & \textbf{Ожидаемый результат} & \textbf{Фактический результат} \\
    %****************************************************************************************************
    \hline
    \hline
    \endhead

   %****************************************************************************************************

  \multicolumn{4}{|c|}{\textbf{\textit{Изменена основная форма РМК, уменьшен размер}}} \\
  \hline
  \hline
  \Rownum & Запустить конфигурацию магазина выбрав пользователя <<Абрамовская Е. (кассир)>> & 1.Открылся общий интерфейс программы;\par
  2. Отображаются разделы <<Главное>> и <<Продажи>>;\par
  3. Открылась обработка <<Рабочее место кассира>>  &  \\

  \hline
  \Rownum	& Нажать кнопку \keys{Регистрация продаж} в меню РМК & 1. Форма меню РМК закрыта;\par
  2. Открыта форма с информационным сообщением для кассиров;\par
  3. Кнопка \keys{ОК} в нижней части формы недоступна &  \\
  \hline
  \Rownum	& Отметить чек бокс с надписью <<Мною прочитано и понято>> & 1. Чек бокс с надписью <<Мною прочитано и понято>> отмечен ;\par
  2. Кнопка \keys{ОК} в нижней части формы доступна &   \\
  \hline
  \Rownum	& Нажать кнопку \keys{ОК} в нижней части формы & 1. Форма с информационным сообщением для кассиров закрыта.;\par
  2. Открыта форма Рабочего места кассира;\par
  3. Форма имеет корректные размеры и вмещается на экран без вертикальных и горизонтальных полос прокрутки  &  \\
  \hline
  \Rownum & Закрыть рабочее место кассира нажав последовательно горячие клавиши \keys{F10} - \keys{F12}  & Открылось меню <<Рабочего места кассира>>;\par
  &  \\
  \hline
  \Rownum & Нажать кнопку \keys{Завершение работы}   & Конфигурация закрылась
  &  \\
  \hline
  %****************************************************************************************************

\end{longtable}