\begin{longtable}{|p{0.02\linewidth}|p{0.3\linewidth}|p{0.3\linewidth}|p{0.3\linewidth}|}
    %  {|c|c|l|c|}
    \hline
    № & \textbf{Действие} & \textbf{Ожидаемый результат} & \textbf{Фактический результат} \\
    %****************************************************************************************************
    \hline
    \hline
    \endhead

   %****************************************************************************************************

    \multicolumn{4}{|c|}{\textbf{\textit{Проверка алгоритма закрытия смены и опроса кассиров без ввода итоговых сумм}}} \\
\hline
\hline
\Rownum & Запустить конфигурацию магазина  & 1.Открылся общий интерфейс программы;\par
2. Отображаются все доступные разделы  &  \\
\hline
\Rownum & Проверить, что включена константа <<КрюОпросКассираПриОкончанииСмены>>  & &  \\
\hline
\Rownum & Проверить, что включена константа <<КрюОпросКассировСравниватьБезнал>>  & &  \\
\hline
\Rownum & Проверить, что включена константа <<КрюОпросКассировСравниватьОРП>>  & &  \\
\hline
\Rownum	& Закрыть конфигурацию  & Конфигурация закрыта  &  \\
\hline



\Rownum & Запустить конфигурацию магазина выбрав пользователя <<Абрамовская Е. (кассир)>> & 1.Открылся общий интерфейс программы;\par
2. Отображаются разделы <<Главное>> и <<Продажи>>;\par
3. Открылась обработка <<Рабочее место кассира>>  &  \\
\hline
\Rownum	& Нажать кнопку \keys{Регистрация продаж} в меню РМК & 1. Форма меню РМК закрыта;\par
2. Открыта форма с информационным сообщением для кассиров;\par
3. Кнопка \keys{ОК} в нижней части формы недоступна &  \\
\hline
\Rownum	& Отметить чек бокс с надписью <<Мною прочитано и понято>> & 1. Чек бокс с надписью <<Мною прочитано и понято>> отмечен ;\par
2. Кнопка \keys{ОК} в нижней части формы доступна &   \\
\hline
\Rownum	& Нажать кнопку \keys{ОК} в нижней части формы & 1. Форма с информационным сообщением для кассиров закрыта.;\par
2. Открыта форма Рабочего места кассира  &  \\
\hline
\Rownum	& Нажать кнопку \keys{Поиск (F11)} в верхней части формы или горячую клавишу \keys{F11} & Открыта форма поиска и подбора товара в РМК &  \\
\hline
\Rownum	& Выбрать поиск по наименованию в выпадающем списке <<Поиск>> верхней части формы  & Выбран режим поиска по наименованию &  \\
\hline
\Rownum	& В поле поиска ввести <<Гренки чесночные>>  & В табличной части <<Товары>> осталась номенклатура, в наименовании которой содержится <<Гренки чесночные>> &  \\
\hline
\Rownum	& В табличной части <<Товары>> выбрать позицию с артикулом <<11432>>  & 1. Форма поиска закрылась;\par
2. В табличную часть <<Товары>> формы рабочего места кассира добавлена позиция с артикулом <<11432>> с количеством <<1>> и установленной ценой &  \\
\Rownum	& Установить количество позиции с артикулом <<11432>> равным <<0,150>>  & 1. Количество позиции с артикулом <<11432>> - <<Гренки чесночные>> изменилось на значение <<0,150>>&  \\
\hline
%\Rownum	& Нажать кнопку \keys{Поиск (F11)} в верхней части формы или горячую клавишу \keys{F11} & Открыта форма поиска и подбора товара в РМК &  \\
%\hline
%\Rownum	& Выбрать поиск по наименованию в выпадающем списке <<Поиск>> верхней части формы  & Выбран режим поиска по наименованию &  \\
%\hline
%\Rownum	& В поле поиска ввести <<Трое в лодке светлое>>  & В табличной части <<Товары>> осталась номенклатура, в наименовании которой содержится <<Трое в лодке светлое>> &  \\
%\hline
%\Rownum	& В табличной части <<Товары>> выбрать позицию с артикулом <<11697>>  & 1. Форма поиска закрылась;\par
%2. В табличную часть <<Товары>> формы рабочего места кассира добавлена позиция с артикулом <<11697>> с количеством <<1>> и установленной ценой &  \\
%\hline
%\Rownum	& Нажать \keys{Ctrl} + \keys{M}   & 1. В табличную часть <<Товары>> формы рабочего места кассира добавлена позиция с артикулом <<10340>> - <<ПЭТ бутылка 1,5л>>  &  \\
%\hline
%\Rownum	& Установить количество позиции с артикулом <<10340>> равным <<1>>  & 1. Количество позиции с артикулом <<11697>> - <<Пиво Трое в лодке светлое 1л>> изменилось на значение <<1.5>>&  \\
%\hline
\Rownum	& Нажать кнопку \keys{Оплата (F8)} в верхней части формы или горячую клавишу \keys{F8}  &  Открыта форма оплаты &  \\
\hline
\Rownum	& Нажать кнопку \keys{Нал.(F6)} справа от поля ввода <<Всего к оплате (руб):>> или горячую клавишу \keys{F6}  & В табличную часть <<Виды оплат>> добавлена строка со значениями полей: <<Вида оплаты>> - <<Наличные>>; <<Сумма>> - рассчитанной суммой&  \\
\hline
\Rownum	& Нажать кнопку \keys{Enter} в области цифровых кнопок или горячую клавишу \keys{Ctrl + Enter}  & 1. Форма оплаты закроется;\par
2. На фискальном регистраторе будет напечатан чек;\par
3. Табличная часть <<Товары>> в форме рабочего места кассира очистится
&  \\
\hline
\Rownum & Закрыть рабочее место кассира нажав последовательно горячие клавиши \keys{F10} - \keys{F12}  &1.  Открылось меню <<Рабочего места кассира>>;\par
2. В меню доступны все кнопки   &  \\
\hline
\Rownum & Нажать кнопку \keys{Закрытие смены}   & Открылось окно информации с вопросом: <<Закрыть смену?>>
&  \\
\hline
\Rownum & Нажать кнопку \keys{Да} в левом нижнем углу окна информации  & 1. Открылось окно с заголовком <Проверка непробитых чеков: Закрытие кассовой смены>>\par
2. На форме присутствуют кнопки: <<Отчет без гашения>>, <Закрытие смены>>, <<Отмена закрытия смены>>
3. Присутствует информациях о суммах продаж и сторно по данным системы.
&  \\
\Rownum & Нажать кнопку \keys{Закрытие смены}   & Открылось окно ошибки с сообщением о невозможности провести сверку итогов
&  \\
\hline
\Rownum & Нажать кнопку \keys{Закрыть}  на форме с информацией об ошибке  & Форма с сообщением об ошибке закрылась
&  \\
\hline
\Rownum & Закрыть форму <<Проверка не пробитых чеков>>   & 1. Зарыта форма <<Проверка не пробитых чеков>>;\par
2. На экране форма  с заголовком <<РМК (управляемый режим)>>
&  \\
\hline

\Rownum &  Нажать кнопку \keys{Готово} в левом нижнем углу формы &  Снизу в окне сообщения информация об ошибке: <<Обнаружено несовпадения результатов. Повторите ввод сумм.>>
&  \\
\hline

\Rownum &  Закрыть окно с сообщением об ошибке & 1. Окно с сообщением об ошибке закрыто
&  \\
\hline
\Rownum &  Нажать кнопку \keys{Готово} в левом нижнем углу формы & 1. Форма <<РМК (управляемый режим)>> закрыта;\par
2. Открыто информационное окно <<Кассовая смена в системе закрыта:>>;\par
3. В текстовом поле формы содержится сообщение: <<Отчет о розничных продажах не сформирован>>, сумма выемки и остаток в кассе, <<Z-отчет распечатан>>
&  \\
\hline

\Rownum &  Закрыть информационное окно & 1. Окно с информацией закрыто;\par
2. На экране открыта форма документа <<Отчет о розничных продажах>> \par
3. Строка в табличной части <<Товары>> содержит товар с артикулом <<11432>>  - <<Гренки чесночные>> с количеством <<0,150>>
&  \\
\hline

\Rownum &  Закрыть форму документа <<Отчет о розничных продажах>> & 1. Форма документа закрыта;\par
2. Открыта форма <<Меню>> РМК;\par
3. Снизу в окне сообщения информация об ошибке:<<Контрольные суммы по данным опроса кассира не совпадают! Обратитесь к товароведу.>>
&  \\
\hline
\Rownum & Нажать кнопку \keys{Завершение работы}   & Конфигурация закрылась
&  \\
\hline


%****************************************************************************************************



    \multicolumn{4}{|c|}{\textbf{\textit{Проверка алгоритма закрытия смены и опроса кассиров с вводом не верных сумм}}} \\
\hline
\hline
\Rownum & Запустить конфигурацию магазина  & 1.Открылся общий интерфейс программы;\par
2. Отображаются все доступные разделы  &  \\
\hline
\Rownum & Проверить, что включена константа <<КрюОпросКассираПриОкончанииСмены>>  & &  \\
\hline
\Rownum & Проверить, что включена константа <<КрюОпросКассировСравниватьБезнал>>  & &  \\
\hline
\Rownum & Проверить, что включена константа <<КрюОпросКассировСравниватьОРП>>  & &  \\
\hline
\Rownum	& Закрыть конфигурацию  & Конфигурация закрыта  &  \\
\hline



\Rownum & Запустить конфигурацию магазина выбрав пользователя <<Абрамовская Е. (кассир)>> & 1.Открылся общий интерфейс программы;\par
2. Отображаются разделы <<Главное>> и <<Продажи>>;\par
3. Открылась обработка <<Рабочее место кассира>>  &  \\
\hline
\Rownum	& Нажать кнопку \keys{Регистрация продаж} в меню РМК & 1. Форма меню РМК закрыта;\par
2. Открыта форма с информационным сообщением для кассиров;\par
3. Кнопка \keys{ОК} в нижней части формы недоступна &  \\
\hline
\Rownum	& Отметить чек бокс с надписью <<Мною прочитано и понято>> & 1. Чек бокс с надписью <<Мною прочитано и понято>> отмечен ;\par
2. Кнопка \keys{ОК} в нижней части формы доступна &   \\
\hline
\Rownum	& Нажать кнопку \keys{ОК} в нижней части формы & 1. Форма с информационным сообщением для кассиров закрыта.;\par
2. Открыта форма Рабочего места кассира  &  \\
\hline
\Rownum	& Нажать кнопку \keys{Поиск (F11)} в верхней части формы или горячую клавишу \keys{F11} & Открыта форма поиска и подбора товара в РМК &  \\
\hline
\Rownum	& Выбрать поиск по наименованию в выпадающем списке <<Поиск>> верхней части формы  & Выбран режим поиска по наименованию &  \\
\hline
\Rownum	& В поле поиска ввести <<Гренки чесночные>>  & В табличной части <<Товары>> осталась номенклатура, в наименовании которой содержится <<Гренки чесночные>> &  \\
\hline
\Rownum	& В табличной части <<Товары>> выбрать позицию с артикулом <<11432>>  & 1. Форма поиска закрылась;\par
2. В табличную часть <<Товары>> формы рабочего места кассира добавлена позиция с артикулом <<11432>> с количеством <<1>> и установленной ценой &  \\
\Rownum	& Установить количество позиции с артикулом <<11432>> равным <<0,150>>  & 1. Количество позиции с артикулом <<11432>> - <<Гренки чесночные>> изменилось на значение <<0,150>>&  \\
\hline

\Rownum	& Нажать кнопку \keys{Оплата (F8)} в верхней части формы или горячую клавишу \keys{F8}  &  Открыта форма оплаты &  \\
\hline
\Rownum	& Нажать кнопку \keys{Нал.(F6)} справа от поля ввода <<Всего к оплате (руб):>> или горячую клавишу \keys{F6}  & В табличную часть <<Виды оплат>> добавлена строка со значениями полей: <<Вида оплаты>> - <<Наличные>>; <<Сумма>> - рассчитанной суммой&  \\
\hline
\Rownum	& Нажать кнопку \keys{Enter} в области цифровых кнопок или горячую клавишу \keys{Ctrl + Enter}  & 1. Форма оплаты закроется;\par
2. На фискальном регистраторе будет напечатан чек;\par
3. Табличная часть <<Товары>> в форме рабочего места кассира очистится
&  \\
\hline
\Rownum & Закрыть рабочее место кассира нажав последовательно горячие клавиши \keys{F10} - \keys{F12}  &1.  Открылось меню <<Рабочего места кассира>>;\par
2. В меню доступны все кнопки   &  \\
\hline
\Rownum & Нажать кнопку \keys{Закрытие смены}   & Открылось окно информации с вопросом: <<Закрыть смену?>>
&  \\
\hline
\Rownum & Нажать кнопку \keys{Да} в левом нижнем углу окна информации  & 1. Открылось окно с заголовком <Проверка непробитых чеков: Закрытие кассовой смены>>\par
2. На форме присутствуют кнопки: <<Отчет без гашения>>, <Закрытие смены>>, <<Отмена закрытия смены>>
3. Присутствует информациях о суммах продаж и сторно по данным системы.
&  \\
\Rownum & Нажать кнопку \keys{Закрытие смены}   & Открылось окно ошибки с сообщением о невозможности провести сверку итогов
&  \\
\hline
\Rownum & Нажать кнопку \keys{Закрыть}  на форме с информацией об ошибке  & Форма с сообщением об ошибке закрылась
&  \\
\hline
\Rownum & Закрыть форму <<Проверка не пробитых чеков>>   & 1. Зарыта форма <<Проверка не пробитых чеков>>;\par
2. На экране форма  с заголовком <<РМК (управляемый режим)>>
&  \\
\hline
\Rownum & В поле ввода <<Общая сумма (выручка, без всяких вычетов), 1>> ввести значение <<11>>   & В поле ввода <<Общая сумма (выручка, без всяких вычетов), 1>> введено значение <<11>> &  \\
\hline
\Rownum &  Нажать кнопку \keys{Готово} в левом нижнем углу формы &  Снизу в окне сообщения информация об ошибке: <<Обнаружено несовпадения результатов. Повторите ввод сумм.>>
&  \\
\hline

\Rownum &  Закрыть окно с сообщением об ошибке & 1. Окно с сообщением об ошибке закрыто
&  \\
\hline
\Rownum &  Нажать кнопку \keys{Готово} в левом нижнем углу формы & 1. Форма <<РМК (управляемый режим)>> закрыта;\par
2. Открыто информационное окно <<Кассовая смена в системе закрыта:>>;\par
3. В текстовом поле формы содержится сообщение: <<Отчет о розничных продажах не сформирован>>, сумма выемки и остаток в кассе, <<Z-отчет распечатан>>
&  \\
\hline

\Rownum &  Закрыть информационное окно & 1. Окно с информацией закрыто;\par
2. На экране открыта форма документа <<Отчет о розничных продажах>> \par
3. Строка в табличной части <<Товары>> содержит товар с артикулом <<11432>>  - <<Гренки чесночные>> с количеством <<0,150>>
&  \\
\hline

\Rownum &  Закрыть форму документа <<Отчет о розничных продажах>> & 1. Форма документа закрыта;\par
2. Открыта форма <<Меню>> РМК;\par
3. Снизу в окне сообщения информация об ошибке:<<Контрольные суммы по данным опроса кассира не совпадают! Обратитесь к товароведу.>>
&  \\
\hline
\Rownum & Нажать кнопку \keys{Завершение работы}   & Конфигурация закрылась
&  \\
\hline


%****************************************************************************************************


    \multicolumn{4}{|c|}{\textbf{\textit{Проверка алгоритма закрытия смены и опроса кассиров с возвратом и вводом  верных сумм}}} \\
\hline
\hline
\Rownum & Запустить конфигурацию магазина  & 1.Открылся общий интерфейс программы;\par
2. Отображаются все доступные разделы  &  \\
\hline
\Rownum & Проверить, что включена константа <<КрюОпросКассираПриОкончанииСмены>>  & &  \\
\hline
\Rownum & Проверить, что включена константа <<КрюОпросКассировСравниватьБезнал>>  & &  \\
\hline
\Rownum & Проверить, что включена константа <<КрюОпросКассировСравниватьОРП>>  & &  \\
\hline
\Rownum	& Закрыть конфигурацию  & Конфигурация закрыта  &  \\
\hline



\Rownum & Запустить конфигурацию магазина выбрав пользователя <<Абрамовская Е. (кассир)>> & 1.Открылся общий интерфейс программы;\par
2. Отображаются разделы <<Главное>> и <<Продажи>>;\par
3. Открылась обработка <<Рабочее место кассира>>  &  \\
\hline
\Rownum	& Нажать кнопку \keys{Регистрация продаж} в меню РМК & 1. Форма меню РМК закрыта;\par
2. Открыта форма с информационным сообщением для кассиров;\par
3. Кнопка \keys{ОК} в нижней части формы недоступна &  \\
\hline
\Rownum	& Отметить чек бокс с надписью <<Мною прочитано и понято>> & 1. Чек бокс с надписью <<Мною прочитано и понято>> отмечен ;\par
2. Кнопка \keys{ОК} в нижней части формы доступна &   \\
\hline
\Rownum	& Нажать кнопку \keys{ОК} в нижней части формы & 1. Форма с информационным сообщением для кассиров закрыта.;\par
2. Открыта форма Рабочего места кассира  &  \\
\hline

%*************************************************************************************************************
%                       ЧЕК С Гренками
%*************************************************************************************************************
\Rownum	& Нажать кнопку \keys{Поиск (F11)} в верхней части формы или горячую клавишу \keys{F11} & Открыта форма поиска и подбора товара в РМК &  \\
\hline
\Rownum	& Выбрать поиск по наименованию в выпадающем списке <<Поиск>> верхней части формы  & Выбран режим поиска по наименованию &  \\
\hline
\Rownum	& В поле поиска ввести <<Гренки чесночные>>  & В табличной части <<Товары>> осталась номенклатура, в наименовании которой содержится <<Гренки чесночные>> &  \\
\hline
\Rownum	& В табличной части <<Товары>> выбрать позицию с артикулом <<11432>>  & 1. Форма поиска закрылась;\par
2. В табличную часть <<Товары>> формы рабочего места кассира добавлена позиция с артикулом <<11432>> с количеством <<1>> и установленной ценой &  \\
\Rownum	& Установить количество позиции с артикулом <<11432>> равным <<0,150>>  & 1. Количество позиции с артикулом <<11432>> - <<Гренки чесночные>> изменилось на значение <<0,150>>&  \\
\hline

\Rownum	& Нажать кнопку \keys{Оплата (F8)} в верхней части формы или горячую клавишу \keys{F8}  &  Открыта форма оплаты &  \\
\hline
\Rownum	& Нажать кнопку \keys{Нал.(F6)} справа от поля ввода <<Всего к оплате (руб):>> или горячую клавишу \keys{F6}  & В табличную часть <<Виды оплат>> добавлена строка со значениями полей: <<Вида оплаты>> - <<Наличные>>; <<Сумма>> - рассчитанной суммой&  \\
\hline
\Rownum	& Нажать кнопку \keys{Enter} в области цифровых кнопок или горячую клавишу \keys{Ctrl + Enter}  & 1. Форма оплаты закроется;\par
2. На фискальном регистраторе будет напечатан  чек;\par
3. Табличная часть <<Товары>> в форме рабочего места кассира очистится
&  \\
\hline

%*************************************************************************************************************
%                       ЧЕК С Пакетом
%*************************************************************************************************************

\Rownum	& Нажать кнопку \keys{Поиск (F11)} в верхней части формы или горячую клавишу \keys{F11} & Открыта форма поиска и подбора товара в РМК &  \\
\hline
\Rownum	& Выбрать поиск по наименованию в выпадающем списке <<Поиск>> верхней части формы  & Выбран режим поиска по наименованию &  \\
\hline
\Rownum	& В поле поиска ввести <<Пакет Крюгер Хаус маленький>>  & В табличной части <<Товары>> осталась номенклатура, в наименовании которой содержится <<Пакет Крюгер Хаус маленький>> &  \\
\hline
\Rownum	& В табличной части <<Товары>> выбрать позицию с артикулом <<10347>>  & 1. Форма поиска закрылась;\par
2. В табличную часть <<Товары>> формы рабочего места кассира добавлена позиция с артикулом <<10347>> с количеством <<1>> и установленной ценой &  \\
\hline

\Rownum	& Нажать кнопку \keys{Оплата (F8)} в верхней части формы или горячую клавишу \keys{F8}  &  Открыта форма оплаты &  \\
\hline
\Rownum	& Нажать кнопку \keys{Нал.(F6)} справа от поля ввода <<Всего к оплате (руб):>> или горячую клавишу \keys{F6}  & В табличную часть <<Виды оплат>> добавлена строка со значениями полей: <<Вида оплаты>> - <<Наличные>>; <<Сумма>> - рассчитанной суммой&  \\
\hline
\Rownum	& Нажать кнопку \keys{Enter} в области цифровых кнопок или горячую клавишу \keys{Ctrl + Enter}  & 1. Форма оплаты закроется;\par
2. На фискальном регистраторе будет напечатано два чека;\par
3. Табличная часть <<Товары>> в форме рабочего места кассира очистится
&  \\
\hline

%*************************************************************************************************************
%                       ЧЕК Возврат пакета
%*************************************************************************************************************


\Rownum	& Нажать кнопку \keys{Возврат (F5)} в области цифровых кнопок или горячую клавишу \keys{F5}  & 1. Открылась форма возврата чека;\par
2. В верхней части формы поля <<С:>> и <<По:>> в качестве значения имеют текущую дату;\par
3. Табличная часть <<Товары>> в форме рабочего места кассира очистится &  \\
\hline
\Rownum	& В табличной части выбрать со старшим номером чека и суммой за пакет  & Выбрана  строка &  \\
\hline
\Rownum	& Нажать кнопку \keys{Команда выбрать}, кнопка с изображением зеленого треугольника, справа от табличной части, первая сверху. или горячую клавишу \keys{Ctrl + Enter}   & Открылась форма <<Выбор причины возврата>> в табличной части присутствует строка: <<Возврат от покупателя>> &  \\
\hline


\Rownum	& Выбрать строку со значеним  <<Возврат от покупателя>>  & Строка со значеним  <<Возврат от покупателя>> выбрана  &  \\
\hline


\Rownum	& Нажать кнопку \keys{Команда выбрать}, кнопка с изображением зеленого треугольника, справа от табличной части, первая сверху. или горячую клавишу \keys{Ctrl + Enter}   & 1.  Форма <<Выбор причины возврата>> закрылась;\par
2. В табличную часть <<Товары>> формы рабочего места кассира добавлена позиция с артикулом <<10347>> - <<Пакет Крюгер Хаус маленький>> с количеством <<1>> и установленной ценой;\par
3. В верхней части формы в области цифровых кнопок кнопка \keys{Возврат (F5)} изменилась на кнопку \keys{Продажа (F5)} и цвет наименования стал красным  &  \\
\hline

\Rownum	& Нажать кнопку \keys{Оплата (F8)} в области цифровых кнопок или горячую клавишу \keys{F8}  & Открыта форма оплаты &  \\
\hline
\Rownum	& Нажать кнопку \keys{Нал.(F6)} справа от поля ввода <<Всего к оплате (руб):>> или горячую клавишу \keys{F6}  & В табличную часть <<Виды оплат>> добавлена строка со значениями полей: <<Вида оплаты>> - <<Наличные>>; <<Сумма>> - рассчитанной суммой&  \\
\hline
\Rownum	& Нажать кнопку \keys{Enter} в области цифровых кнопок или горячую клавишу \keys{Ctrl + Enter}  & 1. Форма оплаты закроется;\par
2. На фискальном регистраторе будет напечатан чек возврата;\par
3. Табличная часть <<Товары>> в форме рабочего места кассира очистится
&  \\
\hline


\Rownum & Закрыть рабочее место кассира нажав последовательно горячие клавиши \keys{F10} - \keys{F12}  &1.  Открылось меню <<Рабочего места кассира>>;\par
2. В меню доступны все кнопки   &  \\
\hline
\Rownum & Нажать кнопку \keys{Закрытие смены}   & Открылось окно информации с вопросом: <<Закрыть смену?>>
&  \\
\hline
\Rownum & Нажать кнопку \keys{Да} в левом нижнем углу окна информации  & 1. Открылось окно с заголовком <Проверка непробитых чеков: Закрытие кассовой смены>>\par
2. На форме присутствуют кнопки: <<Отчет без гашения>>, <Закрытие смены>>, <<Отмена закрытия смены>>
3. Присутствует информациях о суммах продаж и сторно по данным системы.
&  \\
\Rownum & Нажать кнопку \keys{Закрытие смены}   & Открылось окно ошибки с сообщением о невозможности провести сверку итогов
&  \\
\hline
\Rownum & Нажать кнопку \keys{Закрыть}  на форме с информацией об ошибке  & Форма с сообщением об ошибке закрылась
&  \\
\hline
\Rownum & Закрыть форму <<Проверка не пробитых чеков>>   & 1. Зарыта форма <<Проверка не пробитых чеков>>;\par
2. На экране форма  с заголовком <<РМК (управляемый режим)>>
&  \\
\hline
\Rownum & В поле ввода <<Общая сумма (выручка, без всяких вычетов), 1>> ввести значение равное сумме всех чеков   & В поле ввода <<Общая сумма (выручка, без всяких вычетов), 1>> введено равное сумме всех чеков &  \\
\hline
\Rownum & В поле ввода <<Общая сумма возвратов (если были), 3>> ввести значение равное сумме чека возврата  & В поле ввода <<<Общая сумма возвратов (если были), 3>> введено равное сумме чека возврата &  \\
\hline



\Rownum &  Нажать кнопку \keys{Готово} в левом нижнем углу формы & 1. Форма <<РМК (управляемый режим)>> закрыта;\par
2. Открыто информационное окно <<Кассовая смена в системе закрыта:>>;\par
3. В текстовом поле формы содержится сообщение: <<Отчет о розничных продажах не сформирован>>, сумма выемки и остаток в кассе, <<Z-отчет распечатан>>
&  \\
\hline


\Rownum & Нажать кнопку \keys{Завершение работы}   & Конфигурация закрылась
&  \\
\hline


%****************************************************************************************************




\end{longtable}