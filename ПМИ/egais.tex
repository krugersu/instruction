\section{ЕГАИС}
\subsection{Объекты тестирования, описанные в разделе}

Тестирование рабочего места кассира нужно проводить на кассе и только при подключенном оборудовании ( фискальные регистратор, сканер штрихкодов, весы, табло покупателя)
\begin{longtable}{p{0.05\linewidth}p{0.4\linewidth}p{0.4\linewidth}}
    %  \toprule

   	\hline
   	1 & Вид объекта  & Документ \\
   	\hline
   	& Имя & ТТНВходящаяЕГАИС \\
   	\hline
   	& Синоним  & Товарно-транспортная накладная ЕГАИС (входящая) \\
   	\hline
%   	2 & Вид объекта  & Документ \\
%   	\hline
%   	& Имя & ТТНИсходящаяЕГАИС \\
%   	\hline
%   	& Синоним  & Товарно-транспортная накладная ЕГАИС (исходящая) \\
%   	\hline
     2 & Вид объекта  & Документ \\
    \hline
    & Имя & АктПостановкиНаБалансЕГАИС \\
    \hline
    & Синоним  & Акт постановки на баланс ЕГАИС \\
    \hline
    3 & Вид объекта  & Документ \\
    \hline
    & Имя & АктСписанияЕГАИС \\
    \hline
    & Синоним  & Акт списания ЕГАИС \\
    \hline
    4 & Вид объекта  & Документ \\
    \hline
    & Имя & ПередачаВРегистр2ЕГАИС \\
    \hline
    & Синоним  & Передача в регистр №2 ЕГАИС \\
    \hline
    5 & Вид объекта  & Документ \\
    \hline
    & Имя & ВозвратИзРегистра2ЕГАИС \\
    \hline
    & Синоним  & Возврат из регистра №2 ЕГАИС \\
    \hline
     6 & Вид объекта & Документ \\
    \hline
    %     \hline
    %    \endhead
    & Имя & Проверка алгоритмов \\
    \hline
    & Синоним  & Проверка алгоритмов \\
    \hline
    \bottomrule %%% верхняя линейка
\end{longtable}
