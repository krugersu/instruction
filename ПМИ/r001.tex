\section{Дополнительные описания}

\subsection{Работа с весами}\label{1001}

Для решения задачи автоматической "оттарки" на весах, было сделано следующее:
Написана внешняя утилита-загрузчик по работе с весами "Scales.exe". Утилита-загрузчик должна быть расположена в каталоге с базой данных.
Используется расширение \Nameref{512}.
В расширении после штатной загрузки весов происходит вызов  утилиты-загрузчика с помощью вызова штатной функции "ЗапуститьПриложение", с передачей ей в командной строке:


\begin{itemize}
	\item ПутьКФайлуЗагрузчика - Каталог с программой + имя файла утилиты-загрузчика
	\item P\_RemoteHost - Адрес весов для обмена
	\item P\_RemotePort - Порт весов для обмена
	\item P\_TimeoutUDP - Таймаут 			 
\end{itemize}

Утилита-загрузчик читает данные из весов и на каждую запись (на каждое PLU) в весах создает пять дополнительных записей, в которых указан вес тары, а номер PLU увеличивается на 1000 с каждой тарой после первой, первая увеличивается на 2000, что бы оставить свободной первую тысячу. 

\begin{itemize}
	\item контейнер маленький  0,006г PLU +2000
	\item контейнер средний    0,008г PLU +3000
	\item контейнер большой    0,012г PLU +4000	
	\item коробка фри          0,020г PLU +5000
	\item пакет бумажный       0,015г PLU +6000	 
\end{itemize}
%\footnote{}
 Затем весы полностью очищаются и затем загружаются измененные данные.Если при загрузке весы не пустые, то загрузка останавливается.
 В случае если что то пошло не так (неполные данные в весах, отсутствие позиций и пр.) Нужно повторить выгрузку.
 
 
\subsection{Себестоимость номенклатуры}\label{1002}	 

В программе отключен штатный механизм расчета себестоимости. Расчет себестоимости производит ся в собственном общем модуле "крюРасчетСебестоимости"

В данные момент расчет себестоимости происходит при проведении следующих документов:  
\begin{itemize}
	\item Оприходование товаров
	\item Перемещение товаров
	\item Поступление товаров
	\item Сборка товаров
	
\end{itemize}

\subsection{Движение возвратной тары и оборудования}\label{1003}	 

Для решения задачи учета возвратной тары и оборудования создан новый регистр накопления "ТараНаСкладах". 

Регистрация движений в регистре происходит при проведении следующих документов:  
\begin{itemize}
	\item Возврат товаров поставщику
	\item Поступление товаров
\end{itemize}

Для корректного отображения движения тары и оборудования, т.к. тара и оборудование могут придти только от конкретного поставщика и списание со склада может быть тоже только с указанием конкретного контрагента, то происходит блокировка проведения при наличии возвратной тары или оборудования в табличной части товары в  документах:

\begin{itemize}
	\item Перемещение товаров
	\item Оприходование товаров
	\item Приходный ордер на товары
	\item Расходный ордер на товары
\end{itemize}

Этот механизм реализован в общем модуле "крюРасчетСебестоимости"
	 

\subsection{Механизм сборки товаров}\label{1004}	 

Документ "СборкаТоваров" создается в момент проведения документа "ОтчетОРозничныхПродажах".
Если документ "ОтчетОРозничныхПродажах" проводится повторно, то предварительно, ранее созданные документы "СборкаТоваров" на основании текущего, удаляются и создаются новые. Этот механизм позволяет актуализировать возможные изменения в документе "ОтчетОРозничныхПродажах".

Документ "СборкаТоваров" создается на каждую строку номенклатуры табличной части "Товары", документа  "ОтчетОРозничныхПродажах", которая имеет запись в регистре сведений "КомплектующиеНоменклатуры" (кулинария).
Документ "СборкаТоваров" приходует на склад нужное количество комплектующих для корректной продажи конечной позиции номенклатуры. Так же в момент проведения документа "СборкаТоваров" расчитывается себестоимость итоговой позиции на основании себестоимости комплектующих.
При этом документ "СборкаТоваров" записывается по времени раньше, чем документ "ОтчетОРозничныхПродажах". Это сделано для корректного списания остатков

