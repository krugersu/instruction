\subsection{Формирование документа Товарно-транспортная накладная ЕГАИС (входящая)}

\begin{warning}
    \textbf{Внимание!!!! Тестирование работы ЕГАИС производится в сотрудничестве со специалистом заказчика. (если не будет предоставлен дополнительный доступ к тестовым серверам) \\
        Упоминание в тексте заказчика означает необходимость выполнения действий с его стороны.}
\end{warning}



\renewcommand{\arraystretch}{1.8} %% расстояние между строками таблицы
%\begin{landscape}
\begin{longtable}{|p{0.02\linewidth}|p{0.3\linewidth}|p{0.3\linewidth}|p{0.3\linewidth}|}
    %  {|c|c|l|c|}
    \hline
    № & \textbf{Действие} & \textbf{Ожидаемый результат} & \textbf{Фактический результат} \\
    %****************************************************************************************************
   \hline
   \hline
   \Rownum & Запустить конфигурацию магазина  & 1.Открылся общий интерфейс программы;\par
   2. Отображаются все доступные разделы  &  \\
   \hline
   \Rownum & Перейти в раздел <<Закупки>>   & 1. Открылся отдел <<Закупки>>
   &  \\

   \hline
   \Rownum	& Выбрать пункт <<Входящие ТТН>>  & Открылся журнал документов <<Входящие товарно-транспортные накладные ЕГАИС>>   &  \\
    \Rownum	& Нажать кнопку \keys{Выполнить обмен} в шапке жернала & 1. Выполнился обмен с транспортным модулем ЕГАИС;\par
    2. Произошла загрузка документов <<Товарно-транспортная накладная ЕГАИС (входящая)>>;\par
    3. Статус у загруженных документов <<Принят из ЕГАИС>> ;\par
    4. Значение в колонке <<Дальнейшее действие>> <<Выполните проверку>>  &  \\
    \hline
    \Rownum	& Открыть первый из  поступивших документов <<Товарно-транспортная накладная ЕГАИС (входящая)>>  & 1. Открыта форма документа <<Товарно-транспортная накладная ЕГАИС (входящая)>>;\par
    2. В шапке документы надписи имеют следующие значения: <<Статус: Принят из ЕГАИС>>, <<Выполните проверку или откажитесь от накладной>> &   \\
    \hline
    \Rownum	& Переключится на вкладку <<Товары>>  & Открыта вкладка <<Товары>> &  \\
    \hline
    \Rownum	& Кликнуть на надпись <<Проверить поступившую алкогольную продукцию>> & Открыта форма документа <<Проверка поступившей алкогольной продукции>> &  \\
    \hline
    \Rownum	& Кликнуть на надпись <<Проверить поступившую алкогольную продукцию>> & Открыта форма  <<Проверка поступившей алкогольной продукции>> &  \\
    \hline
    \Rownum	& Нажать кнопку \keys{Заполнить фактическое количество}  & Колонка <<Факт>> в табличной части формы заполнена количеством равным количеству в колонке <<По документу>>&  \\
    \hline
    \Rownum	& Нажать кнопку \keys{Проверка завершена}  & 1. Форма  <<Проверка поступившей алкогольной продукции>> закрыта;\par
    2. Текст надписи <<Проверить поступившую алкогольную продукцию>> изменился на <<Результаты проверки алкогольной продукции>>;\par
    3. В шапке документа текст надписи  <<Выполните проверку или откажитесь от накладной>> изменился на <<Подтвердите получение или откажитесь от накладной>>   &  \\
    \hline
    \Rownum	& Переключится на вкладку <<Основное>>  & Открыта вкладка <<Основное>> &  \\
    \hline
    \Rownum	& Кликнуть на надпись <<Оформить поступление>> справ от поля ввода <<Документ поступления>> & 1. Открыта форма нового документа <<Поступление товаров>>;\par
    2. Шапка и табличная часть заполнена данными из текущего документа <<Товарно-транспортная накладная ЕГАИС (входящая)>>;\par
    3. Количество в табличной части корректно пересчитано из даллов в литры &  \\
    \hline
    \Rownum	& Нажать кнопку \keys{Провести и закрыть} & 1. Форма  документа <<Поступление товаров>> закрыта;\par
    2. Открыта форма текущего документа <<Товарно-транспортная накладная ЕГАИС (входящая)>> &  \\
    \hline

    \Rownum	& Кликнуть на надпись <<Подтвердите получение>> & Текст надписей <<Статус: Принят из ЕГАИС>> и <<Подтвердите получение или откажитесь от накладной>> изменился на <<Статус: К подтверждению (передан в УТМ)>>  &  \\
    \hline

    \Rownum	& Нажать кнопку \keys{Провести и закрыть} & Форма  документа  <<Товарно-транспортная накладная ЕГАИС (входящая)>> закрыта&  \\
    \hline
    \Rownum	& Нажать кнопку \keys{Выполнить обмен} & 1. Статус  документа  <<Товарно-транспортная накладная ЕГАИС (входящая)>> сменился на <<К подтверждению (передан в УТМ)>>, а значение в колонке <<Дальнейшее действие>> на <<Ожидайте получения квитанции получен ЕГАИС>>&  \\
    \hline
    \Rownum	& Попросите специалиста заказчика подтвердить накладную & &  \\
    \hline
    \Rownum	& Нажать кнопку \keys{Выполнить обмен} & 1. Статус  документа  <<Товарно-транспортная накладная ЕГАИС (входящая)>> сменился на <<Подтвержден>>, а значение в колонке <<Дальнейшее действие>> на <<Не требуется>>&  \\
    \hline


    %****************************************************************************************************


\end{longtable}