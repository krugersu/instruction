\section{Недописанное и шаблоны}
\subsection{Добавить в РМК (управляемый режим)}
\renewcommand{\arraystretch}{1.8} %% расстояние между строками таблицы
%\begin{landscape}
\begin{longtable}{|p{0.02\linewidth}|p{0.3\linewidth}|p{0.3\linewidth}|p{0.3\linewidth}|}
 \multicolumn{4}{|c|}{\textbf{\textit{Проверка блокировки возврата покупателю при сложных чеках и оплате по безналичному расчету}}} \\
   \hline
   \hline
   \Rownum & Запустить конфигурацию магазина выбрав пользователя <<Абрамовская Е. (кассир)>> & 1.Открылся общий интерфейс программы;\par
   2. Отображаются разделы <<Главное>> и <<Продажи>>;\par
   3. Открылась обработка <<Рабочее место кассира>>  &  \\
   \hline
   \Rownum	& Нажать кнопку \keys{Регистрация продаж} в меню РМК & 1. Форма меню РМК закрыта;\par
   2. Открыта форма с информационным сообщением для кассиров;\par
   3. Кнопка \keys{ОК} в нижней части формы недоступна &  \\
   \hline
   \Rownum	& Отметить чек бокс с надписью <<Мною прочитано и понято>> & 1. Чек бокс с надписью <<Мною прочитано и понято>> отмечен ;\par
   2. Кнопка \keys{ОК} в нижней части формы доступна &   \\
   \hline
   \Rownum	& Нажать кнопку \keys{ОК} в нижней части формы & 1. Форма с информационным сообщением для кассиров закрыта.;\par
   2. Открыта форма Рабочего места кассира  &  \\
   \hline
   \Rownum	& Нажать кнопку \keys{Поиск (F11)} в верхней части формы или горячую клавишу \keys{F11} & Открыта форма поиска и подбора товара в РМК &  \\
   \hline
   \Rownum	& Выбрать поиск по наименованию в выпадающем списке <<Поиск>> верхней части формы  & Выбран режим поиска по наименованию &  \\
   \hline
   \Rownum	& В поле поиска ввести <<Гренки чесночные>>  & В табличной части <<Товары>> осталась номенклатура, в наименовании которой содержится <<Гренки чесночные>> &  \\
   \hline
   \Rownum	& В табличной части <<Товары>> выбрать позицию с артикулом <<11432>>  & 1. Форма поиска закрылась;\par
   2. В табличную часть <<Товары>> формы рабочего места кассира добавлена позиция с артикулом <<11432>> с количеством <<1>> и установленной ценой &  \\
   \Rownum	& Установить количество позиции с артикулом <<11432>> равным <<0,150>>  & 1. Количество позиции с артикулом <<11432>> - <<Гренки чесночные>> изменилось на значение <<0,150>>&  \\
   \hline
   \Rownum	& Нажать кнопку \keys{Поиск (F11)} в верхней части формы или горячую клавишу \keys{F11} & Открыта форма поиска и подбора товара в РМК &  \\
   \hline
   \Rownum	& Выбрать поиск по наименованию в выпадающем списке <<Поиск>> верхней части формы  & Выбран режим поиска по наименованию &  \\
   \hline
   \Rownum	& В поле поиска ввести <<Трое в лодке светлое>>  & В табличной части <<Товары>> осталась номенклатура, в наименовании которой содержится <<Трое в лодке светлое>> &  \\
   \hline
   \Rownum	& В табличной части <<Товары>> выбрать позицию с артикулом <<11697>>  & 1. Форма поиска закрылась;\par
   2. В табличную часть <<Товары>> формы рабочего места кассира добавлена позиция с артикулом <<11697>> с количеством <<1>> и установленной ценой &  \\
   \hline
   \Rownum	& Нажать \keys{Ctrl} + \keys{M}   & 1. В табличную часть <<Товары>> формы рабочего места кассира добавлена позиция с артикулом <<10340>> - <<ПЭТ бутылка 1,5л>>  &  \\
   \hline
   \Rownum	& Установить количество позиции с артикулом <<10340>> равным <<1>>  & 1. Количество позиции с артикулом <<11697>> - <<Пиво Трое в лодке светлое 1л>> изменилось на значение <<1.5>>&  \\
   \hline
   \Rownum	& Нажать кнопку \keys{Оплата (F8)} в верхней части формы или горячую клавишу \keys{F8}  &  Открыта форма оплаты &  \\
   \hline
   \Rownum	& Нажать кнопку \keys{Нал.(F6)} справа от поля ввода <<Всего к оплате (руб):>> или горячую клавишу \keys{F6}  & В табличную часть <<Виды оплат>> добавлена строка со значениями полей: <<Вида оплаты>> - <<Наличные>>; <<Сумма>> - рассчитанной суммой&  \\
   \hline
   \Rownum	& Нажать кнопку \keys{Enter} в области цифровых кнопок или горячую клавишу \keys{Ctrl + Enter}  & 1. Форма оплаты закроется;\par
   2. На фискальном регистраторе будет напечатано два чека;\par
   3. Табличная часть <<Товары>> в форме рабочего места кассира очистится;\par
   &  \\
   \if 0
   Дописать возврат, вызов формы выбора чека и получения ошибки при двойном чеке и безнале
   \fi
   \hline
   %****************************************************************************************************



     \multicolumn{4}{|c|}{\textbf{\textit{Анализ чека на наличие в нем сообщения об ошибке}}} \\
   \hline
   \hline
   \Rownum & Запустить конфигурацию магазина выбрав пользователя <<Абрамовская Е. (кассир)>> & 1.Открылся общий интерфейс программы;\par
   2. Отображаются разделы <<Главное>> и <<Продажи>>;\par
   3. Открылась обработка <<Рабочее место кассира>>  &  \\

   \hline
   %****************************************************************************************************


   \multicolumn{4}{|c|}{\textbf{\textit{Алгоритм фиксации времени прохождения этапов набора, проведения и пробития чека}}} \\
   \hline
   \hline
   \Rownum & Запустить конфигурацию магазина выбрав пользователя <<Абрамовская Е. (кассир)>> & 1.Открылся общий интерфейс программы;\par
   2. Отображаются разделы <<Главное>> и <<Продажи>>;\par
   3. Открылась обработка <<Рабочее место кассира>>  &  \\

   \hline
   %****************************************************************************************************

    \multicolumn{4}{|c|}{\textbf{\textit{При возникновении ошибок прежде чем пробивать новый чек, необходимо закончить работу со старым}}} \\
   \hline
   \hline
   \Rownum & Запустить конфигурацию магазина выбрав пользователя <<Абрамовская Е. (кассир)>> & 1.Открылся общий интерфейс программы;\par
   2. Отображаются разделы <<Главное>> и <<Продажи>>;\par
   3. Открылась обработка <<Рабочее место кассира>>  &  \\

   \hline
   %****************************************************************************************************


     \multicolumn{4}{|c|}{\textbf{\textit{Изменена основная форма РМК, уменьшен размер}}} \\
   \hline
   \hline
   \Rownum & Запустить конфигурацию магазина выбрав пользователя <<Абрамовская Е. (кассир)>> & 1.Открылся общий интерфейс программы;\par
   2. Отображаются разделы <<Главное>> и <<Продажи>>;\par
   3. Открылась обработка <<Рабочее место кассира>>  &  \\

   \hline
   %****************************************************************************************************
\end{longtable}
