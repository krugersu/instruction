\subsection{Формирование документа Акт постановки на баланс ЕГАИС}

\begin{longtable}{|p{0.02\linewidth}|p{0.3\linewidth}|p{0.3\linewidth}|p{0.3\linewidth}|}
    %  {|c|c|l|c|}
    \hline
    № & \textbf{Действие} & \textbf{Ожидаемый результат} & \textbf{Фактический результат} \\
    %****************************************************************************************************
    \hline
    \hline
    \endhead
    \multicolumn{4}{|c|}{\textbf{\textit{Проверка на наличие и функцию кнопки<<Акт постановки на баланс ЕГАИС>> }}} \\
    \hline

    \hline
    \Rownum &  Перейти в раздел Склад, выбрать <<Акты постановки на баланс>>.  & 1. Открылся список документов  <<Акт постановки на баланс ЕГАИС>>;\par
    2. Отображаются все документы &  \\
    \hline
    \Rownum & Открыть существующий документ  & 1. Открылась форма существующего документа
    &  \\

    \hline
    \Rownum	& Убедится в наличии кнопки  \keys{Акт постановки на баланс ЕГАИС}   & Кнопка  \keys{Акт постановки на баланс ЕГАИС} присутствует  &  \\
    \hline
    \Rownum	& Нажать на кнопку  \keys{Акт постановки на баланс ЕГАИС}   & Сформирована печатная форма <<Акт постановки на баланс ЕГАИС>>  &  \\
    \hline
    %****************************************************************************************************



    %****************************************************************************************************

    \multicolumn{4}{|c|}{\textbf{\textit{Движение по регистру <<Остатки алкогольной продукции в торговом зале ЕГАИС>>}}} \\
    \hline

    \hline
    \Rownum &  Перейти в раздел Склад, выбрать <<Акты постановки на баланс>>.  & 1. Открылся список документов  <<Акт постановки на баланс ЕГАИС>>;\par
    2. Отображаются все документы &  \\
    \hline
    \Rownum & Открыть существующий документ  & 1. Открылась форма существующего документа
    &  \\
    \hline
    \Rownum	& Нажать кнопку \keys{Провести} &  Документ проводится без ошибок &  \\
    \hline
    \Rownum	& Выбрать команду <<Движения документа>> & Откроется отчет по движениям документа &  \\
    \hline
    \Rownum	& Найти в отчете движения по регистру <<Остатки алкогольной продукции в торговом зале ЕГАИС>> & Движения документа по регистру сведений <<Остатки алкогольной продукции в торговом зале ЕГАИС>> присутствуют. Измерения <<Период>>, <<Организация ЕГАИС>>,<<Склад>>, <<Номенклатура>>, <<Алкогольная продукция>>, <<Справка Б>>, <<Документ приход>> заполнены. Значение ресурса <<Количество упаковок>> заполнено  &  \\
    \hline
    %****************************************************************************************************
\end{longtable}
