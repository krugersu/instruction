\subsection{Формирование документа Отчет о розничных продажах}

\renewcommand{\arraystretch}{1.8} %% расстояние между строками таблицы
%\begin{landscape}
%****************************************************************************************************
\begin{longtable}{|p{0.02\linewidth}|p{0.3\linewidth}|p{0.3\linewidth}|p{0.3\linewidth}|}
    %  {|c|c|l|c|}
    \hline
    № & \textbf{Действие} & \textbf{Ожидаемый результат} & \textbf{Фактический результат} \\
    \hline
    \hline
    \endhead
    \multicolumn{4}{|c|}{\textbf{\textit{Проверка блокировки проведения документа в центральном узле}}} \\
    \hline
    \hline
    \Rownum & Запустить конфигурацию главного узла РИБ  & Запущен центральный узел &  \\
    \hline
    \Rownum &Перейти в раздел <<Продажи>>, выбрать <<Отчеты о розничных продажах>>.  & 1. Открылся список документов  <<Отчеты о розничных продажах>>;\par
    2. Отображаются все документы &  \\
    \hline
    \Rownum & Открыть любой проведенный документ & Открылась форма проведенного документа;\par
    &  \\
    \hline
    \Rownum & Нажать кнопку \keys{Провести и закрыть} & 1. Выходит сообщение об ошибке <<Документ Отчет о розничных продажах можно проводить только в том узле, в котором он создан>> ;\par
    2. Документ не проводится  &  \\
    \hline

    %****************************************************************************************************


\end{longtable}
%****************************************************************************************************


%****************************************************************************************************
\begin{longtable}{|p{0.02\linewidth}|p{0.3\linewidth}|p{0.3\linewidth}|p{0.3\linewidth}|}
    %  {|c|c|l|c|}
    \hline
    № & \textbf{Действие} & \textbf{Ожидаемый результат} & \textbf{Фактический результат} \\
    \hline
    \hline
    \endhead
    \multicolumn{4}{|c|}{\textbf{\textit{Проверка проведения документа в узле магазина}}} \\
    \hline
    \hline
    \Rownum & Запустить конфигурацию  узла магазина  & Запущен центральный узел &  \\
    \hline
    \Rownum &Перейти в раздел <<Продажи>>, выбрать <<Отчеты о розничных продажах>>.  & 1. Открылся список документов  <<Отчеты о розничных продажах>>;\par
    2. Отображаются все документы &  \\
    \hline
    \Rownum & Открыть любой проведенный документ & Открылась форма проведенного документа;\par
    &  \\
    \hline
    \Rownum & Нажать кнопку \keys{Провести и закрыть} &  Документ  проводится  &  \\
    \hline

\end{longtable}
%****************************************************************************************************


%****************************************************************************************************
\begin{longtable}{|p{0.02\linewidth}|p{0.3\linewidth}|p{0.3\linewidth}|p{0.3\linewidth}|}
    %  {|c|c|l|c|}
    \hline
    № & \textbf{Действие} & \textbf{Ожидаемый результат} & \textbf{Фактический результат} \\
    \hline
    \hline
    \endhead
    \multicolumn{4}{|c|}{\textbf{\textit{Проверка создания документа <<Сборка товаров>>}}} \\
    \hline
    \hline
    \Rownum & Запустить конфигурацию главного узла РИБ  & Запущен ЦУ узел &  \\
    \hline
    \Rownum &Перейти в раздел <<Продажи>>, выбрать <<Отчеты о розничных продажах>>.  & 1. Открылся список документов  <<Отчеты о розничных продажах>>;\par
    2. Отображаются все документы &  \\
    \hline
    \Rownum & Создать новый документ по кнопке \keys{Создать}  & 1. Открылась форма создания документа;\par
    2. По умолчанию в открывшейся форме заполнено только поле <<Дата>> &  \\
    \hline
    \Rownum & Заполнить реквизит <<Касса (ККМ)>> значением с кодом <<ШТ-000004>>, <<75\_№2 (Кулинария) Шмидта 9>> &Заполнен реквизит <<Касса (ККМ)>> и <<Магазин>> ;    &  \\
    \hline
    \Rownum	& Нажать кнопку <<Добавить>> в табличной части <<Товары>>  & Открылась форма выбора справочника <<Номенклатура>>  &  \\
    \hline

    \Rownum	& Выбрать из справочника <<Номенклатура>> элемент с кодом <<КН000793>> - <<Гренки чесночные>> & Заполнились поля в табличной части <<Код>>, <<Артикул>>, <<Номенклатура>>, <<Ед.изм>>,<<Цена>>,<<Склад>>, <<НДС>> &  \\
    \hline
    \Rownum	&Заполнить поле <<Количество>> значением <<0,200>>  & Заполнилось поле <<Количество>> &  \\
    \hline
    \Rownum	& Нажать кнопку \keys{Провести} &  Документ проводится без ошибок &  \\
    \hline
    \Rownum	& Выбрать команду <<Связанные документы>> & Открылся отчет <<Связанные документы>> &  \\
    \hline
    \Rownum	& Найти в отчете документ <<Сборка товаров>>, открыть его & 1. Документ <<Сборка товаров>> открыт;\par
    2. Все реквизиты заполнены ;\par
    3. Значение реквизита <<Номенклатура>>  равно <<Гренки чесночные>>;\par
    4. Значение реквизита <<Количество>> равно <<0,200>>;\par
    5. Документ проведен. &  \\
    \hline
    \Rownum	& Выбрать команду <<Движения документа>> & Открылся отчет по движениям документа &  \\
    \hline
    \Rownum	& Найти в отчете движения по регистру <<Себестоимость номенклатуры>> & Движения документа по регистру сведений <<Себестоимость номенклатуры>> присутствуют. Измерения <<Период>>, <<Магазин>>, <<Номенклатура>> заполнены. Значение ресурса <<Цена>> заполнено  &  \\
    \hline

\end{longtable}
%****************************************************************************************************



%****************************************************************************************************
\begin{longtable}{|p{0.02\linewidth}|p{0.3\linewidth}|p{0.3\linewidth}|p{0.3\linewidth}|}
    %  {|c|c|l|c|}
    \hline
    № & \textbf{Действие} & \textbf{Ожидаемый результат} & \textbf{Фактический результат} \\
    \hline
    \hline
    \endhead
    \multicolumn{4}{|c|}{\textbf{\textit{Проверка формата даты в форме списка}}} \\
    \hline
    \hline
    \Rownum & Запустить конфигурацию  узла магазина  & Запущен узел магазина &  \\
    \hline
    \Rownum &Перейти в раздел <<Продажи>>, выбрать <<Отчеты о розничных продажах>>.  & 1. Открылся список документов  <<Отчеты о розничных продажах>>;\par
    2. Отображаются все документы &  \\
    \hline
    \Rownum & Убедиться, что в колонке <<Дата>> отображается дата документа вместо со временем. & В колонке <<Дата>> отображается дата документа вместо со временем;\par
    &  \\
     \hline
\end{longtable}
%****************************************************************************************************



%****************************************************************************************************
\begin{longtable}{|p{0.02\linewidth}|p{0.3\linewidth}|p{0.3\linewidth}|p{0.3\linewidth}|}
    %  {|c|c|l|c|}
    \hline
    № & \textbf{Действие} & \textbf{Ожидаемый результат} & \textbf{Фактический результат} \\
    \hline
    \hline
    \endhead
    \multicolumn{4}{|c|}{\textbf{\textit{Проверка доступности запроса количества чеков}}} \\
    \hline
    \hline
    \Rownum & Запустить конфигурацию  узла магазина  & Запущен узел магазина &  \\
    \hline
    \Rownum &Перейти в раздел <<Продажи>>, выбрать <<Отчеты о розничных продажах>>.  & 1. Открылся список документов  <<Отчеты о розничных продажах>>;\par
    2. Отображаются все документы &  \\
    \hline
    \Rownum & Открыть любой проведенный документ & Открылась форма проведенного документа;\par
    &  \\
    \hline
    \Rownum & Проверить, что в шапке документа есть кнопка \keys{Определить текущее количество чеков} & Кнопка \keys{Определить текущее количество чеков} присутствует   &  \\
    \hline
     \hline
    \Rownum & Нажать кнопку \keys{Определить текущее количество чеков} & Заголовок кнопки изменится на   \keys{Количество чеков стало: <<Х>>}, где <<Х>> реальное количество чеков привязанное к данному документу    &  \\
    \hline
\end{longtable}
%****************************************************************************************************


%\begin{verbatim}
%Если  (Константы.крюКонролироватьПроведениеОРП.Получить()) И  (ПланыОбмена.ГлавныйУзел() = Неопределено) Тогда
%    Сообщить("Документ Отчет о розничных продажах можно проводить только в том узле, в котором он создан");
%    Отказ = Истина;
%КонецЕсли;
%
%\end{verbatim}

%\begin{algorithmic}[1]
%    \IF{\(i\leqslant0\)} \STATE \(i\gets1\) \ELSE
%    \IF{\(i\geqslant0\)} \STATE \(i\gets0\)
%    \COMMENT{смысла в этом алгоритме не ищите}
%    \ENDIF
%    \ENDIF
%    \ENSURE \(i\geqslant0\)
%    \FORALL{\(\xi \in \mathcal{A}\)}
%    \STATE \(\mathcal{B}\gets\xi^2\)
%    \ENDFOR
%    \RETURN \(\mathcal{B}\)
%\end{algorithmic}