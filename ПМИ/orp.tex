\subsection{Формирование документа Отчет о розничных продажах}

\renewcommand{\arraystretch}{1.8} %% расстояние между строками таблицы
%\begin{landscape}
\begin{longtable}{|p{0.02\linewidth}|p{0.3\linewidth}|p{0.3\linewidth}|p{0.3\linewidth}|}
    %  {|c|c|l|c|}
    \hline
    № & \textbf{Действие} & \textbf{Ожидаемый результат} & \textbf{Фактический результат} \\
    %****************************************************************************************************
    \hline
    \hline
    \endhead
    \multicolumn{4}{|c|}{\textbf{\textit{Проверка блокировки проведения документа в центральном узле}}} \\
    \hline
    \hline
    \Rownum & Запустить конфигурацию главного узла РИБ  & Запущен ЦУ узел &  \\
    \hline
    \Rownum &Перейти в раздел <<Продажи>>, выбрать <<Отчеты о розничных продажах>>.  & 1. Открылся список документов  <<Отчеты о розничных продажах>>;\par
    2. Отображаются все документы &  \\
    \hline
    \Rownum & Открыть любой проведенный документ & Открылась форма проведенного документа;\par
    &  \\
    \hline
    \Rownum & Нажать кнопку \keys{Провести и закрыть} & 1. Выходит сообщение об ошибке <<Документ Отчет о розничных продажах можно проводить только в том узле, в котором он создан>> ;\par
    2. Документ не проводится  &  \\
    \hline

    %****************************************************************************************************


\end{longtable}

\begin{longtable}{|p{0.02\linewidth}|p{0.3\linewidth}|p{0.3\linewidth}|p{0.3\linewidth}|}
    %  {|c|c|l|c|}
    \hline
    № & \textbf{Действие} & \textbf{Ожидаемый результат} & \textbf{Фактический результат} \\
    %****************************************************************************************************
    \hline
    \hline
    \endhead
    \multicolumn{4}{|c|}{\textbf{\textit{Проверка проведения документа в узле магазина}}} \\
    \hline
    \hline
    \Rownum & Запустить конфигурацию  узла магазина  & Запущен узел магазина &  \\
    \hline
    \Rownum &Перейти в раздел <<Продажи>>, выбрать <<Отчеты о розничных продажах>>.  & 1. Открылся список документов  <<Отчеты о розничных продажах>>;\par
    2. Отображаются все документы &  \\
    \hline
    \Rownum & Открыть любой проведенный документ & Открылась форма проведенного документа;\par
    &  \\
    \hline
    \Rownum & Нажать кнопку \keys{Провести и закрыть} &  Документ  проводится  &  \\
    \hline

    %****************************************************************************************************


\end{longtable}



%\begin{verbatim}
%Если  (Константы.крюКонролироватьПроведениеОРП.Получить()) И  (ПланыОбмена.ГлавныйУзел() = Неопределено) Тогда
%    Сообщить("Документ Отчет о розничных продажах можно проводить только в том узле, в котором он создан");
%    Отказ = Истина;
%КонецЕсли;
%
%\end{verbatim}

%\begin{algorithmic}[1]
%    \IF{\(i\leqslant0\)} \STATE \(i\gets1\) \ELSE
%    \IF{\(i\geqslant0\)} \STATE \(i\gets0\)
%    \COMMENT{смысла в этом алгоритме не ищите}
%    \ENDIF
%    \ENDIF
%    \ENSURE \(i\geqslant0\)
%    \FORALL{\(\xi \in \mathcal{A}\)}
%    \STATE \(\mathcal{B}\gets\xi^2\)
%    \ENDFOR
%    \RETURN \(\mathcal{B}\)
%\end{algorithmic}