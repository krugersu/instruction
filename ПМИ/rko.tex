\subsection{Формирование документа Расходный кассовый ордер}

\renewcommand{\arraystretch}{1.8} %% расстояние между строками таблицы
\begin{longtable}{|p{0.02\linewidth}|p{0.3\linewidth}|p{0.3\linewidth}|p{0.3\linewidth}|}
    %  {|c|c|l|c|}
    \hline
    № & \textbf{Действие} & \textbf{Ожидаемый результат} & \textbf{Фактический результат} \\
    %****************************************************************************************************
    \hline
    \hline
    \endhead
      \multicolumn{4}{|c|}{\textbf{\textit{Проверка появления реквизита <<крюБанкВноситель>>}}} \\
   \hline
   \Rownum &Перейти в раздел <<Финансы>>, выбрать <<Расходные кассовые ордера>>.  & 1. Открылся список документов  <<Расходные кассовые ордера>>;\par
   2. Отображаются все документы &  \\
   \hline
   \Rownum & Создать новый документ с видом операции <<Сдача ДС в банк>> по кнопке \keys{Создать}  & 1. Открылась форма создания документа;\par
   2. По умолчанию в открывшейся форме заполнено поле <<Операция>>, <<Дата документа>>;\par
   3. В шапке документа присутствует реквизит <<Банк-вноситель>> &  \\
   \hline
    %****************************************************************************************************



    %****************************************************************************************************

       \hline
   \hline

   \multicolumn{4}{|c|}{\textbf{\textit{Добавляется печатная форма «ПечатьПрепроводительнаяВедомостьНакладнаяКСумке»}}} \\
   \hline


   %****************************************************************************************************


   \multicolumn{4}{|c|}{\textbf{\textit{Изменено заполнение параметра печати «НаименованиеБанкаВносителя\_БИК» для инкассации.}}} \\


   %****************************************************************************************************

   \hline

   \multicolumn{4}{|c|}{\textbf{\textit{Добавлено заполнение структуры для заполнение печатных форм РКО}}} \\


   %****************************************************************************************************

   \hline

   \multicolumn{4}{|c|}{\textbf{\textit{При изменении кассы корректно заполняется значение глобальной переменной НомерДокументаКассыККМ}}} \\


   %****************************************************************************************************

   \hline

   \multicolumn{4}{|c|}{\textbf{\textit{Корректно устанавливается номер чека ККМ при печати чека "Клиент"с учетом наших двух касс}}} \\


   %****************************************************************************************************
     %****************************************************************************************************

   \hline

   \multicolumn{4}{|c|}{\textbf{\textit{Корректно устанавливается номер чека ККМ при печати чека «КлиентИнкассация» с учетом наших двух касс}}} \\

    %****************************************************************************************************

    \hline

    \multicolumn{4}{|c|}{\textbf{\textit{При открытии документа корректно заполняется значение глобальной переменной НомерДокументаКассыККМ с учетом нашего разделения на две кассы.}}} \\
    \hline

    %****************************************************************************************************
\end{longtable}
