\subsection{Формирование документа Возврат товаров поставщику}






\renewcommand{\arraystretch}{1.8} %% расстояние между строками таблицы
%\begin{landscape}
\begin{longtable}{|p{0.02\linewidth}|p{0.3\linewidth}|p{0.3\linewidth}|p{0.3\linewidth}|}
    %  {|c|c|l|c|}
    \hline
    № & \textbf{Действие} & \textbf{Ожидаемый результат} & \textbf{Фактический результат} \\
    %****************************************************************************************************
    \hline
    \hline
    \endhead
    \multicolumn{4}{|c|}{\textbf{\textit{Проверка на номенклатуру помеченную на удаление}}} \\
    \hline
    \hline
    \Rownum & Проверить, что включена константа <<крюПроверятьПомеченнуюНаУдалениеВДокументах>>  & &  \\
    \hline
    \Rownum &Перейти в раздел Закупки, выбрать <<Возвраты товаров поставщикам>>.  & 1. Открылся список документов  <<Возвраты товаров поставщикам>>;\par
    2. Отображаются все документы &  \\
    \hline
    \Rownum & Создать новый документ по кнопке \keys{Создать}  & 1. Открылась форма создания документа;\par
    2. По умолчанию в открывшейся форме заполнено поле <<Магазин>> &  \\
    \hline
    \Rownum & Заполнить реквизит <<Поставщик>> значением <<Метро>> &Заполнен <<Поставщик>> значением <<Метро>> ;    &  \\
    \hline
    \Rownum	& Нажать кнопку выбора складов & В форме выбора складов будет доступен только склад привязанный к текущему магазину  &  \\
    \hline
    \Rownum	& Выбрать склад & Заполнены реквизиты <<Склад>> и <<Организация>>  &  \\
    \hline
    \Rownum	& Заполнить реквизит <<Причина возврата>> указав в качестве значения элемент выпадающего списка <<Возврат товара>> & Реквизит <<Причина возврата>> заполнен значением <<Возврат товара>> &  \\
    \hline
    \Rownum	& Нажать кнопку <<Добавить>> в табличной части <<Товары>>  & Откроется форма выбора справочника <<Номенклатура>>  &  \\
    \hline
    \Rownum	& Выбрать из справочника <<Номенклатура>> элемент помеченный на удаление & Заполнились поля в табличной части <<Код>>, <<Артикул>>, <<Номенклатура>>, <<Ед.изм>>, <<НДС>> &  \\
    \hline
    \Rownum	&Заполнить поле <<Количество>> значением <<1>>  & Заполнилось поле <<Количество>> &  \\
    \hline
    \Rownum	& Заполнить поле <<Цена>> значением <<1>>  & Заполнилось поле <<Цена>> &  \\
    \hline
    \Rownum	& Нажать кнопку \keys{Провести и закрыть} & 1. Программа выдает сообщение о неудачи проведения документа;\par 2. При закрытии окна сообщения в строке сообщений появляется текст ошибке с информацией, что документ содержит удаленную номенклатуру с указанием номеров строк и наименований &  \\
%****************************************************************************************************

%****************************************************************************************************

    %****************************************************************************************************

    %****************************************************************************************************
    \hline
    \hline
    \multicolumn{4}{|c|}{\textbf{\textit{Учет тары}}} \\
    \hline
    \hline
    \Rownum & Проверить, что включена константа <<крюПроводитьТару>>  & &  \\
    \hline
    \Rownum &Перейти в раздел Закупки, выбрать <<Возвраты товаров поставщикам>>.  & 1. Открылся список документов  <<Возвраты товаров поставщикам>>;\par
    2. Отображаются все документы &  \\
    \hline
    \Rownum & Создать новый документ по кнопке \keys{Создать}  & 1. Открылась форма создания документа;\par
    2. По умолчанию в открывшейся форме заполнено поле <<Магазин>> &  \\
    \hline
    \Rownum & Заполнить реквизит <<Поставщик>> значением <<Метро>> &Заполнен <<Поставщик>> значением <<Метро>> ;    &  \\
    \hline
    \Rownum	& Нажать кнопку выбора складов & В форме выбора складов будет доступен только склад привязанный к текущему магазину  &  \\
    \hline
    \Rownum	& Выбрать склад & Заполнены реквизиты <<Склад>> и <<Организация>>  &  \\
    \hline
    \Rownum	& Заполнить реквизит <<Причина возврата>> указав в качестве значения элемент выпадающего списка <<Возврат кег>> & Реквизит <<Причина возврата>> заполнен значением <<Возврат кег>>  &  \\
    \hline
    \Rownum	& Нажать кнопку <<Добавить>> в табличной части <<Товары>>  & Откроется форма выбора справочника <<Номенклатура>>  &  \\
    \hline
    \Rownum	& Выбрать из справочника <<Номенклатура>> элемент с кодом <<00000013>> - <<КЕГ (50 л)>> & Заполнились поля в табличной части <<Код>>, <<Артикул>>, <<Номенклатура>>, <<Ед.изм>>, <<НДС>> &  \\
    \hline
    \Rownum	&Заполнить поле <<Количество>> значением <<1>>  & Заполнилось поле <<Количество>> &  \\
    \hline
    \Rownum	& Заполнить поле <<Цена>> значением <<1>>  & Заполнилось поле <<Цена>> &  \\
    \hline
    \Rownum	& Нажать кнопку \keys{Провести} &  Документ проводится без ошибок &  \\
    \hline
    \Rownum	& Выбрать команду <<Движения документа>> & Откроется отчет по движениям документа &  \\
    \hline
    \Rownum	& Найти в отчете движения по регистру <<Тара на складах>> & Движения документа по регистру накопления <<Тара на складах>> присутствуют. Измерения <<Период>>, <<Склад>>, <<Номенклатура>>, <<Поставщик>> заполнены. Значение ресурса <<Количество>> равно единице  &  \\
    \hline
    %****************************************************************************************************

    %****************************************************************************************************
    \hline
    \hline
    \multicolumn{4}{|c|}{\textbf{\textit{Очистка реквизита <<УчитыватьНДС>> и <<ЦенаВключаетНДС>>}}} \\
    \hline
    \hline
    \Rownum & Проверить, что включена константа <<крюОчищатьНДСВВозвратеПоставщику>>  & &  \\
    \hline
    \Rownum &Перейти в раздел Закупки, выбрать <<Возвраты товаров поставщикам>>.  & 1. Открылся список документов  <<Возвраты товаров поставщикам>>;\par
    2. Отображаются все документы &  \\
    \hline
   \Rownum & Создать новый документ по кнопке \keys{Создать}  & 1. Открылась форма нового документа;\par
   2. По умолчанию в открывшейся форме заполнено поле <<Магазин>>\par
   3. Значение реквизитов <<ЦенаВключаетНДС>> и <<УчитыватьНДС>> на вкладке  <<Дополнительно>> равно <<Ложь>> &  \\
   \hline
    %****************************************************************************************************

     %****************************************************************************************************
    \hline
    \hline
    \multicolumn{4}{|c|}{\textbf{\textit{Печатная форма ТОРГ 12}}} \\
    \hline
    \hline
    \Rownum &Перейти в раздел Закупки, выбрать <<Возвраты товаров поставщикам>>.  & 1. Открылся список документов  <<Возвраты товаров поставщикам>>;\par
    2. Отображаются все документы &  \\
    \hline
    \Rownum & Открыть любой существующий документ  & 1. Открылась форма существующего документа;\par
     &  \\
     \hline
    \Rownum	& По кнопке выбора печатных форм выбрать <<ТОРГ-12(Товарная накладная на возврат)>>  & Сформировалась печатная форма &  \\
    \hline
    \Rownum	& Проверить представление организации и поле <<Вид операции>> в шапке & 1. В представлении организации присутствует КПП организации\par
    2. Поле <<Вид операции>> заполнено значением <<Возврат>>  &  \\


    %****************************************************************************************************





    \hline
    \Rownum	& test &  &  \\ %\nopagebreak Для запрещения разбиения страниц применяется команда \nopagebreak сразу после двух слешей в конце строчки.
    \hline
\end{longtable}