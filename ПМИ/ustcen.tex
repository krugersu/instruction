\subsection{Формирование документа Установка цен номенклатуры}

\renewcommand{\arraystretch}{1.8} %% расстояние между строками таблицы
%\begin{landscape}
\begin{longtable}{|p{0.02\linewidth}|p{0.3\linewidth}|p{0.3\linewidth}|p{0.3\linewidth}|}
    %  {|c|c|l|c|}
    \hline
    № & \textbf{Действие} & \textbf{Ожидаемый результат} & \textbf{Фактический результат} \\
    %****************************************************************************************************
    \hline
    \hline
    \endhead
    \multicolumn{4}{|c|}{\textbf{\textit{Механизм заполнения цен}}} \\
    \hline
       \hline
   \Rownum &  Перейти в раздел Склад, выбрать <<Возвраты из регистра №2>>.  & 1. Открылся список документов  <<Возвраты из регистра №2>>;\par
   2. Отображаются все документы &  \\
   \hline
   \Rownum & Открыть существующий документ  & 1. Открылась форма существующего документа
   &  \\
   \hline
   \Rownum	& Нажать кнопку \keys{Провести} &  Документ проводится без ошибок &  \\
   \hline
   \Rownum	& Выбрать команду <<Движения документа>> & Откроется отчет по движениям документа &  \\
   \hline
   \Rownum	& Найти в отчете движения по регистру <<Остатки алкогольной продукции в торговом зале ЕГАИС>> & Движения документа по регистру сведений <<Остатки алкогольной продукции в торговом зале ЕГАИС>> присутствуют. Измерения <<Период>>, <<Организация ЕГАИС>>,<<Склад>>, <<Номенклатура>>, <<Алкогольная продукция>>, <<Справка Б>>, <<Документ приход>> заполнены. Значение ресурса <<Количество упаковок>> заполнено  &  \\
   \hline

    %****************************************************************************************************


\end{longtable}