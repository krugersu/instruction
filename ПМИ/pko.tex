\subsection{Формирование документа Приходный кассовый ордер}

\renewcommand{\arraystretch}{1.8} %% расстояние между строками таблицы
%\begin{landscape}
\begin{longtable}{|p{0.02\linewidth}|p{0.3\linewidth}|p{0.3\linewidth}|p{0.3\linewidth}|}
    %  {|c|c|l|c|}
    \hline
    № & \textbf{Действие} & \textbf{Ожидаемый результат} & \textbf{Фактический результат} \\
    %****************************************************************************************************
    \hline
    \hline
    \endhead
    \multicolumn{4}{|c|}{\textbf{\textit{Создание на основании РКО}}} \\
    \hline
    \Rownum &Перейти в раздел <<Финансы>>, выбрать <<Расходные кассовые ордера>>.  & 1. Открылся список документов  <<Расходные кассовые ордера>>;\par
    2. Отображаются все документы &  \\
    \hline
    \Rownum & Открыть существующий документ с операцией <<Прочие расходы>>  & 1. Открылась форма существующего документа
    &  \\

    \hline
    \Rownum	& Нажать кнопку <<Создать на основании>>  & Появился выпадающий список с вариантами документов   &  \\
    \hline
    \Rownum	& Выбрать <<Приходный кассовый ордер>> & 1. Открылась форма нового документа <<Приходный кассовый ордер>>;\par
    2. Заполнены реквизиты <<Дата документа>>, <<Контрагент>>, <<Сумма документа>>, <<Операция>>;\par
    3. Создание нового документа <<Приходный кассовый ордер>> на основании текущего <<Расходного кассового ордера>> успешно &  \\
    \hline
    %****************************************************************************************************



    %****************************************************************************************************

    \multicolumn{4}{|c|}{\textbf{\textit{Возможность проведения без основания при ДДС Подотчет}}} \\
    \hline
   \Rownum &Перейти в раздел <<Финансы>>, выбрать <<Приходные кассовые ордера>>.  & 1. Открылся список документов  <<Приходные кассовые ордера>>;\par
   2. Отображаются все документы &  \\
   \hline
  \Rownum & Создать новый документ с видом операции <<Прочие доходы>> по кнопке \keys{Создать}  & 1. Открылась форма создания документа;\par
  2. По умолчанию в открывшейся форме заполнено поле <<Операция>> &  \\
  \hline
  \Rownum & Заполнить реквизит <<Контрагент>> значением  <<Вася>> тип <<Физические лица>> & Заполнен реквизит <<Контрагент>>    &  \\
  \hline
  \Rownum	& Заполнить реквизит <<Касса>> значением <<ООО "КРЮГЕР ХАУС" КОЛЬЦОВО (Новосибирск, Шмидта ул, 9) (75. Шмидта 9, Новосибирск)>> & Заполнены реквизиты: 1. <<Касса>> значением <<ООО "КРЮГЕР ХАУС" КОЛЬЦОВО (Новосибирск, Шмидта ул, 9) (75. Шмидта 9, Новосибирск)>>;\par
  2. <<Организация>> значением <<ООО "КРЮГЕР ХАУС" КОЛЬЦОВО (Новосибирск, Шмидта ул, 9)>>
    &  \\
  \hline
  \Rownum & Заполнить реквизит <<Сумма>> значением  <<1,0>>  & Заполнен  реквизит <<Сумма>>    &  \\

  \hline
  \Rownum	& Нажать кнопку <<Добавить>> в табличной части <<Расшифровка платежа>>,   & В табличную часть <<Расшифровка платежа>> добавлена строка с заполненным  реквизитом <<Сумма>>  &  \\
  \hline
  \Rownum	& Заполнить реквизит <<Статья ДДС>> в строке  значением с кодом <<ЦБ-000261>>, <<Подотчет>> & Заполнен реквизит <<Статья ДДС> значением <<Подотчет>>  &  \\
  \hline
  \Rownum	&Заполнить реквизит  <<Субконто>> в строке  значением  <<75. Шмидта 9, Новосибирск>>  & Заполнилось поле <<Субконто>> значением <<75. Шмидта 9, Новосибирск>>   &  \\
  \hline

    \Rownum	& Нажать кнопку \keys{Провести и закрыть} & 1. Программа выдает сообщение о неудаче проведения документа;\par 2. При закрытии окна сообщения в строке сообщений появляется текст ошибке с информацией: <<Возврат из подотчета можно вводить только на основании РКО!>> &  \\
    \hline
    %****************************************************************************************************
\end{longtable}
