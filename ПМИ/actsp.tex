\subsection{Формирование документа Акт списания ЕГАИС}

\renewcommand{\arraystretch}{1.8} %% расстояние между строками таблицы
%\begin{landscape}
\begin{longtable}{|p{0.02\linewidth}|p{0.3\linewidth}|p{0.3\linewidth}|p{0.3\linewidth}|}
    \hline
    № & \textbf{Действие} & \textbf{Ожидаемый результат} & \textbf{Фактический результат} \\
    %****************************************************************************************************
    \hline
    \hline
    \endhead
    \multicolumn{4}{|c|}{\textbf{\textit{Проверка на наличие и функцию кнопки <<Перезаполнить на основании ОРП>>}}} \\
    \hline

    \hline
    \Rownum &  Перейти в раздел Склад, выбрать <<Акты списания>>.  & 1. Открылся список документов  <<Акты списания ЕГАИС>>;\par
    2. Отображаются все документы &  \\
    \hline
    \Rownum & Открыть существующий документ  & 1. Открылась форма существующего документа
    &  \\

    \hline
    \Rownum	& Убедится в наличии кнопки  \keys{Перезаполнить на основании ОРП}   & Кнопка  \keys{Перезаполнить на основании ОРП} присутствует  &  \\
    \hline
    \Rownum	& Убедиться, что документ создан на основании документа <<Отчет о розничных продажах>> (проверив реквизит <<Основание>>), если это не так, выбрать другой документ & Выбран документ созданный на основании документа <<Отчет о розничных продажах>>   &  \\
    \hline
    \Rownum	& Открыть документ основание  & Открыт документ <<Отчет о розничных продажах>>  &  \\
    \hline

    \Rownum	& Удалить из табличной части <<Товары>> строки с алкогольной продукцией & Из табличной части товары документа <<Отчет о розничных продажах>> удалены строки с алкогольной продукцией &  \\
    \hline

    \Rownum	& Нажать кнопку \keys{Провести и закрыть}  & Документ <<Отчет о розничных продажах>> проведен и закрыт  &  \\

    \hline
    \Rownum	& Нажать на кнопку  \keys{Перезаполнить на основании ОРП} в документе <<Акт списания ЕГАИС>>.  & Табличная часть <<Товары>> документа  <<Акт списания ЕГАИС>>, очищена  &  \\
    \hline
%****************************************************************************************************

  \multicolumn{4}{|c|}{\textbf{\textit{Проверка на наличие и функцию кнопки<<Акт списания ЕГАИС>> }}} \\
  \hline

  \hline
  \Rownum &  Перейти в раздел Склад, выбрать <<Акты списания>>.  & 1. Открылся список документов  <<Акт списания ЕГАИС>>;\par
  2. Отображаются все документы &  \\
  \hline
  \Rownum & Открыть существующий документ  & 1. Открылась форма существующего документа
  &  \\

  \hline
  \Rownum	& Убедится в наличии кнопки  \keys{Акт списания ЕГАИС}   & Кнопка  \keys{Акт списания ЕГАИС} присутствует  &  \\
  \hline
  \Rownum	& Нажать на кнопку  \keys{Акт списания ЕГАИС}   & Сформирована печатная форма <<Акт списания ЕГАИС>>  &  \\
  \hline
  %****************************************************************************************************


%****************************************************************************************************

    \multicolumn{4}{|c|}{\textbf{\textit{Движение по регистру <<Остатки алкогольной продукции в торговом зале ЕГАИС>>}}} \\
    \hline
    \hline
    \Rownum &  Перейти в раздел Склад, выбрать <<Акты списания>>.  & 1. Открылся список документов  <<Акт списания ЕГАИС>>;\par
    2. Отображаются все документы &  \\
    \hline
    \Rownum & Открыть существующий документ  & 1. Открылась форма существующего документа
    &  \\
    \hline
    \Rownum	& Нажать кнопку \keys{Провести} &  Документ проводится без ошибок &  \\
    \hline
    \Rownum	& Выбрать команду <<Движения документа>> & Откроется отчет по движениям документа &  \\
    \hline
    \Rownum	& Найти в отчете движения по регистру <<Остатки алкогольной продукции в торговом зале ЕГАИС>> & Движения документа по регистру сведений <<Остатки алкогольной продукции в торговом зале ЕГАИС>> присутствуют. Измерения <<Период>>, <<Организация ЕГАИС>>,<<Склад>>, <<Номенклатура>>, <<Алкогольная продукция>>, <<Справка Б>>, <<Документ приход>> заполнены. Значение ресурса <<Количество упаковок>> заполнено  &  \\
    \hline
%****************************************************************************************************


    %  {|c|c|l|c|}

%****************************************************************************************************
\hline

\multicolumn{4}{|c|}{\textbf{\textit{Акт списания создается на начало дня при закрытии смены}}} \\
\hline
\hline
 \hline
\Rownum	& \cool\ &   &  \\
\hline
%****************************************************************************************************

    %  {|c|c|l|c|}

%****************************************************************************************************
\hline

\multicolumn{4}{|c|}{\textbf{\textit{При установленной константе "крюОтправлятьАктыЕГАИСПриЗакрытииСмены"Акты списания отправляются при закрытии кассовой смены}}} \\
\hline
 \hline
\Rownum	& \cool\ &   &  \\
\hline

%****************************************************************************************************



\end{longtable}