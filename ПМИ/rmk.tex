\section{РМК}
\subsection{Объекты тестирования, описанные в разделе}

Тестирование рабочего места кассира нужно проводить на кассе и только при подключенном оборудовании ( фискальные регистратор, сканер штрихкодов, весы, табло покупателя)
\begin{longtable}{p{0.05\linewidth}p{0.4\linewidth}p{0.4\linewidth}}
    %  \toprule
    \hline
    1 & Вид объекта & Обработка \\
    \hline
    %     \hline
    %    \endhead
    & Имя & РМКУправляемыйРежим \\
    \hline
    & Синоним  & РМК (управляемый режим) \\
    \hline


    \bottomrule %%% верхняя линейка
\end{longtable}

\newpage
\subsection{РМК (управляемый режим)}
\renewcommand{\arraystretch}{1.8} %% расстояние между строками таблицы
%\begin{landscape}
\begin{longtable}{|p{0.02\linewidth}|p{0.3\linewidth}|p{0.3\linewidth}|p{0.3\linewidth}|}
    %  {|c|c|l|c|}
    \hline
    № & \textbf{Действие} & \textbf{Ожидаемый результат} & \textbf{Фактический результат} \\
    %****************************************************************************************************
    \hline
    \hline
    \endhead
    \multicolumn{4}{|c|}{\textbf{\textit{Проверка запуска РМК при старте}}} \\
    \hline
    \Rownum & Запустить конфигурацию магазина  & 1.Открылся общий интерфейс программы;\par
    2. Отображаются все доступные разделы  &  \\
    \hline
    \Rownum & Перейти в раздел <<Администрирование>>   & 1. Открылся отдел <<Администрирование>>
    &  \\

    \hline
    \Rownum	& Выбрать пункт <<Настройки пользователей и прав>>  & Открылся раздел <<Настройки пользователей и прав>>   &  \\
    \hline
    \Rownum	& Выбрать пункт  <<Пользователи>> & Открылся раздел <<Пользователи>> &  \\
    \hline
    \Rownum & В списке пользователей открыть пользователя с именем  <<Абрамовская Екатерина>> & Открылась форма элемента справочника  <<Пользователи>> со значением <<Абрамовская Екатерина>> &  \\
    \hline
    \Rownum	& Перейти в редактирование закладки <<Группы>> & 1. Открылся список групп, в которые включен пользователь;\par
    2. В списке выбранна группа <<Кассиры>>  &  \\
    \hline
    \Rownum	& Снять выбор с группы <<Кассиры>> и установить выбор на группу <<Заведующие магазинами>>  & Снят выбор с группы <<Кассиры>> и установлен выбор на группу <<Заведующие магазинами>>  &  \\
    \hline
    \Rownum	& Нажать кнопку \keys{Записать}  & Изменения сохранились &  \\
    \hline
    \Rownum	& Перейти в редактирование закладки <<Основное>>  & Открылась форма с основными настройками пользователя  &  \\
    \hline
    \Rownum	& Нажать кнопку \keys{Записать и закрыть} & Закрылась форма элемента справочника  <<Пользователи>> со значением <<Абрамовская Екатерина>>  &  \\
    \hline
    \Rownum	& Закрыть конфигурацию  & Конфигурация закрылась  &  \\
    \hline
    \Rownum & Запустить конфигурацию магазина выбрав пользователя <<Абрамовская Е. (кассир)>> & 1.Открылся общий интерфейс программы;\par
    2. Отображаются все доступные разделы;\par
    3. Обработка <<Рабочее место кассира>> не открылась &  \\
    \hline
     \hline
    \Rownum & Перейти в раздел <<Администрирование>>   & 1. Открылся отдел <<Администрирование>>
    &  \\

    \hline
    \Rownum	& Выбрать пункт <<Настройки пользователей и прав>>  & Открылся раздел <<Настройки пользователей и прав>>   &  \\
    \hline
    \Rownum	& Выбрать пункт  <<Пользователи>> & Открылся раздел <<Пользователи>> &  \\
    \hline
    \Rownum & В списке пользователей открыть пользователя с именем  <<Абрамовская Екатерина>> & Открылась форма элемента справочника  <<Пользователи>> со значением <<Абрамовская Екатерина>> &  \\
    \hline
    \Rownum	& Перейти в редактирование закладки <<Группы>> & 1. Открылся список групп, в которые включен пользователь;\par
    2. В списке выбранна группа <<Заведующие магазинами>>  &  \\
    \hline
    \Rownum	& Снять выбор с группы <<Заведующие магазинами>> и установить выбор на группу <<Кассиры>>  & Снят выбор с группы <<Заведующие магазинами>> и установлен выбор на группу <<Кассиры>>  &  \\
    \hline
    \Rownum	& Нажать кнопку \keys{Записать}  & Изменения сохранились &  \\
    \hline
    \Rownum	& Перейти в редактирование закладки <<Основное>>  & Открылась форма с основными настройками пользователя  &  \\
    \hline

    \Rownum	& Нажать кнопку \keys{Записать и закрыть} & Закрылась форма элемента справочника  <<Пользователи>> со значением <<Абрамовская Екатерина>>  &  \\
    \hline
    \Rownum	& Закрыть конфигурацию  & Конфигурация закрылась  &  \\
    \hline
    \Rownum & Запустить конфигурацию магазина выбрав пользователя <<Абрамовская Е. (кассир)>> & 1.Открылся общий интерфейс программы;\par
    2. Отображаются разделы <<Главное>> и <<Продажи>>;\par
    3. Открылась обработка <<Рабочее место кассира>>  &  \\
    \hline
    %****************************************************************************************************



    %****************************************************************************************************

    \multicolumn{4}{|c|}{\textbf{\textit{Проверка возможности открытия кассовой смены}}} \\
    \hline
     \hline
    \Rownum & Запустить конфигурацию магазина выбрав пользователя <<Абрамовская Е. (кассир)>> & 1.Открылся общий интерфейс программы;\par
    2. Отображаются разделы <<Главное>> и <<Продажи>>;\par
    3. Открылась обработка <<Рабочее место кассира>>  &  \\
    \hline
    \Rownum	& Нажать кнопку \keys{Открытие смены} в меню РМК & 1. Кассовая смена открыта;\par
    2. На фискальном регистраторе напечатан чек открытия смены &  \\
    \hline
    %****************************************************************************************************


     \multicolumn{4}{|c|}{\textbf{\textit{Проверка возможности открытия кассовой смены}}} \\
    \hline
    \hline
    \Rownum & Запустить конфигурацию магазина выбрав пользователя <<Абрамовская Е. (кассир)>> & 1.Открылся общий интерфейс программы;\par
    2. Отображаются разделы <<Главное>> и <<Продажи>>;\par
    3. Открылась обработка <<Рабочее место кассира>>  &  \\
    \hline
    \Rownum	& Нажать кнопку \keys{Регистрация продаж} в меню РМК & 1. Форма меню РМК закрыта;\par
    2. Открыта форма с информационным сообщением для кассиров;\par
    3. Кнопка \keys{ОК} в нижней части формы недоступна &  \\
    \hline
    \Rownum	& Отметить чек бокс с надписью <<Мною прочитано и понято>> & 1. Чек бокс с надписью <<Мною прочитано и понято>> отмечен ;\par
    2. Кнопка \keys{ОК} в нижней части формы доступна &   \\
    \hline
    \Rownum	& Нажать кнопку \keys{ОК} в нижней части формы & 1. Форма с информационным сообщением для кассиров закрыта.;\par
    2. Открыта форма Рабочего места кассира  &  \\
    \hline
    \Rownum	& Проверить кнопки в верхней части формы & Присутствуют кнопки:\par 1. <<Меню>>;\par
    2. <<Поиск>>;\par
    3. <<Ред.строки>>;\par
    4. <<Возврат>>;\par
    5. <<Бонусы>>;\par
    6. <<Оплата>>;\par
    7. <<Проверить чеки ККМ>> &  \\
    \hline
     \Rownum	& Проверить наличие надписи <<ОстатокБонусов>> & 1. Присутствует надпись <<Остаток бонусов на карте>>;\par
    2. Справа от надписи поле ввода со значением <<0>>;\par
    3. Справа от поля ввода кнопка <<Обновить состояния дисконтного сервера>> &  \\
    \hline
    \Rownum	& Проверить состав полей в табличной части <<Товары>> & Присутствуют только поля:\par
    1. <<Артикул>>;\par
    2. <<Номенклатура>>;\par
    3. <<Количество>>;\par
    4. <<Цена>>;\par
    5. <<Сумма>> &  \\
    \hline
    \Rownum	& Нажать кнопку \keys{Поиск (F11)} в верхней части формы или горячую клавишу \keys{F11} & Открыта форма поиска и подбора товара в РМК &  \\
    \hline
    \Rownum	& Нажать на текстовую метку <<Показать информацию>> в нижней части формы  & 1. В нижней части формы отображается табличная часть;\par
    2. Под табличной частью чек бокс <<остатки>> недоступен для редактирования &  \\
    \hline
    %****************************************************************************************************
\end{longtable}
