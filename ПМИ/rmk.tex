\section{РМК}
\subsection{Объекты тестирования, описанные в разделе}

Тестирование рабочего места кассира нужно проводить на кассе и только при подключенном оборудовании ( фискальные регистратор, сканер штрихкодов, весы, табло покупателя)
\begin{longtable}{p{0.05\linewidth}p{0.4\linewidth}p{0.4\linewidth}}
    %  \toprule
    \hline
    1 & Вид объекта & Обработка \\
    \hline
    %     \hline
    %    \endhead
    & Имя & РМКУправляемыйРежим \\
    \hline
    & Синоним  & РМК (управляемый режим) \\
    \hline


    \bottomrule %%% верхняя линейка
\end{longtable}

\newpage
\subsection{РМК (управляемый режим)}
\renewcommand{\arraystretch}{1.8} %% расстояние между строками таблицы
%\begin{landscape}
\begin{longtable}{|p{0.02\linewidth}|p{0.3\linewidth}|p{0.3\linewidth}|p{0.3\linewidth}|}
    %  {|c|c|l|c|}
    \hline
    № & \textbf{Действие} & \textbf{Ожидаемый результат} & \textbf{Фактический результат} \\
    %****************************************************************************************************
    \hline
    \hline
    \endhead
    \multicolumn{4}{|c|}{\textbf{\textit{Проверка запуска РМК при старте}}} \\
    \hline
    \Rownum & Запустить конфигурацию магазина  & 1.Открылся общий интерфейс программы;\par
    2. Отображаются все доступные разделы  &  \\
    \hline
    \Rownum & Перейти в раздел <<Администрирование>>   & 1. Открылся отдел <<Администрирование>>
    &  \\

    \hline
    \Rownum	& Выбрать пункт <<Настройки пользователей и прав>>  & Открылся раздел <<Настройки пользователей и прав>>   &  \\
    \hline
    \Rownum	& Выбрать пункт  <<Пользователи>> & Открылся раздел <<Пользователи>> &  \\
    \hline
    \Rownum & В списке пользователей открыть пользователя с именем  <<Абрамовская Екатерина>> & Открылась форма элемента справочника  <<Пользователи>> со значением <<Абрамовская Екатерина>> &  \\
    \hline
    \Rownum	& Перейти в редактирование закладки <<Группы>> & 1. Открылся список групп, в которые включен пользователь;\par
    2. В списке выбранна группа <<Кассиры>>  &  \\
    \hline
    \Rownum	& Снять выбор с группы <<Кассиры>> и установить выбор на группу <<Заведующие магазинами>>  & Снят выбор с группы <<Кассиры>> и установлен выбор на группу <<Заведующие магазинами>>  &  \\
    \hline
    \Rownum	& Нажать кнопку \keys{Записать}  & Изменения сохранились &  \\
    \hline
    \Rownum	& Перейти в редактирование закладки <<Основное>>  & Открылась форма с основными настройками пользователя  &  \\
    \hline
    \Rownum	& Нажать кнопку \keys{Записать и закрыть} & Закрылась форма элемента справочника  <<Пользователи>> со значением <<Абрамовская Екатерина>>  &  \\
    \hline
    \Rownum	& Закрыть конфигурацию  & Конфигурация закрылась  &  \\
    \hline
    \Rownum & Запустить конфигурацию магазина выбрав пользователя <<Абрамовская Е. (кассир)>> & 1.Открылся общий интерфейс программы;\par
    2. Отображаются все доступные разделы;\par
    3. Обработка <<Рабочее место кассира>> не открылась &  \\
    \hline
     \hline
    \Rownum & Перейти в раздел <<Администрирование>>   & 1. Открылся отдел <<Администрирование>>
    &  \\

    \hline
    \Rownum	& Выбрать пункт <<Настройки пользователей и прав>>  & Открылся раздел <<Настройки пользователей и прав>>   &  \\
    \hline
    \Rownum	& Выбрать пункт  <<Пользователи>> & Открылся раздел <<Пользователи>> &  \\
    \hline
    \Rownum & В списке пользователей открыть пользователя с именем  <<Абрамовская Екатерина>> & Открылась форма элемента справочника  <<Пользователи>> со значением <<Абрамовская Екатерина>> &  \\
    \hline
    \Rownum	& Перейти в редактирование закладки <<Группы>> & 1. Открылся список групп, в которые включен пользователь;\par
    2. В списке выбранна группа <<Заведующие магазинами>>  &  \\
    \hline
    \Rownum	& Снять выбор с группы <<Заведующие магазинами>> и установить выбор на группу <<Кассиры>>  & Снят выбор с группы <<Заведующие магазинами>> и установлен выбор на группу <<Кассиры>>  &  \\
    \hline
    \Rownum	& Нажать кнопку \keys{Записать}  & Изменения сохранились &  \\
    \hline
    \Rownum	& Перейти в редактирование закладки <<Основное>>  & Открылась форма с основными настройками пользователя  &  \\
    \hline

    \Rownum	& Нажать кнопку \keys{Записать и закрыть} & Закрылась форма элемента справочника  <<Пользователи>> со значением <<Абрамовская Екатерина>>  &  \\
    \hline
    \Rownum	& Закрыть конфигурацию  & Конфигурация закрылась  &  \\
    \hline
    \Rownum & Запустить конфигурацию магазина выбрав пользователя <<Абрамовская Е. (кассир)>> & 1.Открылся общий интерфейс программы;\par
    2. Отображаются разделы <<Главное>> и <<Продажи>>;\par
    3. Открылась обработка <<Рабочее место кассира>>  &  \\
    \hline
    %****************************************************************************************************



    %****************************************************************************************************

    \multicolumn{4}{|c|}{\textbf{\textit{Возможность проведения без основания при ДДС Подотчет}}} \\
    \hline
    \Rownum &Перейти в раздел <<Финансы>>, выбрать <<Приходные кассовые ордера>>.  & 1. Открылся список документов  <<Приходные кассовые ордера>>;\par
    2. Отображаются все документы &  \\
    \hline
    \Rownum & Создать новый документ с видом операции <<Прочие доходы>> по кнопке \keys{Создать}  & 1. Открылась форма создания документа;\par
    2. По умолчанию в открывшейся форме заполнено поле <<Операция>> &  \\
    \hline
    \Rownum & Заполнить реквизит <<Контрагент>> значением  <<Вася>> тип <<Физические лица>> & Заполнен реквизит <<Контрагент>>    &  \\
    \hline
    \Rownum	& Заполнить реквизит <<Касса>> значением <<ООО "КРЮГЕР ХАУС" КОЛЬЦОВО (Новосибирск, Шмидта ул, 9) (75. Шмидта 9, Новосибирск)>> & Заполнены реквизиты: 1. <<Касса>> значением <<ООО "КРЮГЕР ХАУС" КОЛЬЦОВО (Новосибирск, Шмидта ул, 9) (75. Шмидта 9, Новосибирск)>>;\par
    2. <<Организация>> значением <<ООО "КРЮГЕР ХАУС" КОЛЬЦОВО (Новосибирск, Шмидта ул, 9)>>
    &  \\
    \hline
    \Rownum & Заполнить реквизит <<Сумма>> значением  <<1,0>>  & Заполнен  реквизит <<Сумма>>    &  \\

    \hline
    \Rownum	& Нажать кнопку <<Добавить>> в табличной части <<Расшифровка платежа>>,   & В табличную часть <<Расшифровка платежа>> добавлена строка с заполненным  реквизитом <<Сумма>>  &  \\
    \hline
    \Rownum	& Заполнить реквизит <<Статья ДДС>> в строке  значением с кодом <<ЦБ-000261>>, <<Подотчет>> & Заполнен реквизит <<Статья ДДС> значением <<Подотчет>>  &  \\
    \hline
    \Rownum	&Заполнить реквизит  <<Субконто>> в строке  значением  <<75. Шмидта 9, Новосибирск>>  & Заполнилось поле <<Субконто>> значением <<75. Шмидта 9, Новосибирск>>   &  \\
    \hline

    \Rownum	& Нажать кнопку \keys{Провести и закрыть} & 1. Программа выдает сообщение о неудаче проведения документа;\par 2. При закрытии окна сообщения в строке сообщений появляется текст ошибке с информацией: <<Возврат из подотчета можно вводить только на основании РКО!>> &  \\
    \hline
    %****************************************************************************************************
\end{longtable}
