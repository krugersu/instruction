%http://texblog.org/2011/09/09/10-ways-to-customize-tocloflot/

%Автоматически вписывать картинки в ширину страницы:
%\includegraphics[maxwidth=\linewidth]{foobar}

\if 0
Так можно сделать многострочный комментарий.
Это маленький хак. Его надо использовать.
\fi

\documentclass[twoside,11pt,a4paper,notitlepage]{report}

\makeatletter
\newcommand*{\toccontents}{\@starttoc{toc}}
\usepackage{pscyr}
\usepackage[T2A]{fontenc}
\usepackage[utf8]{inputenc}
\renewcommand{\rmdefault}{CMR}
\usepackage[russian]{babel}
\usepackage[dvipsnames]{xcolor}
%\usepackage{lscape}
%\textwidth=27cm
%\usepackage{rotating}% http://ctan.org/pkg/rotating
\usepackage[landscape,margin=1in]{geometry}

% \usepackage[pdftex]{lscape}

%\usepackage[backref]{hyperref}
%\usepackage{makeidx}
%\makeindex % команда для создания предметного указателя

\usepackage{listings}

% Создание индекса
%\usepackage{makeidx}
%\makeindex

%История изменений
%\usepackage{vhistory}
\usepackage[owncaptions]{vhistory}%[tocentry] включить в оглавление


\usepackage[most]{tcolorbox} % для управления цветом
\definecolor{block-gray}{gray}{0.90} % уровень прозрачности (1 - максимум)
\newtcolorbox{myquote}{colback=block-gray,grow to right by=-10mm,grow to left by=-10mm,
	boxrule=0pt,boxsep=0pt,breakable} % настройки области с изменённым фоном

%*************************
\usepackage{shorttoc}% Краткое оглавление
%*************************

%*************************
\usepackage{tocloft}% Управление оглавлением
%*************************



%Fncychar позволит выбрать несколько различных стилей, красиво оформляющих наименование глав.
%\usepackage[Glenn]{fncychap} % выбираем стиль Glenn

%Пакет titlesec позволяет вносить изменения в стандартный стиль главы, то есть переопределять его.
%\usepackage{titlesec, blindtext, color} % подключаем нужные пакеты
%\definecolor{gray75}{gray}{0.75} % определяем цвет
%\newcommand{\hsp}{\hspace{20pt}} % длина линии в 20pt
%% titleformat определяет стиль
%\titleformat{\chapter}[hang]{\Huge\bfseries}{\thechapter\hsp\textcolor{gray75}{|}\hsp}{0pt}{\Huge\bfseries}

\usepackage{listings}      % для кусков программ
% (понимает синтаксис некоторых языков, чума просто)
\usepackage{alltt}         % для того же
\usepackage{amssymb}       % прикольно для формул

\pagestyle{plain} % нумерация страниц вкл.

%http://tex.stackexchange.com/questions/181183/combine-usepackagetimes-and-fontspec-setmainfont
%http://andreyolegovich.ru/PC/LaTeX.php#base

%********************
% Набор предоставляет дополнительные математические символы, множество удобных возможностей для оформления математических формул (например, упрощённую работу с многострочными формулами) и используется почти во всех LaTeX-документах, в которых есть сколько-нибудь сложные формулы.
%********************
\usepackage{amsmath}



\usepackage{titlesec}
\usepackage{csquotes} % ещё одна штука для цитат

\usepackage{sectsty}
\subsectionfont{\Large\underline}

\usepackage{graphicx}
\usepackage{sidecap}
\usepackage{xcolor}
\usepackage{pdfpages}
\usepackage{comment}
\usepackage{textcomp}
\usepackage{wrapfig}
\usepackage{sectsty}
\usepackage{lipsum}
\usepackage{fancyhdr}
\usepackage{datetime}
\chapterfont{\centering}

\usepackage{caption}
\usepackage{subcaption}


\usepackage[T2A]{fontenc}
\usepackage{lscape}
\usepackage{makecell}
\usepackage{multicol}
\usepackage{floatrow}
\usepackage{float}
\usepackage{caption}
\usepackage{lastpage} % Allows referencing of the last page to allow footer to read: "Page [Current page] of [Total number of pages]."

\renewcommand{\familydefault}{\sfdefault} %Изменит стандартный шрифт документа на сансерифный.

\definecolor{MSBlue}{rgb}{.204,.353,.541}
\definecolor{MSLightBlue}{rgb}{.31,.506,.741}
%\titleformat*{\section}{\Large\bfseries\sffamily\color{MSBlue}}
%\titleformat*{\subsection}{\large\bfseries\sffamily\color{MSLightBlue}}
%\titleformat*{\subsubsection}{\itshape\subsubsectionfont}

\titleformat
{\chapter} % command
[display] % shape
{\bfseries\Large\itshape} % format
{Story No. \ \thechapter} % label
{0.5ex} % sep
{
	\rule{\textwidth}{1pt}
	\vspace{1ex}
	\centering
} % before-code
[
\vspace{-0.5ex}%
\rule{\textwidth}{0.3pt}
] % after-code


\titleformat{\section}[wrap]
{\normalfont\bfseries}
{\thesection.}{0.5em}{}

\titlespacing{\section}{12pc}{1.5ex plus .1ex minus .2ex}{1pc}

%*****************************************************
% Часто требуется, чтобы номер рисунка содержал в себе номер главы (вроде Рис. 1.1). Чтобы была сделана нумерация по главам, достаточно изменить счётчик рисунков в преамбуле документа вот так:
\renewcommand{\thefigure}{\thesection.\arabic{figure}}
%*****************************************************

%*****************************************************
%Если вас не устраивает вид подрисуночной подписи (например, вместо "Рис. 1:" необходимо "Рис. 1 --- "), используйте пакет caption. В частности, для установки тире в качестве разделителя, вставьте в преамбулу документа следующий код:
\RequirePackage{caption}
\DeclareCaptionLabelSeparator{defffis}{ --- }
\captionsetup{justification=centering,labelsep=defffis}
%*****************************************************

%\usepackage[colorlinks=true,linkcolor=blue]{hyperref}

%*****************************************************
%Как сделать, чтобы уравнения нумеровались независимо по главам в LaTeX?
%В преамбуле
\makeatletter \@addtoreset{equation}{section} \makeatother
\makeatletter \@addtoreset{figure}{section} \makeatother
%*****************************************************

\addto\captionsrussian{
	\def\figurename{Рисунок}
}


\usepackage[hypcap]{caption}

%*****************************************************
% Начинать секции с новой страницы
\usepackage{titlesec}
\newcommand{\sectionbreak}{\clearpage}
%*****************************************************


%*****************************************************
%Чтобы в генерированном PDF работали гиперссылки, то надо подключить модуль hyperref (и если хотите их разрисовать, то модуль по работе с цветами xcolor):

\usepackage{hyperref}
\usepackage{nameref}
% Цвета для гиперссылок
\definecolor{linkcolor}{HTML}{012b37} % цвет ссылок
\definecolor{urlcolor}{HTML}{012b37} % цвет гиперссылок
\hypersetup{pdfstartview=FitH,  linkcolor=linkcolor,urlcolor=urlcolor, colorlinks=true}
%*****************************************************

%*****************************************************
%Для удаления номеров страниц из \listoffigures
\makeatletter
\newcommand{\emptypage}[1]{%
	\cleardoublepage
	\begingroup
	\let\ps@plain\ps@empty
	\pagestyle{empty}
	#1
	\cleardoublepage}
\makeatletter
%*****************************************************


\setcounter{secnumdepth}{3}
%\usepackage{enumitem}
\usepackage[shortlabels]{enumitem}
\setlist[enumerate]{leftmargin=*,align=left,label=\thesubsection.\arabic*.}
%\usepackage{enumerate}
\usepackage{longtable}

%\renewcommand{\rmdefault}{ftm}
\renewcommand{\rmdefault}{cmr}
\renewcommand{\thesection}{\arabic{section}}
%\usepackage{enumitem}
%%% Страница
%\usepackage{extsizes} % Возможность сделать 14-й шрифт
\usepackage{geometry} % Простой способ задавать поля
\geometry{top=15mm}
\geometry{bottom=10mm}
\geometry{left=10mm}
\geometry{right=10mm}

%*****************************************************
%Here is how you can increase the space between the number and the caption in your \listoffigures. Add the following two lines before your \begin{document}:
\usepackage{tocloft}
\setlength{\cftfignumwidth}{3em}
%With the tocloft-package you can control the design of table of contents, figures and tables.
%*****************************************************

% раскомментировать, чтобы увидеть забавную картинку размещения текста
%\usepackage{layout}
% увидеть, что не сработало и раскомментировать \layout внизу
%%%%%%%%%%%%%%%%%%%%%%%%%%%%%%%%%%%%%%%%%%%%%%%%%%%%%%%%%%%%%%%%%%%%%%%%%%%%%%%%


%%%%%%%%%%%%%%%%%%%%%%%%%%%%%%%%%%%%%%%%%%%%%%%%%%%%%%%%%%%%%%%%%%%%%%%%%%%%%%%%
%%%
%%% мелочи жизни (переопределение буллетов для списков)
%%%
\renewcommand{\labelitemii}{{$\mathbf{+}$}}
\renewcommand{\labelitemiii}{{$\mathbf{++}$}}

%%%%%%%%%%%%%%%%%%%%%%%%%%%%%%%%%%%%%%%%%%%%%%%%%%%%%%%%%%%%%%%%%%%%%%%%%%%%%%%%
%%%
%%% Моя любимая настройка параметров страницы. По умолчанию колонка узковатая
%%%
\voffset=-10mm
\topmargin=0mm
\headheight=5mm
\headsep=10mm

\textheight=237mm
\footskip=10mm

\oddsidemargin=-2mm
\evensidemargin=-15mm
% регулирует расстояние sidenotes от края страницы
\hoffset=5mm
\textwidth=250mm
\marginparsep=10mm
%%%%%%%%%%%%%%%%%%%%%%%%%%%%%%%%%%%%%%%%%%%%%%%%%%%%%%%%%%%%%%%%%%%%%%%%%%%%%%%%

\usepackage{caption}

\usepackage{svn}


\pagestyle{fancy}
%\fancyfoot[]{вер. 1.05}

%\fancyhead[C]{Страница \thepage \; из \pageref{LastPage}}
%\fancyhead[RE]{\slshape\nouppercase{\rightmark}}
%\fancyhead[LO]{\slshape\nouppercase{\leftmark}}
%\fancyfoot[C]{Страница \thepage \; из \pageref{LastPage}}
%\renewcommand{\headrulewidth}{0pt}
%\renewcommand{\footrulewidth}{0pt}
%\lhead{\footnotesize \parbox{11cm}{Draft 1} }

% Allows calling chapter and section names in headers and footers.
%\renewcommand{\chaptermark}[1]{%
%	\markboth{\chaptername\ \thechapter}
%	{\noexpand\firstsubsectiontitle}}
%\renewcommand{\sectionmark}[1]{}
%\renewcommand{\subsectionmark}[1]{%
%	\markright{#1}\gdef\firstsubsectiontitle{#1}}
%\newcommand\firstsubsectiontitle{}



\lhead{\footnotesize \parbox{11cm}}
%\lfoot{\footnotesize \parbox{11cm}{\textit{2}}}
%\cfoot{}
\rhead{\footnotesize  \chaptername \ - \rightmark}
%\rfoot{\footnotesize Page \thepage\ of \pageref{LastPage}}
%\fancyfoot[C]{Страница \thepage \; из \pageref{LastPage}}
\fancyfoot{} % Clear all footer fields
\fancyfoot[RO,L] {v.\vhCurrentVersion \ \vhCurrentDate}  % Версия и дата
\fancyfoot[RO,R]{Страница \thepage \; из \pageref{LastPage}} % Page number on right in footer

%\renewcommand{\headheight}{24pt}
\setlength{\headheight}{4pt}
\renewcommand{\footrulewidth}{0pt}
%\setlength\headheight{80.0pt}
%\addtolength{\textheight}{-80.0pt}
%\chead{\includegraphics[width=\textwidth]{img/log1o.png}}
%\cfoot{\includegraphics[width=\textwidth]{img/foot.png}}

\graphicspath{{images/}}



%*****************************************************
%Номера страниц, включающие номер главы
\usepackage[auto]{chappg} %%% this is to set the page numbers as Chapter-Page.
%*****************************************************



\newdate{date}{28}{01}{2016}
\date{\displaydate{date}}
%Increase the value of tocdepth and secnumdepth. The tocdepth value determines to which level the sectioning commands are printed in the ToC (they are always included in the .toc file but ignored otherwise). The secnumdepth value determines up to what level the sectioning titles are numbered. They are LaTeX counters and you can set them using
\setcounter{tocdepth}{1}
\setcounter{secnumdepth}{4}

\renewcommand{\theenumi}{\arabic{enumi}}
\renewcommand{\labelenumi}{\arabic{enumi}}
\renewcommand{\theenumii}{.\arabic{enumii}}
\renewcommand{\labelenumii}{\arabic{enumi}.\arabic{enumii}.}
\renewcommand{\theenumiii}{.\arabic{enumiii}}
\renewcommand{\labelenumiii}{\arabic{enumi}.\arabic{enumii}.\arabic{enumiii}.}


\usepackage{titlepic}



%\titlespacing\section{0pt}{12pt plus 4pt minus 2pt}{0pt plus 2pt minus 2pt}
%\titlespacing{\subsection}{0pt}{\parskip}{-\parskip}

\def\capfigure{figure}

\def\captable{table}

\long\def\@makecaption#1#2{%

	\vskip\abovecaptionskip

	\ifx\@captype\capfigure

	\centering #1~--~#2 \par

	\else

	#1~--~#2 \par

	\fi

	\vskip\belowcaptionskip}

\setlength\abovecaptionskip{2\p@}

\setlength\belowcaptionskip{1\p@}



%%%%%%%%%%%%%%%%%%%%%%%%%%%%%%%%%%%%%%%%%%%%%%%%%%%%%%%%%%%%%%%%%%%%%%%%%%%%%%%%
%%%
%%% маленький хак, новое окружение 'algorithm' (см. использование ниже)
%%%
\newlength{\algboxsp}
\setlength{\algboxsp}{2mm}
\newsavebox{\algbox}
\newenvironment{basealgorithm}
{\begin{lrbox}{\algbox}\begin{minipage}{\textwidth}\begin{alltt}}
			{\end{alltt}\end{minipage}\end{lrbox}
	\fbox{
		\parbox{0.95\textwidth}{
			\makebox[0mm]{}
			\\[\algboxsp]
			\mbox{\hspace{\algboxsp}}
			\usebox{\algbox}
			\\[\algboxsp] } }}

\newenvironment{algorithm}[1]
{\begin{figure}[btp]\def\algcptn{\caption{#1}}\begin{basealgorithm}}
		{\end{basealgorithm}\algcptn\end{figure}}



%**********************************
% Todo notes - example from http://www.texample.net/tikz/examples/todo-notes/
\usepackage{verbatim}
\usepackage[colorinlistoftodos]{todonotes}
%**********************************

%\usepackage{sidenotes}


\usepackage{geometry}

\usepackage{snotez}



%**********************************************************

% Vertically aligning a marginnote and a section title
%**********************************************************
\usepackage{lipsum}

\usepackage{marginnote}
\reversemarginpar % To put the margin pars on the left
\renewcommand*{\marginfont}{\normalfont\normalsize}

\usepackage{tikz}
\usetikzlibrary{calc}
\usetikzlibrary{backgrounds}

\newcommand*{\Date}[4]{%
	\begin{tikzpicture}[show background rectangle,inner frame sep=0pt,text width=1cm,align=center]
	\node [fill=orange] at (0,0)                                (dayofweek)  {#1};
	\node [fill=white ] at ($(dayofweek)  +(0,-\baselineskip)$) (dayofmonth) {#2};
	\node [fill=white ] at ($(dayofmonth) +(0,-\baselineskip)$) (month)      {#3};
	\node [fill=orange] at ($(month)      +(0,-\baselineskip)$) (dayofmonth) {#4};
	\end{tikzpicture}
}

%**********************************************************
\usepackage{geometry}
\usepackage{marginnote}






\renewcommand{\cftchapfont}{\scshape}
\renewcommand{\cftsecfont}{\bfseries}
%\renewcommand{\cftfigfont}{Figure }
%\renewcommand{\cfttabfont}{Table }

\usepackage{eso-pic}


\AddToShipoutPicture{%

	\AtPageLowerLeft{%
		\hspace*{.02\textwidth}%
		\rotatebox{90}{%
			\begin{minipage}{\paperheight}
				\fontsize{6}{6}\selectfont
				%				\centering\textcopyright~\today{} ТД Крюгер
					\textcopyright~ ТД Крюгер
	%				\textcopyright~ ТД Крюгер тел.техподдержки 8-913 016 0854
			\end{minipage} %
		}
	} %
}%

%How can I put real notes in the margin?
%**********************************************************
\usepackage{xparse}
\usepackage{tikz}
\usetikzlibrary{calc,fit, decorations.pathmorphing}

\makeatletter
% http://tex.stackexchange.com/questions/39296/simulating-hand-drawn-lines
\pgfdeclaredecoration{penciline}{initial}{
	\state{initial}[width=+\pgfdecoratedinputsegmentremainingdistance,auto corner on length=1mm,]{
		\pgfpathcurveto%
		{% From
			\pgfqpoint{\pgfdecoratedinputsegmentremainingdistance}
			{\pgfdecorationsegmentamplitude}
		}
		{%  Control 1
			\pgfmathrand
			\pgfpointadd{\pgfqpoint{\pgfdecoratedinputsegmentremainingdistance}{0pt}}
			{\pgfqpoint{-\pgfdecorationsegmentaspect\pgfdecoratedinputsegmentremainingdistance}%
				{\pgfmathresult\pgfdecorationsegmentamplitude}
			}
		}
		{%TO
			\pgfpointadd{\pgfpointdecoratedinputsegmentlast}{\pgfpoint{1pt}{1pt}}
		}
	}
	\state{final}{}
}
\makeatother
\newcommand{\tikzmark}[1]{\tikz[overlay,remember picture] \node (#1) {};}
\newcommand{\CommentText}[3]{\tikzmark{#1}#3\tikzmark{#2}}
\NewDocumentCommand{\CommentPar}{%
	O{}% #1 = draw options for the referenced word
	O{}% #2 = draw options for the comment
	O{}% #3 = draw options for the connecting line
	m  % #4 = left \tikzmark name
	m  % #5 = left \tikzmark name
	m  % #6 = comment
}{%
	\begin{tikzpicture}[overlay,remember picture,decoration=penciline, thick]
	\node [shape=rectangle,inner sep=0, draw=blue, ,rounded corners=2pt, fit={(#4.south) ($(#5.north)+(0,0.75ex)$)}, decorate, #1] (Source) {};
	\node at ($(#4)!0.5!(#5)$) [blue, font=\itshape, rounded corners=5pt, decorate, #2] (Label) {#6};
	\draw [draw=red, decorate, #3] (Label) to (Source);
	\end{tikzpicture}
}

%**********************************************************
\usepackage[os=win]{menukeys}
% меняестся стиль, тени у кнопок
%**********************************************************
%\changemenucolor{gray}{txt}{named}{red} %Изменение цвета
\renewmenumacro{\keys}[>]{shadowedroundedkeys}
\renewmenumacro{\menu}{roundedmenus} % default: menus
%\newmenumacro{\button}
%**********************************************************


% белые кнопки вызов \keystroke{Ctrl}
\newcommand*\keystroke[1]{%
	\tikz[baseline=(key.base)]
	\node[%
	draw,
	fill=white,
	drop shadow={shadow xshift=0.25ex,shadow yshift=-0.25ex,fill=black,opacity=0.75},
	rectangle,
	rounded corners=2pt,
	inner sep=1pt,
	line width=0.5pt,
	font=\scriptsize\sffamily
	](key) {#1\strut}
	;
}

%%%%%%%%%%%%%%%%%%%%%%%%%%%%%%%%%%%%%%%%%%%%%%%%%%%
% Для рамки "Внимание"
%\usepackage{fourier}

\usepackage[utf8]{inputenc}
\usepackage{newunicodechar}

\newcommand\Warning{%
	\makebox[1.4em][c]{%
		\makebox[0pt][c]{\raisebox{.1em}{\small!}}%
		\makebox[0pt][c]{\color{red}\Large$\bigtriangleup$}}}%

\newunicodechar{⚠}{\Warning}



\usepackage{blindtext}
\usepackage{pifont,mdframed}

\newenvironment{warning}
{\par\begin{mdframed}[linewidth=2pt,linecolor=red]%
		\begin{list}{}{\leftmargin=1cm
				\labelwidth=\leftmargin}\item[\Large \Warning]} %				\labelwidth=\leftmargin}\item[\Large\ding{43}]}
		{\end{list}\end{mdframed}\par}

%%%%%%%%%%%%%%%%%%%%%%%%%%%%%%%%%%%%%%%%%%%%%

%%%%%%%%%%%%%%%%%%%%%%%%%%%%%%%%%%%%%%%%%%%%%%%%%%%%%%%%%%%%%%%%%%%%%%%%%%
%How to remove headers and footers for pages between chapters?
\makeatletter
\renewcommand*{\cleardoublepage}{\clearpage\if@twoside \ifodd\c@page\else
	\hbox{}%
	\thispagestyle{empty}%
	\newpage%
	\if@twocolumn\hbox{}\newpage\fi\fi\fi}
\makeatother
%%%%%%%%%%%%%%%%%%%%%%%%%%%%%%%%%%%%%%%%%%%%%%%%%%%%%%%%%%%%%%%%%%%%%%%%%%
% http://tex.stackexchange.com/questions/39017/how-to-influence-the-position-of-float-environments-like-figure-and-table-in-lat
% СЧЕТЧИКИ / COUNTERS
%    totalnumber (default 3) =Макс кол-во флоатс на странице
%                             max number of floats in a page
%    topnumber (default 2) = макс кол-во флоатс вверху страницы
%                            max number of floats in the top area
%    bottomnumber (default 1) = макс кол-во флоатс внизу страницы
%                               max number of floats in the bottom area
% РАЗМЕРЫ (доли страницы) / AREAS (use \renewcommand)
%    \topfraction (default 0.7) макс доля, проходящаяся на верх страницы
%                               maximum size of the top area
%    \bottomfraction (default 0.3)  макс доля приходящаяся на низ
%                                   maximum size of the bottom area
%    \textfraction (default 0.2)  миним доля, которая должна быть занята текстом
%                                 minimum size of the text area, i.e., the area that must not be occupied by floats
%\setlength{\intextsep}{4ex} % remove extra space above and below in-line float
%\setlength{\floatsep }{1ex} % remove extra space above and below in-line float

%% Попробуйте поизменять параметры и понаблюдайте за эффектом
%% Try changing the below parameters to see the effect
%\setcounter{totalnumber}{10}
%\setcounter{topnumber}{10}
%\setcounter{bottomnumber}{10}
%\renewcommand{\topfraction}{1}
%\renewcommand{\bottomfraction}{1}
%\renewcommand{\textfraction}{10}




\setlength{\abovecaptionskip}{-1pt}
\setlength{\belowcaptionskip}{-1pt}
%\usepackage[section]{placeins}
%\setlength{\textfloatsep}{5pt plus 1.0pt minus 2.0pt}

%\setcounter{totalnumber}{10}
% \setcounter{topnumber}{10}

%\renewcommand{\topfraction}{1}
% \renewcommand{\textfraction}{0}

%\setlength{\textfloatsep}{10pt plus 1.0pt minus 2.0pt}
%\setlength{\floatsep}{5pt plus 1.0pt minus 1.0pt}
%\setlength{\intextsep}{5pt plus 1.0pt minus 1.0pt}


%%%%%%%%%%%%%%%%%%%%%%%%%%%%%%%%%%%%%%%%%%%%%%%%%%%%%%%%%%%%%%%%%%%%%%%%%%%%%%%%%%
%\Nameref{name}
%которая печатает в виде текста гиперссылки на мишень name как название раздела, в котором находится %мишень, так и номер страницы, на которой она находится. По умолчанию ссылка на номер
%страницы печатается на английском языке:
%В разделе ‘Закладки’ on page 7 . . . В разделе \Nameref{sec:bookmarks} \dots
%Поэтому для документов на русском языке команду \Nameref надо переопределить, скажем, так

\renewcommand{\Nameref}[1]{<<\nameref{#1}>> на стр.~\pageref{#1}}
%%%%%%%%%%%%%%%%%%%%%%%%%%%%%%%%%%%%%%%%%%%%%%%%%%%%%%%%%%%%%%%%%%%%%%%%%%%%%%%%%%
%\usepackage[auto]{chappg} %%% this is to set the page numbers as Chapter-Page.
%\setlist{nolistsep} % уменьшение интервала между строками списка


\usepackage{booktabs}

\usepackage{lastpage} % Allows referencing of the last page to allow footer to read: "Page [Current page] of [Total number of pages]."

%*****************************************************
% Начинать subсекции с новой страницы

\newcommand{\subsectionbreak}{\clearpage}
%*****************************************************

%\usepackage{tikz}

%\usetikzlibrary{ arrows}
%\usetikzlibrary{mindmap,trees} % библиотеки TikZ
%\usetikzlibrary{chains, shapes.misc}
%\usetikzlibrary{positioning}
%
%\tikzset{
%    nonterminal/.style={
%        rectangle,
%        minimum size=6mm,
%        very thick,
%        draw=red!50!black!50,
%        top color=white,
%        bottom color=red!50!black!20,
%        font=\itshape},
%    terminal/.style={
%        rounded rectangle,
%        minimum size=6mm,
%        very thick,draw=black!50,
%        top color=white,bottom color=black!20,
%        font=\ttfamily}
%}
%\usetikzlibrary{mindmap}
%\pagestyle{empty}
%\usetikzlibrary{trees}
%\usepackage{verbatim}
%
%\usetikzlibrary{%
%    arrows, % стрелки
%    shapes.misc, % фигуры
%    chains, % цепочки
%    positioning, % позиционирование элементов
%    scopes, % создание дополнительных веток
%    shadows % тени
%}
%% указание цветов
%\definecolor{Gray}{RGB}{102,102,102}
%\definecolor{LightGray}{RGB}{178,178,178}
%\definecolor{Red}{RGB}{204,0,0}
%\definecolor{Pink}{RGB}{255,102,102}




%\usetikzlibrary{graphs}