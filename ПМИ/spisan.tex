\subsection{Формирование документа Списание товаров}

    \renewcommand{\arraystretch}{1.8} %% расстояние между строками таблицы
%\begin{landscape}
\begin{longtable}{|p{0.02\linewidth}|p{0.3\linewidth}|p{0.3\linewidth}|p{0.3\linewidth}|}
    %  {|c|c|l|c|}
    \hline
    № & \textbf{Действие} & \textbf{Ожидаемый результат} & \textbf{Фактический результат} \\
    %****************************************************************************************************
    \hline
    \hline
    \endhead
    \multicolumn{4}{|c|}{\textbf{\textit{Проверка на номенклатуру помеченную на удаление}}} \\
    \hline
    \hline
    \Rownum & Проверить, что включена константа <<крюПроверятьПомеченнуюНаУдалениеВДокументах>>  & &  \\
    \hline
    \Rownum &Перейти в раздел Склад, выбрать <<Списание товаров>>.  & 1. Открылся список документов  <<Списание товаров>>;\par
    2. Отображаются все документы &  \\
    \hline
    \Rownum & Создать новый документ по кнопке \keys{Создать}  & 1. Открылась форма создания документа;\par
    2. По умолчанию в открывшейся форме заполнено поле <<Магазин>> &  \\
    \hline
    \Rownum & Заполнить реквизит <<Склад>> значением соответствующим выбранному магазину &Заполнен <<Склад отправитель>> и <<Организация>> ;    &  \\
    \hline
    \Rownum	& Заполнить реквизит <<Аналитика хоз. операции>> значением <<Оприходование товаров (ВОЗВРАТ ОТ ПОКУПАТЕЛЯ)>> & Заполны реквизиты: 1. <<Магазин>> значением <<75. Шмидта 9, Новосибирск>>;\par
    2. <<Склад>> значением <<75. Шмидта 9, Новосибирск>>;\par
    3. <<Организация>> значением <<ООО "КРЮГЕР ХАУС" КОЛЬЦОВО (Новосибирск, Шмидта ул, 9)>>;\par
    4. <<Аналитика хоз. операции>> значением <<Списание на затраты (ВОЗВРАТ ОТ ПОКУПАТЕЛЯ-ЗАМЕНА)>>  &  \\
    \hline
    \Rownum	& Нажать кнопку <<Добавить>> в табличной части <<Товары>>  & Откроется форма выбора справочника <<Номенклатура>>  &  \\
    \hline
    \Rownum	& Выбрать из справочника <<Номенклатура>> элемент помеченный на удаление & Заполнились поля в табличной части <<Код>>, <<Артикул>>, <<Номенклатура>>, <<Ед.изм>> &  \\
    \hline
    \Rownum	&Заполнить поле <<Количество>> значением <<1>>  & Заполнилось поле <<Количество>> &  \\
    \hline
    \Rownum	& Заполнить поле <<Цена>> значением <<1>>  & 1. Заполнилось поле <<Цена>>;\par
    2. Заполнилось поле <<Сумма>> &  \\
    \hline
    \Rownum	& Нажать кнопку \keys{Провести и закрыть} & 1. Программа выдает сообщение о неудаче проведения документа;\par 2. При закрытии окна сообщения в строке сообщений появляется текст ошибке с информацией, что документ содержит удаленную номенклатуру с указанием номеров строк и наименований &  \\
    \hline
    %****************************************************************************************************



    %****************************************************************************************************

    \multicolumn{4}{|c|}{\textbf{\textit{Учет тары}}} \\
    %   \hline
    \hline
    \Rownum & Проверить, что включена константа <<крюПроводитьТару>>  & &  \\
    \hline
    \Rownum &Перейти в раздел Склад, выбрать <<Списание товаров>>.  & 1. Открылся список документов  <<Списание товаров>>;\par
    2. Отображаются все документы &  \\
    \hline
    \Rownum & Создать новый документ по кнопке \keys{Создать}  & 1. Открылась форма нового документа;\par
    2. По умолчанию в открывшейся форме заполнено поле <<Магазин>> &  \\
    \hline
    \Rownum & Заполнить реквизит <<Склад>> значением соответствующим выбранному магазину &Заполнен <<Склад отправитель>> и <<Организация>> ;    &  \\
    \hline
    \Rownum	& Заполнить реквизит <<Аналитика хоз. операции>> значением <<Оприходование товаров (ВОЗВРАТ ОТ ПОКУПАТЕЛЯ)>> & Заполны реквизиты: 1. <<Магазин>> значением <<75. Шмидта 9, Новосибирск>>;\par
    2. <<Склад>> значением <<75. Шмидта 9, Новосибирск>>;\par
    3. <<Организация>> значением <<ООО "КРЮГЕР ХАУС" КОЛЬЦОВО (Новосибирск, Шмидта ул, 9)>>;\par
    4. <<Аналитика хоз. операции>> значением <<Списание на затраты (ВОЗВРАТ ОТ ПОКУПАТЕЛЯ-ЗАМЕНА)>>  &  \\
    \hline
    \Rownum	& Нажать кнопку <<Добавить>> в табличной части <<Товары>>  & Откроется форма выбора справочника <<Номенклатура>>  &  \\
    \hline
    \Rownum	& Выбрать из справочника <<Номенклатура>> элемент с кодом <<00000013>> - <<КЕГ (50 л)>> & Заполнились поля в табличной части <<Код>>, <<Артикул>>, <<Номенклатура>>, <<Ед.изм>> &  \\
    \hline
    \Rownum	&Заполнить поле <<Количество>> значением <<1>>  & Заполнилось поле <<Количество>> &  \\
    \hline
    \Rownum	& Заполнить поле <<Цена>> значением <<1>>  & Заполнилось поле <<Цена>>;\par
    2. Заполнилось поле <<Сумма>> &  \\
    \hline
    \Rownum	& Нажать кнопку \keys{Провести и закрыть} & 1. Программа выдает сообщение о неудаче проведения документа;\par 2. При закрытии окна сообщения в строке сообщений появляется текст ошибке с информацией, что документ содержит возвратную тару или оборудование с указанием номеров строк &  \\
    \hline
    %****************************************************************************************************
\end{longtable}



