\section{Сценарий тестирования}

Тестирование производится под пользователем со следующими настройками:

\begin{tabular}{|c|c|}
	\hline
	Права доступа &  \\
	\hline
	Профили пользователя &  \\
	\hline
	Вид доступа &  \\
	\hline
	Значения доступа &  \\
	\hline
	Персональные настройки &  \\
	\hline
	Основная организация &  \\
	\hline
	Основное подразделение &  \\
	\hline
	Показывать счета учета в документах &  \\
	\hline
\end{tabular}

\section{Закупки}
\subsection{Объекты тестирования, описанные в разделе}

\begin{tabular}{p{0.05\linewidth}p{0.4\linewidth}p{0.4\linewidth}}
	\toprule
	%	\hline
	1 & Вид объекта & Документ \\
	\hline
	  & Имя & ПоступлениеТоваров \\
	\hline
	 & Синоним  & Поступление товаров \\
	\hline
	2 & Вид объекта  & Документ \\
	\hline
 	 & Имя & ВозвратТоваровПоставщику \\
	\hline
	 & Синоним  & Возврат товаров поставщику \\
	\hline
	3 & Вид объекта  & Документ \\
	\hline
	& Имя & ПеремещениеТоваров \\
	\hline
	& Синоним  & Перемещение товаров \\
	\hline
	4 & Вид объекта  & Документ \\
	\hline
	& Имя & ТТНВходящаяЕГАИС \\
	\hline
	& Синоним  & Товарно-транспортная накладная ЕГАИС (входящая) \\
	\hline
	5 & Вид объекта  & Документ \\
	\hline
	& Имя & ТТНИсходящаяЕГАИС \\
	\hline
	& Синоним  & Товарно-транспортная накладная ЕГАИС (исходящая) \\
%	\hline
	\bottomrule %%% верхняя линейка
\end{tabular}


\vspace{\baselineskip}
\newpage
\subsection{Формирование документа Поступление товаров}


\begin{tabular}{|c|c|c|c|}
	\hline
	№ & Действие & Ожидаемый результат & Фактический результат \\
	\hline
   1 &  &  &  \\
	\hline
	&  &  &  \\
	\hline
	&  &  &  \\
	\hline
	&  &  &  \\
	\hline
	&  &  &  \\
	\hline
	&  &  &  \\
	\hline
	&  &  &  \\
	\hline
	&  &  &  \\
	\hline
	&  &  &  \\
	\hline
\end{tabular}




%\begin{tikzpicture}
%\path[mindmap,concept color=black,text=white]
%node[concept] {Computer Science}
%[clockwise from=0]
%child[concept color=green!50!black] {
%    node[concept] {practical}
%    [clockwise from=90]
%    child { node[concept] {algorithms} }
%    child { node[concept] {data structures} }
%    child { node[concept] {pro\-gramming languages} }
%    child { node[concept] {software engineer\-ing} }
%}
%child[concept color=blue] {
%    node[concept] {applied}
%    [clockwise from=-30]
%    child { node[concept] {databases} }
%    child { node[concept] {WWW} }
%}
%child[concept color=red] { node[concept] {technical} }
%child[concept color=orange] { node[concept] {theoretical} };
%\end{tikzpicture}



%\begin{tikzpicture}
%%% Добавим стиль рисования всех узлов
%\graph[nodes={align=center,rectangle,draw=black}, grow down sep, branch right sep] {
%    Диакритика? ->
%    {
%        "Много",
%        Малао -> "чтото" ->
%        {
%            " фра? ->
%            {
%                "Ой$\dots$он",
%                вроде нет -> алба
%            },
%            "только" -> фин
%        }
%    } ->
%    Французский
%};
%\end{tikzpicture}