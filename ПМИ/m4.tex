\section{Сценарий тестирования}

Тестирование производится под пользователем со следующими настройками:

\begin{tabular}{|c|c|}
	\hline
	Права доступа &  \\
	\hline
	Профили пользователя &  \\
	\hline
	Вид доступа &  \\
	\hline
	Значения доступа &  \\
	\hline
	Персональные настройки &  \\
	\hline
	Основная организация &  \\
	\hline
	Основное подразделение &  \\
	\hline
	Показывать счета учета в документах &  \\
	\hline
\end{tabular}

\section{Закупки}
\subsection{Объекты тестирования, описанные в разделе}

\begin{tabular}{p{0.05\linewidth}p{0.4\linewidth}p{0.4\linewidth}}
	\toprule
	%	\hline
	1 & Вид объекта & Документ \\
	\hline
	  & Имя & ПоступлениеТоваров \\
	\hline
	 & Синоним  & Поступление товаров \\
	\hline
	2 & Вид объекта  & Документ \\
	\hline
 	 & Имя & ВозвратТоваровПоставщику \\
	\hline
	 & Синоним  & Возврат товаров поставщику \\
	\hline
	3 & Вид объекта  & Документ \\
	\hline
	& Имя & ПеремещениеТоваров \\
	\hline
	& Синоним  & Перемещение товаров \\
	\hline
	4 & Вид объекта  & Документ \\
	\hline
	& Имя & ТТНВходящаяЕГАИС \\
	\hline
	& Синоним  & Товарно-транспортная накладная ЕГАИС (входящая) \\
	\hline
	5 & Вид объекта  & Документ \\
	\hline
	& Имя & ТТНИсходящаяЕГАИС \\
	\hline
	& Синоним  & Товарно-транспортная накладная ЕГАИС (исходящая) \\
%	\hline
	\bottomrule %%% верхняя линейка
\end{tabular}


\vspace{\baselineskip}
\newpage
\subsection{Формирование документа Поступление товаров}

%\begin{landscape}
\begin{longtable}{|p{0.02\linewidth}|p{0.3\linewidth}|p{0.3\linewidth}|p{0.3\linewidth}|}
  %  {|c|c|l|c|}
	\hline
	№ & \textbf{Действие} & \textbf{Ожидаемый результат} & \textbf{Фактический результат} \\
	\hline
    \hline
    \endfirsthead
    \multicolumn{4}{|c|}{\textbf{\textit{Проверка на номенклатуру помеченную на удаление}}} \\
    \hline
    \hline
    \Rownum & Проверить, что включена константы <<крюПроверятьПомеченнуюНаУдалениеВДокументах>>  & &  \\
    \hline
    \Rownum &Перейти в раздел Закупки, выбрать <<Поступление товаров>>.  & 1. Открылся список документов  <<Поступление товаров>>;\par
    2. Отображаются все документы &  \\
	\hline
    \Rownum & Создать новый документ по кнопке \keys{Создать}  & 1. Открылась форма создания документа;\par
    2. По умолчанию в открывшейся форме заполнено поле <<Магазин>> &  \\
	\hline
    \Rownum & Заполнить реквизит <<Поставщик>> значением <<Метро>> &Заполнен <<Поставщик>> значением <<Метро>> ;    &  \\
	\hline
    \Rownum	& Нажать кнопку выбора складов & В форме выбора складов будет доступен только склад привязанный к текущему магазину  &  \\
	\hline
    \Rownum	& Выбрать склад & Заполнены реквизиты <<Склад>> и <<Организация>>  &  \\
	\hline
    \Rownum	& Нажать кнопку <<Добавить>> в табличной части <<Товары>> & Откроется форма выбора справочника <<Номенклатура>>  &  \\
	\hline
    \Rownum	& Выбрать из справочника <<Номенклатура>> элемент помеченный на удаление & Заполнились поля в табличной части <<Код>>, <<Артикул>>, <<Номенклатура>>, <<Ед.изм>>, <<НДС>> &  \\
	\hline
    \Rownum	&Заполнить поле <<Количество>> значением <<1>>  & Заполнилось поле <<Количество>> &  \\
	\hline
    \Rownum	& Заполнить поле <<Цена>> значением <<1>>  & Заполнилось поле <<Цена>> &  \\
	\hline
    \Rownum	& Нажать кнопку \keys{Провести и закрыть} & 1. Программа выдает сообщение о неудачи проведения документа;\par 2. При закрытии окна сообщения в строке сообщений появляется текст ошибке с информацией, что документ содержит удаленную номенклатуру с указанием номеров строк и наименований &  \\
    \hline
\Rownum	& test & resr &  \\
\hline
\Rownum	& test &  &  \\
\hline
\Rownum	& test &  &  \\
\hline

\Rownum	& test & resr &  \\
\hline
\Rownum	& test &  &  \\
\hline
\Rownum	& test &  &  \\
\hline

\Rownum	& test & resr &  \\
\hline
\Rownum	& test &  &  \\
\hline
\Rownum	& test &  &  \\
\hline

\Rownum	& test11 & resr &  \\
\hline
\Rownum	& test111 &  &  \\
\hline
\Rownum	& test &  &  \\
\hline
\end{longtable}
%\end{landscape}



%\begin{tikzpicture}
%\path[mindmap,concept color=black,text=white]
%node[concept] {Computer Science}
%[clockwise from=0]
%child[concept color=green!50!black] {
%    node[concept] {practical}
%    [clockwise from=90]
%    child { node[concept] {algorithms} }
%    child { node[concept] {data structures} }
%    child { node[concept] {pro\-gramming languages} }
%    child { node[concept] {software engineer\-ing} }
%}
%child[concept color=blue] {
%    node[concept] {applied}
%    [clockwise from=-30]
%    child { node[concept] {databases} }
%    child { node[concept] {WWW} }
%}
%child[concept color=red] { node[concept] {technical} }
%child[concept color=orange] { node[concept] {theoretical} };
%\end{tikzpicture}



%\begin{tikzpicture}
%%% Добавим стиль рисования всех узлов
%\graph[nodes={align=center,rectangle,draw=black}, grow down sep, branch right sep] {
%    Диакритика? ->
%    {
%        "Много",
%        Малао -> "чтото" ->
%        {
%            " фра? ->
%            {
%                "Ой$\dots$он",
%                вроде нет -> алба
%            },
%            "только" -> фин
%        }
%    } ->
%    Французский
%};
%\end{tikzpicture}