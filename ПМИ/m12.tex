\section{РМК}
%\marginnote{\Date{Ср.}{08}{Апр.}{2020}}[-20pt]
\subsection{Алгоритмы РМК}


\begin{itemize}
	\item Реализована разбивка основного чека на два если присутствует номенклатура с разными системами налогообложения
	\item При подборе товара скрыта колонка "остаток"
	\item Алгоритм расчета с использование карты "Халва"
	\item Алгоритм логирующий этапы пробития чека ККМ
	\item При возникновении ошибок прежде чем пробивать новый чек, необходимо закончить работу со старым
	\item Алгоритм автоматического подбора тары при покупке разливного пива
	\item Изменена основная форма РМК, уменьшен размер
	\item Изменен шаблон вывода сообщения при превышении остатка на складе, теперь показывается только склад по которому превышен остаток и не показывается на какое количество
	\item Также при проверке отрицательных остатков, анализируется реквизит номенклатуры "ОтпускатьВМинус" и если он установлен, то дальнейшая проверка на отрицательные остаток не производится
	\item Перед началом оплаты проверяется, что если до закрытия кассовой смены осталось 10 мин и меньше, то выполняется запрет продажи
 	\item При установленной константе "крюБлокировкаПараллельнойОплатыЭквайринг" оплата по безналу одновременно на двух кассах блокируется
 	\item Если чек был разбит согласно технологии на два, а оплата была по безналу, то процедура возврата блокируется
 	\item Обработка РМК должна открываться только, если пользователь входит в группу "Кассиры"
 	\item Перед открытием смены выполняется закрытие эквайринговой смены ?
 	\item В форме меню, если есть необработанные чеки блокируются все элементы кроме "Регистрации продаж" и выходит сообщение с предложением разобраться с ошибками чеков
 	\item В форме подбора товаров не показываются остатки
 	\item Анализ чека на наличие в нем сообщения об ошибке
	\item Перед началом продаж кассиру показывается информационное сообщение с необходимостью поставить "галочку", что он это сообщение прочел и понял
	\item
 	\item
 	\item
 	\item
 	\item
 	\item
 	\item
	
\end{itemize}


