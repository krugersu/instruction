\section{Описание алгоритма}

\begin{itemize}
	
\item Структура проекта

\begin{figure}[H]
	\includegraphics[width=0.95\textwidth]{code.png}
	\caption{<<Структура проекта>>.}
	\label{ris:code.png}
\end{figure}

\subsection{Алгоритм}
\item При запуске программы происходит чтение настроек. Если чтение удачно, запускается функция <<main()>>, в противном случае программа завершает свою работу.

%\begin{figure}[H]
%	\includegraphics[width=0.65\textwidth]{code1.png}
%	\caption{<<Структура проекта>>.}
%	\label{ris:code1.png}
%\end{figure}


\begin{tcolorbox}
	\begin{lstlisting}[language=Python,ndkeywordstyle=\color{darkgray}\bfseries,identifierstyle=\color{black},stringstyle=\color{red}\ttfamily,showstringspaces=false,keepspaces=true,extendedchars=\true]
if __name__ == "__main__":

	# Чтение настроек
	m_conf = m_config.m_Config()   
	rc =  m_conf.loadConfig()
	if not rc == None:
		main()
	else:
		logger.info(u'Программа завершила работу') 
	\end{lstlisting}
\end{tcolorbox}

\par

\item При запуске функции <<main()>> происходит соединение с базой данных и создается объект для работы с ней, так же создается объект для работы с Http запросами.

\begin{tcolorbox}
	\begin{lstlisting}[language=Python,ndkeywordstyle=\color{darkgray}\bfseries,identifierstyle=\color{black},stringstyle=\color{red}\ttfamily,showstringspaces=false,keepspaces=true,extendedchars=\true]
    tData = db.workDb(rc)
	rec_con = m_request.req1C(rc)
	\end{lstlisting}
\end{tcolorbox}


\item Получаем список смен которые были открыты с последней зафиксированной даты, если появились новые открытые смены, тогда формируем и отправляем Http запрос в 1С. 
Если код возврата был успешным (\textbf{200}), тогда меняем дату в файле, на дату открытия последней смены.

\begin{tcolorbox}
	\begin{lstlisting}[language=Python,ndkeywordstyle=\color{darkgray}\bfseries,identifierstyle=\color{black},stringstyle=\color{red}\ttfamily,showstringspaces=false,keepspaces=true,extendedchars=\true]
 # Список открытых смен от последнего зафиксированного времени
	l_workshift_open = tData.get_last_workshift_open()
	# Если нечего отправлять, то не отправляем
	if len(l_workshift_open) > 0:
		status_code = rec_con.post_workshift_open(l_workshift_open)
	# Меняем дату в файле только в случае успешного результата работы 1C
		if status_code == 200:
			tData.save_new_date_open()
		else:
			logger.info(u'status_code_open - ' + str(status_code ))
	\end{lstlisting}
\end{tcolorbox}


\item Получаем список смен которые были закрыты с последней зафиксированной даты, если появились новые закрытые смены, тогда формируем и отправляем Http запрос в 1С. 
Если код возврата был успешным (\textbf{200}), тогда меняем дату в файле, на дату закрытия последней смены.

\begin{tcolorbox}
	\begin{lstlisting}[language=Python,ndkeywordstyle=\color{darkgray}\bfseries,identifierstyle=\color{black},stringstyle=\color{red}\ttfamily,showstringspaces=false,keepspaces=true,extendedchars=\true]
	# Список закрытых смен от последнего зафиксированного времени
	l_workshift = tData.get_last_workshift()
	# Если нечего отправлять, то и  не отправляем
	if len(l_workshift) > 0:
		status_code = rec_con.post_workshift(l_workshift)
	# Меняем дату в файле только в случае успешного результата работы 1С
		if status_code == 200:
			tData.save_new_date()
		else:
			logger.info(u'status_code - ' + str(status_code ))
	\end{lstlisting}
\end{tcolorbox}

\item Завершаем работу программы.

\end{itemize}