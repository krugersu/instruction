\section{Установка и подготовка к использованию}

\marginnote{\Date{Пт.}{27}{Янв.}{2023}}[-40pt]
\setlist[1]{itemsep=-1pt}

\begin{itemize}
	\item На сервере должен быть установлен Python версии >= 3.11
	\item Копируем каталог <<Workshift\_load>> с программой на рабочий сервер
	\item  Заходим в каталог <<src>> и выполняем команду
	\begin{tcolorbox}
		
	\begin{lstlisting}[language=bash]
chmod +x wsh_load.py
	\end{lstlisting}
\end{tcolorbox}
%	\begin{minted}{bash}
%		$ wget http://tex.stackexchange.com
%		\end{minted}
	
	Для того, что бы сделать файл скрипта исполняемым
	
    \item Если отсутствуют, то создаем два файла <<last\_date\_open.txt>>
   и <<last\_date.txt>>.
   В каждый файл записываем любую дату ранее текущей в формате <<2023-01-25 20:31:01>>.

	\item В корневом каталоге выполняем команду для создания виртуального окружения
\begin{tcolorbox}
	
	\begin{lstlisting}[language=Python]
python3.11 -m venv .venv
	\end{lstlisting}
\end{tcolorbox}

   
	\item  Выполняем команду для установки нужных пакетов
\begin{tcolorbox}
	
	\begin{lstlisting}[language=bash]
pip install -r requirements.txt
	\end{lstlisting}
\end{tcolorbox}
    
    
    
	
	
\end{itemize}
