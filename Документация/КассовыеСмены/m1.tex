\section{Установка и подготовка к использованию}
\subsection{Начальные (текущие) системные требования}

\begin{itemize}[label=$\ast$]
	\item	Процессор - Intel(R) Atom(TM) CPU D2500 @ 1.86GHz, 2 ядра
	\item	ОЗУ - 2 Гб 
	\item	SSD - 120 ГБ
	\item	Операционная система (не ниже)  - Ubuntu 22.04 jammy
\end{itemize}	


\subsection{Установленные пакеты}

\begin{itemize}[label=$\ast$]
	\item	Python версии >= 3.11
	\item	Webmin (не обязательно) - панель для администрирования сервера 
	\item	OpenSSH (установка \ref{itm:first})
	\item	редактор Nano (установка \ref{itm:second} )
\end{itemize}	


%\marginnote{\Date{Пт.}{27}{Янв.}{2023}}[-40pt]
%\setlist[1]{itemsep=-1pt}


\subsection{Установка}

\begin{itemize}
	\item Копируем каталог <<Workshift\_load>> с программой на рабочий сервер
	\item  Заходим в каталог <<src>> и выполняем команду
	
	\commandbox*{chmod +x wsh_load.py} 
%	\begin{tcolorbox}
%		
%	\begin{lstlisting}[language=bash]
%chmod +x wsh_load.py
%	\end{lstlisting}
%\end{tcolorbox}

%	\begin{minted}{bash}
%		$ wget http://tex.stackexchange.com
%		\end{minted}
	
	Для того, что бы сделать файл скрипта исполняемым, в противном случае он не будет запускаться.  

	\item Переходим в корневой каталог
	
	\commandbox*{cd ..} 
%	\begin{tcolorbox}
%		\begin{lstlisting}[language=Python]
%	cd ..
%	\end{lstlisting}
%\end{tcolorbox}
 
	\item В корневом каталоге выполняем команду для создания виртуального окружения
	
	\commandbox*{python3.11 -m venv .venv} 
%\begin{tcolorbox}
%	
%	\begin{lstlisting}[language=Python]
%python3.11 -m venv .venv
%	\end{lstlisting}
%\end{tcolorbox}

   
	\item  Выполняем команду для установки нужных пакетов
	
	\commandbox*{pip install -r requirements.txt} 
%\begin{tcolorbox}
%	
%	\begin{lstlisting}[language=bash]
%pip install -r requirements.txt
%	\end{lstlisting}
%\end{tcolorbox}
    
    
    
	
	
\end{itemize}
